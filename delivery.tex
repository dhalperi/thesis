\ifx\mainfile\undefined
\input{chapter_head}
\setcounter{chapter}{5} % Set to n-1!
\fi
%%%%%%%%%%%%%%%%%%%%%%%%%%%%%%%%%%

\cleardoublepage
\chapter{Packet Delivery over MIMO Channels}
\label{chap:delivery}

In this chapter, I experimentally evaluate how well my Effective SNR model predicts packet delivery for 802.11n wireless links. This is the fundamental measure of whether the model can be useful; good predictions are necessary to properly configure links and networks to solve the link and network configuration problems.

To do so, I use my CSI measurement tool to gather a wide range of channel and performance information across 200 wireless links in both testbeds. This captures a wide variety of fading environments, from line-of-sight links in the same room to links between nodes in different rooms with RF barriers and reflectors spread around and between them. I then use this data to evaluate the accuracy of predictions made when using Packet SNR and Effective SNR about whether packets will be delivered.

\section{Experimental Data}
I measured packet delivery over a 20\MHz channel on my two 802.11n testbeds, using links with four different antenna configurations:
\begin{enumerate}
\item The \textbf{SISO} configuration uses a single antenna at each node. This configuration corresponds to 802.11a.
\item The \textbf{SIMO} configuration uses a single transmit antenna but three receive antennas. This is an 802.11a/g/n configuration that uses spatial diversity techniques.
\item The \textbf{MIMO2} configuration uses two spatial streams and three receive antennas. This employs both 802.11n techniques of spatial multiplexing and spatial diversity.
\item The \textbf{MIMO3} configuration uses three antennas at each node to send and receive three spatial streams. This configuration uses spatial multiplexing, but does not additionally benefit from spatial diversity.
\end{enumerate}

For each of these configurations, I measured the packet delivery for each link using each MCS, at at each transmit power level between $-$10\dBm and $+$16\dBm in steps of 2\dB. I sent 1,500-byte packets as constant bit-rate UDP traffic generated by \program{iperf} at 2\Mbps for 5 seconds, about 860 packets total. The receiver also recorded the CSI and per antenna RSSIs and noise floors to measure the RF channel for each correctly received packet. In these experiments, I turned off 802.11's link layer retransmissions in order to observe the underlying packet delivery rate. The experiments were conducted at night on unused 802.11 channels in order to minimize the effects of environmental movement and RF interference on these results.

%Note that CSI and RSSI are measured during the preamble, and so do not depend on the transmit rate, though they vary with the number of streams. 
%Similarly, 3x3 CSI gives us the channel between each pair of transmit and receive antennas, so it also implicitly contains 1x1 CSI\@.

The above testing gives ground truth data to probe variation across 200 links, 26\dB of transmit power, four antenna configurations ranging from SISO to MIMO3, and 8 MCS values per configuration. This covers all of the key variables needed to implement and evaluate my Effective SNR model. 

\section{Computing SNR}
Using these measurements, I computed the packet reception rate (PRR), Packet SNR, and Effective SNRs for all the measured configurations.

\subsection{Computing Packet SNR}
The natural 
When receiving a transmission with multiple antennas, there is one RSSI value per antenna; how should these measurements be combined into a single SNR value for the link? I first convert the per-antenna RSSI and noise measurements to SNRs (\eqref{eq:per_chain_snr}) and then sum the SNRs. This is a straightforward choice for a single spatial stream as it corresponds to receiver processing using maximal-ratio combining (\eqref{eq:mrc}). It is also reasonable for 2- and 3-stream MIMO because the symbols carried on different spatial streams are interleaved coded bits~\cite{80211n}.

\subsection{Computing Effective SNRs}
From these data, I compute Effective SNR values using the model described in \chapref{chap:model}. I parameterized the model using known properties of the Intel IWL5300 devices: they use minimum mean square error (MMSE) MIMO equalizers, and have known, fixed spatial mapping matrices.

\section{Results}
The goal of this chapter is to understand whether Effective SNR is a good metric, whether it is an accurate predictor of packet delivery. In this section, I evaluate this in two ways. The first is via the \define{transition window}, i.e., the SNR regime in which packet delivery for all links goes from near-zero to near-perfect. We saw in \chapref{chap:problem} that this transition occurs rapidly for a wired link (\figref{fig:snr_prr_attenuator}), but occurs over a wide range for wireless links (\figref{fig:snr_prr_26_65}) when using the Packet SNR. A narrow transition window would be one indicator that Effective SNR works well.

The second evaluation metric is \define{rate confusion}, i.e., how many rates might be best at a particular SNR value. The wired link showed clear separation between rates, such that at every SNR value there is a clear best rate. Conversely, because the transition regions of different wireless links overlap, links with the same Packet SNR might support very different rates. Here, I compare the degree of rate confusion between Packet SNR Effective SNR.

\subsection{Transition Windows}
\begin{table}
\centering
\begin{tabular}{cccccc}
\toprule
\multirow{2}{*}{MCS} & \multirow{2}{*}{Rate (Mbps)} & \multicolumn{2}{c}{$\Delta\rho_{\text{packet}}$ (dB)} &
\multicolumn{2}{c}{$\Delta\rho_{\text{eff}}$ (dB)} \\ 
\cmidrule(lr){3-4} \cmidrule(lr){5-6}
& &  ~~5\%--95\% & ~25\%--75\% & ~~5\%--95\% & ~25\%--75\%  \\
\midrule 
0 &  6.5                    & 3.08  & 1.29  & 2.05  & 0.81 \\
1 & 13.0                    & 3.45  & 1.44  & 2.38  & 0.89 \\
2 & 19.5                    & 6.27  & 3.12  & 2.30  & 0.85 \\
3 & 26.0                    & 3.93  & 1.98  & 3.02  & 0.94 \\
4 & 39.0                    & 7.05  & 3.49  & 2.19  & 0.93 \\
5 & 52.0                    & 7.16  & 3.20  & 2.29  & 1.06 \\
6 & 58.5                    & 7.25  & 3.37  & 2.92  & 1.41 \\
7 & 65.0                    & 7.24  & 2.81  & 2.92  & 1.35 \\
\midrule
\multicolumn{2}{c}{Average} & 5.68  & 2.59  & 2.51  & 1.03 \\         
\bottomrule
\end{tabular}
\caption{\label{tab:transitions} Width of SISO transition windows.}
\end{table}

I analyzed the SISO measurements to find the transition window for each of the measured links. I define the window to be the SNR values between which packet delivery rises from 10\% (lossy) to 90\% (reliable) for any link.

\tabref{tab:transitions} gives the width of the transition window (denoted $\Delta\rho$) for SISO rates using the Packet SNR and Effective SNR metrics. I show the 25\%--75\% range of points in the transition window as a measure of the typical link, and the 5\%--95\% range as a measure of most links. A good result here is a narrow window like that measured over a wire (\figref{fig:snr_prr_attenuator}).

The table shows that the transition widths are consistently tight with my model. Most links transition within a window of around 2\dB for most rates. The width of the SNR-based transition windows is typically two to three times looser, especially for the denser modulation schemes like 64-QAM and higher code rates. This means that it is easy for a less than ideal channel to degrade the reception of high rates. However, while the transitions for the last four rates are inflated with SNR, they remain tight with Effective SNR\@. 

The results for Effective SNR are in fact about the best that can be obtained because they are close to textbook transitions for flat-fading channels and those measured over a wire (\figref{fig:snr_prr_attenuator}). A small improvement is surely possible, but this is probably limited by the precision of my measurement data. The IWL5300 gives RSSI, AGC and noise values in dB to the nearest integer, and at most 8-bit CSI over a 48\dB range for 30 out of 56 subcarriers. With these factors, quantization error of at least 1\dB is likely.

The larger significance of narrow transition windows is that, by reducing them enough that they do not overlap, I can unambiguously predict the highest rate that will work for nearly all links nearly all of the time. In contrast, Packet SNR transition windows overlap such that for a given SNR there may be five different best rates across links the testbed. I explore this in the next section.


\subsection{Rate Confusion}
\label{sec:rate_confusion}

\begin{figure}[p]
	\begin{leftfullpage}
	\centering
	\subfigure[SISO Configurations]{
		\includegraphics[width=0.8\textwidth]{figures/delivery_figures/siso_rate_confusion.pdf}%
		\label{fig:snr_rate_step_1x1}%
	}%
	
	\subfigure[SIMO Configurations]{
		\includegraphics[width=0.8\textwidth]{figures/delivery_figures/simo_rate_confusion.pdf}%
		\label{fig:snr_rate_step_1x3}%
	}
	\caption[Rate confusion with Packet SNR and Effective SNR for different antenna configurations]{\label{fig:snr_rate_steps}Rate confusion with Packet SNR and Effective SNR. Excepting very low and high SNRs, one Packet SNR value maps to multiple best rates for different links. For the same data, 
Effective SNR provides a clear indicator of the best rate for nearly all links.}
	\end{leftfullpage}
\end{figure}

\setcounter{figure}{0}
\begin{figure}
\begin{xtrafullpage}
	\centering	
	\setcounter{subfigure}{2}
	\subfigure[MIMO2 Configurations]{
		\includegraphics[width=0.8\textwidth]{figures/delivery_figures/mimo2_rate_confusion.pdf}%
		\label{fig:snr_rate_step_2x3}%
	}%

	\subfigure[MIMO3 Configurations]{
		\includegraphics[width=0.8\textwidth]{figures/delivery_figures/mimo3_rate_confusion.pdf}%
		\label{fig:snr_rate_step_3x3}%
	}
	\vspace{36pt}
\end{xtrafullpage}
\end{figure}

To understand whether the Effective SNR model accurately predicts packet delivery, I analyze the fastest working rate (PRR$\geq$ 90\%) for each link and all NIC settings. \figref{fig:snr_rate_steps} shows the results broken down by configuration. The SISO experiment (\figref{fig:snr_rate_step_1x1}) shows links for both testbeds combined. The remaining graphs \figref{fig:snr_rate_step_1x3}--\ref{fig:snr_rate_step_3x3}  show rates for SIMO, MIMO2 and MIMO3 configurations for the Intel testbed only; it is denser than UW and supports MIMO experiments over our NIC's transmit power range.

These graphs study rate confusion, i.e., the ability of a given SNR value to predict the best rate across a wide set of links. If we consider the fastest rate each link supports at a particular Packet SNR or Effective SNR value, the best link is the link that supports the fastest rate, and the worst link is the slowest. The difference between the best and worst links, i.e., the set of possible best rates for an arbitrary link with a particular SNR, is the rate confusion. I plot this spread in these graphs. Note that the SIMO figure does not include data for the lowest 6.5\Mbps rate, because very few links experience loss at that rate within the transmit power range of the IWL5300---the added spatial diversity makes most links work well, as described below.

Ideally, the best and worst lines would overlap completely, such that the highest rate for a given SNR would be the same for the best and worst links. This rate would then be an accurate prediction for the particular Effective SNR or Packet SNR level. Conversely, gaps between the best and worst lines expose confusion about which rate will be the highest rate for that SNR.

For the SISO (\figref{fig:snr_rate_step_1x1}) and MIMO3 (\figref{fig:snr_rate_step_3x3}) cases, the figures show that using Packet SNR results in a large spread between the best and worst lines. Except for extremely low and high SNRs, nearly all SNRs have at least two---and up to five different---rates as suitable choices for the best rate. That is, Packet SNR often poorly indicates rate.

In sharp contrast, %looking at bottom lines in the graphs, 
the two Effective SNR lines overlap almost all the time, and mostly appear to be a single line. This is almost an ideal result. Effective SNR is a clear indicator of best rate. When there is slight separation, the spread is only between rates that use the same modulation but different amounts of coding. These combinations are also close together in our wired experiments. 

Interestingly, these results show that Packet SNR predictions are much better for the SIMO and MIMO3 cases, though still not as accurate as Effective SNR, particularly for the highest rates. The reason is \emph{spatial diversity}: spare receive antennas gather the received signal and combine to make the channel more frequency-flat, thus bringing the Packet SNR closer to the Effective SNR. This effect is well-known, though typically not observable using real 802.11 NICs which, except my prototype implementation, do not export CSI. This result suggests that Packet SNR \emph{is} a reasonable predictor for an 802.11 configuration with significant diversity. Still, observe that Packet SNR does not transfer well across the antenna modes (as diversity gains and inter-stream interference change unpredictably) which makes this less useful. This is one reason that SISO rate adaptation schemes do not translate to MIMO.

Finally, I note that neither Effective SNR nor Packet SNR performs extremely well at the lowest modulation at low SNRs. I believe this artifact arises from errors in the AGC values reported by the NIC, observed by Judd et al.~\cite{Judd_CHARM} and confirmed by my data for Intel's hardware.

\section{Transmit Power Control}
\label{sec:tx_power_trim}
The results so far show we can predict delivery over a range of transmit powers (as well as other choices). I now show that CSI measured at one transmit power level is useful to predict delivery at a different power level. This is valuable for power control applications, e.g., pruning excess power to reduce co-channel interference~\cite{symphony09, power_control_study, pcmac}. 

\begin{figure}[t]
  \centering
  \includegraphics[width=0.9\textwidth]{figures/eff_vs_snr_qpsk.pdf}
  \caption{Effective SNR (for QPSK) versus packet SNR for flat (left) to faded (right) links.}
  \label{fig:eff_vs_rssi}
\end{figure}

Note that changing transmit power has a different effect (in terms of delivery and highest rate) on real links even if they start at exactly the same rate and SNR. \figref{fig:eff_vs_rssi} plots the Packet SNR versus Effective SNR relationship for six example 1x1 links in T1 and T2.
I compute this data by scaling the CSI measured at maximum transmit power over a range of power levels.
The links range from near-flat to deeply-faded. Correspondingly, they have different slopes. On the left, Packet SNR matches Effective SNR for the nearly flat link. 
However, for the right-most, deeply faded links, the Packet SNR decreases from 25\dB to 15\dB (10$\times$ transmit power reduction) as the Effective SNR only drops by 4\dB (2.5$\times$). This difference in how links harness power makes transmit power control non-trivial.


\begin{figure}[t]
      \centering
      \subfigure[Predicted and measured power saving]{%
      \includegraphics[width=0.5\columnwidth,viewport=0 7 195 110,clip]{figures/power_save_1x1.pdf}%
      \label{fig:power_save_cdf}%
      }%
      \subfigure[Measured PRR corresponding to reduced TX power levels]{%
      \includegraphics[width=0.5\columnwidth,viewport=0 7 195 110,clip]{figures/power_save_final_prr_1x1.pdf}%
      \label{fig:power_prr_cdf}%
      }%
      \caption{\label{fig:power_save_1x1} Power saving and performance impact of pruning excess transmit power. Pruning with Effective SNR is tight (within 0.5\dB) and does not degrade performance. Pruning with packet SNR degrades performance more without much extra savings.} 
\end{figure}

To test predictions across power levels, I consider the goal of trimming excess transmit power, i.e. power that can be removed without causing the highest rate for the link to drop. These experiments start with 88 SISO links from the Intel Labs Testbed configured to radiate 10\mW of transmit power, and one CSI sample per link. Considering transmit power reductions in increments of 2\dB, the Effective SNR model can predict the best supported rate for each reduced power level, and choose the lower power level with the same best rate. The measurements described earlier include the ground truth packet reception ratio for each power level, and thus these data can be used to check the accuracy of the predictions.

\figref{fig:power_save_1x1} shows the power savings and performance degradation of four different threshold schemes.
A good result here is power savings without a loss of performance; the absolute amount of power savings is not meaningful as it depends on the testbed.
The Measured (Optimal) line shows the best that can be done. Measured PRRs at all power levels are used to guide power control decisions. Therefore, the final delivery probabilities are hardly decreased (all links have PRR$>$90\%), yet most links save a little power and some save a lot. %of power.

The graphs show that using Effective SNR to predict how much power to trim has a similarly good tradeoff. Impact on rate remains limited, yet power is saved, more than 10\dB for around 10\% of the links. The gap between the Measured and the Eff SNR lines is due to the fact the Eff SNR thresholds might be slightly conservative for some links. To show that this trimming is tight, consider trimming towards slightly lower thresholds ($\text{Effective SNR}-\text{0.5\dB}$, solid line). This results in little additional power savings, but degrades more links so that they work partially. In comparison, the Pkt SNR line shows the effects of using Packet SNR to save power. The savings are barely increased, but several links are degraded to the point that some stop working altogether.

\section{Interference}
\label{sec:interference}
Finally, it is important to investigate how an Effective SNR-based protocol can cope with interference. This is one of the largest potential weaknesses of this technique, because Effective SNR is based on measurements taken only during the packet preamble.% There are three important components of dealing with interference: 1) maintaining an accurate estimate of interference-free link quality, so that transient interference doesn't cause wild rate swings; 2) recognizing collisions when they occur, to avoid unnecessary rate fallback; and 3) providing an estimate of link quality that enables efficient operation during persistent interference.

I studied the variation of CSI measurements during interference. I chose two nodes at UW that do not detect each other with carrier sense and sent large packets designed to collide, while monitoring the CSI recorded by all other receiving nodes. I also varied the transmit power of the node designated as the interferer from low to high to induce a large range of interfering channels. For all but one of 20 links, the rate predicted by the majority of CSI measurements for correct packets was the same with and without interference; the remaining link was off by a single rate. From these measurements, we can conclude that the mere presence of interference does not completely invalidate Effective SNR values, and thus transient interference will not cause wild swings in transmit rate.

However, for continuous interference Effective SNR will provide an aggressive estimate, and will need another way to compensate. This should be reflected in larger noise floor measurements by the NIC,\footnote{Note that OFDM does not turn interference into inflated RSSI as do the spread spectrum modulations used in 802.11b.} however the platform does not provide this information for dropped packets. An alternative, that I have not yet explored, might be an \define{Effective SINR} metric that incorporates CSI measurements from the interfering nodes to predict packet delivery.

%Because it does not use packet loss as a signal, ESNR will not conservatively reduce rate during interference. Thus transient interference will only cause transient interruptions. However, the Effective SNR measured in a single packet preamble simply does not provide an estimate of link quality when the link experiences persistent interference. One solution is to leverage SoftRate in these circumstances; while Effective SNR can guide the overall choice of antennas and number of spatial streams, SoftRate's continuous estimate of BER may be better suited for choosing rates within one mode. An alternative (that we have not yet explored) might be an \emph{effective SINR} metric incorporating CSI measurements from the interfering nodes to predict packet delivery.

To recognize collisions, I propose to leverage a new MAC feature of the 802.11n packet aggregation mechanism. Block ACKs selectively acknowledge frames in a batch of packets transmitted as one continuous burst. Each packet in the burst has a separate checksum, and thus the Block ACK serves as block-based feedback of packet correctness as the block-based checksums analyzed in PPR~\cite{Jamieson_PPR}. We can therefore use the error patterns in the Block ACK to recognize collisions: when the rate is overselected, errors should be randomly distributed throughout the batch, and bursty when a collision clobbers a continuous part of the batch.

\section{Summary}
From these results, I conclude that Effective SNR consistently and accurately indicates the best rate for nearly all links and all configurations without any per-link calibration. The low degree of rate confusion with Effective SNR in \secref{sec:rate_confusion} implies that it should be possible to define SNR thresholds that clearly define when a rate will work well. I discuss how to choose these thresholds in the next chapter,\footnote{\textcolor{red}{XXX: or, where should this be? It's not really necessary yet, not until we start making algorithmic choices.}}. The results in \secref{sec:tx_power_trim} demonstrate the flexibility of this approach by demonstrating that CSI measurements are valid not just across different rates, but also across transmit power scaling. Finally, I showed that Effective SNR measurements are generally not corrupted by transient interference, as long as the packet preamble from which CSI is measured is relatively interference-free.

Having gained confidence that Effective SNR can be useful, in the remainder of this thesis I evaluate Effective SNR at the application level. In the next chapter, I deploy my model as part of a system that selects the optimal rate for a wireless link.

\textcolor{red}{XXX Which reminds me: where should I mention quantization error?}
%From now on, we use the thresholds in these graphs to predict the working rate for any link. They agree with the measured SNRs on a wired link (\figref{fig:snr_prr_attenuator}), which strongly suggests that the Effective SNR captures the fundamental error characteristics of the link. 

%%%%%%%%%%%%%%%%%%%%%%%%%%%%%%%%%%
\ifx\mainfile\undefined
%
% ==========   Bibliography   ==========
%
%\nocite{*}   % include everything in the uwthesis.bib file
\bibliographystyle{plain}
\bibliography{dhalperi_thesis}

\end{document}
\fi
