\ifx\mainfile\undefined
\input{chapter_head}
\setcounter{chapter}{7} % Set to n-1!
\fi
%%%%%%%%%%%%%%%%%%%%%%%%%%%%%%%%%%

\cleardoublepage
\chapter{Further Applications of Effective SNR}
\label{chap:applications}

In the previous chapter, I showed that my Effective SNR model provides good performance when selecting rates for 802.11a/g and 802.11n links. To further illustrate the model's flexibility and power, this chapter presents my exploration of Effective SNR to various problems of wireless links and wireless networks.

\begin{table}[htp]
	\centering
	\begin{tabular}{lc}
	\toprule
		\textbf{Application of Effective SNR} & \textbf{Described in} \\
	\midrule
		Bitrate/MCS selection & \chapref{chap:rate}, originally in \cite{Halperin_ESNR}\\
		Channel width selection & \chapref{chap:rate}, originally in \cite{Halperin_ESNR}\\
		Antenna selection & \chapref{chap:rate}, originally in \cite{Halperin_ESNR}\\
		Power control & \chapref{chap:rate}, originally in \cite{Halperin_ESNR}\\
		Channel selection & \secref{sec:esnr_chansel}\\
		AP selection & \secref{sec:esnr_apsel}\\
		Path selection/BSS selection in WDS & \secref{sec:esnr_pathsel}\\
		Interference planning \\
		Partial packet recovery/FEC & Bhartia et al.~\cite{Bhartia_FreqDiv}\\
		Beamforming \\
		Multicast rate selection \\
	\bottomrule
	\end{tabular}
	\caption[A variety of applications of Effective SNR]{\label{tab:esnr_uses}A variety of applications of Effective SNR\@.}
\end{table}

\tabref{tab:esnr_uses} shows a list of several potential applications of Effective SNR, as those described in \chapref{chap:problem}. These range from optimizing various parameters of a single Wi-Fi link, such as the MCS or antenna set used, to coordinating many 802.11 nodes in a dense wireless network. Additionally, I identify applications that can be implemented by looking at other aspects of the Channel State Information below in \tabref{tab:csi_uses}. These provide useful primitives that can enable systems to adapt behavior based on the location and movement of the user. Combined, I believe these form the critical building blocks for configuring dense 802.11 networks.

\begin{table}[htp]
	\centering
	\begin{tabular}{lc}
	\toprule
		\textbf{Application of CSI} & \textbf{Described in} \\
	\midrule
		Mobility classification & \secref{sec:esnr_mobility}\\
		Indoor localization & Sen et al.~\cite{Sen_PinLoc,Sen_SpinLoc} \\ 
	\bottomrule
	\end{tabular}
	\caption[A variety of applications of Channel State Information]{\label{tab:csi_uses}A variety of applications of Channel State Information.}
\end{table}

In this chapter, I explore CSI and Effective SNR approaches to implementing four of the above applications. The first three are are selection algorithms for access points, operating channels, and multi-hop paths through networks that represent the main decisions (beyond rate) that need to be made in flexible topologies such as Wi-Fi Direct networks. The last application is to classify whether a device is mobile, to switch between algorithms designed for static vs mobile links, or trigger a search for a new, better operating point when the device begins to move.

%%%%%%%%%%%%%%%%%%%%%%%%%%%%%%%%%%%%%%%%%%%%%%%%%%%%%%%%%%%%%%%%%%%%%%%%%%%%%%%%%%%%%%%%%%%%%%%%%%%%%%%%%%%%%%%%%%%%%%%%%%%%%%%%%%%%%%%%%
\section{Dataset}
To thoroughly evaluate these applications in indoor Wi-Fi networks, I took comprehensive RF measurements in between static 24 devices in my testbed at the University of Washington. 
The IWL5300 devices in the testbed can operate on 35 channels, each 20\MHz wide, defined by IEEE 802.11 standard. Of these 35 channels, 11 overlap in the 83-MHz wide unlicensed 2.4\GHz band, and the remaining 24 non-overlapping channels are spread across three non-contiguous bands between 5.170\GHz and 5.835\GHz.

For the results presented in this section, I measured the physical RF channel (using both RSS and CSI) and actual packet delivery using each of the 24 IEEE 802.11n modulation and coding schemes (\mcs{0}--\mcs{23}) that use  1-, 2-, or 3-streams. I measured these data for all links between 24 testbed nodes at UW\@. Each sender transmits packets with random payloads, sending a total of 2400 packets by interleaving 100 packets from each of the 1-, 2-, or 3-stream 802.11n rates. Each receiver uses 3 antennas for spatial diversity and/or spatial multiplexing. From these data, we can compute the measured throughput $T$ for a particular link~$\ell$ and \mcs{$m$} as
\begin{equation}
	\label{eq:prr_throughput}
	T(\ell,m) = \frac{P(\ell,m)}{100} \cdot B(m),
\end{equation}
where $P(\ell,m)$ is the number of packets delivered on link $\ell$ at \mcs{$m$} and $B(m)$ is the raw bitrate of \mcs{$m$}.
%%%%%%%%%%%%%%%%%%%%%%%%%%%%%%%%%%%%%%%%%%%%%%%%%%%%%%%%%%%%%%%%%%%%%%%%%%%%%%%%%%%%%%%%%%%%%%%%%%%%%%%%%%%%%%%%%%%%%%%%%%%%%%%%%%%%%%%%%
\section{Access point selection}\label{sec:esnr_apsel}
The first protocol operation a wireless client device performs is to join an existing network. This operation typically consists of scanning for a known access point (AP) by sending broadcast packets called \define{probe requests} on different frequencies until a recognized AP is found. In a typical home today, there will likely be one \define{probe response}: the single home AP. However, in a dense Wi-Fi network, such as an enterprise Wireless Distribution System or a Wi-Fi Direct peer-to-peer network, the client needs to choose from among many available responding devices. This is the \define{access point selection} problem.

The general framework for an access point selection algorithm is to examine the access point's probe responses to predict the link performance, and choose the fastest access point (\algref{alg:ap_sel_basic}). The standard approach used today is for the client to measure the Packet SNR from each probe response and connect to the strongest AP (\algref{alg:snr_link_metric}). Here, SNR measurements are used as a proxy for an estimate of the downlink throughput, and the AP with the highest SNR is likely the best. (Other factors than link quality, such as interference and contention with other clients, can affect throughput as well; some systems can take these factors into account. I discuss these and other related works in \chapref{chap:related}.)
%In dual-band networks, some devices may prefer a 5\GHz AP with slightly lower SNR, as long as it exceeds a minimum threshold, based on the optimistic assumption that interference is lower in the 5\GHz band.
In this chapter, I evaluate the ability of Effective SNR to improve this decision process (\algref{alg:eff_snr_link_metric}) in those cases for which the Packet SNR is a poor indicator of performance.

%Normally, a client scanning for a network cycles through the available channels sends a probe request at the lowest rate (including a single stream and 20\MHz channels), and all APs or repeaters in range respond. We propose that the client instead send multiple probes that use the lowest 6.5\Mbps rate, but vary the number of streams and channel width in decreasing order. In this way, the coordinator and all repeaters that measure CSI from the probes can compute the Effective SNR for the uplink. The probe responses can now include the computed Effective SNR to better inform the client's choice. If the client includes its transmit power level in the probe request (or if the responder makes a conservative estimate), then the responder can combine this information with the CSI measured from the probe to compute the Effective SNR for the downlink. It can then send the probe response at a faster rate than the base rate and reduce the overhead of the probe response.

%%%%%%%%%%%%%%%%%%%%%%%%%%%%%%%%%%%%%%%%%%%%%%%%
\begin{algorithm}[tp]
\caption{\label{alg:ap_sel_basic}\fcall{AccessPointSelection(AP Set $A$, Sender $s$)}}
\begin{algorithmic}[1]
\RETURN $\argmax_{a\in A} \fcall{GetMetric}(a, s)$ \hfill \COMMENT{choose the AP with the best downlink metric}
\end{algorithmic}
\end{algorithm}
%%%%%%%%%%%%%%%%%%%%%%%%%%%%%%%%%%%%%%%%%%%%%%%%
\begin{algorithm}[tp]
\caption{\label{alg:snr_link_metric}\fcall{GetMetric-PacketSNR(Sender $s$, Receiver $r$)}}
\begin{algorithmic}[1]
\STATE Measure the Packet SNR $\rho$ at $r$ from $s$
\RETURN $\rho$
\end{algorithmic}
\end{algorithm}
%%%%%%%%%%%%%%%%%%%%%%%%%%%%%%%%%%%%%%%%%%%%%%%%
\begin{algorithm}[tp]
\caption{\label{alg:eff_snr_link_metric}\fcall{GetMetric-EffectiveSNR(Sender $s$, Receiver $r$)}}
\begin{algorithmic}[1]
\STATE Measure the CSI at $r$ from $s$ \hfill \COMMENT{a full CSI including all TX and RX antennas}
\STATE Compute the Effective SNR estimates $\rho_{\text{eff},m}$ for each MCS $m$
\STATE Determine whether each MCS works by comparing $\rho_{\text{eff},m} \geq \tau_m$
\RETURN the bitrate $B(m)$ for the fastest working \mcs{$m$}
\end{algorithmic}
\end{algorithm}
%%%%%%%%%%%%%%%%%%%%%%%%%%%%%%%%%%%%%%%%%%%%%%%%

\subsection{Characterization of access points}
I begin by characterizing whether access point selection matters in my testbed. How much does a good or bad choice of access point impact performance?

To generate data for this evaluation, I first filtered the data set to the 11 channels in the 5\GHz band for which there is no overlapping UW Wi-Fi network. Then I considered the access point selection problem by considering each client and channel in turn. In particular, for node $c$ playing the role of a client, I defined $A$ to be the the set of access points that responded to a probe from that client on a particular channel, using the Packet SNRs measured from potential AP $a \in A$. By considering each channel independently each client generates 11 data points, for a total of $11*24=264$ simulated client association attempts. To ensure that AP choice can matter, I eliminated 17 clients that had fewer than 3 responding APs on a particular channel, leaving 247 total.

\figref{fig:ap_sel_rel_diff} and \figref{fig:ap_sel_tpt_diff} show the results, framed as the difference in throughput (relative or absolute) from the best choice AP for the worst, median, and average choices. These graphs show that for these clients, a bad choice of AP can be really bad: the worst responding AP offers less than half of the best AP's throughput in 95\% of cases, and in most cases the random (average) and median APs aren't much better. In absolute terms, the downlink from the median AP is 50\Mbps to 100\Mbps worse than from the best AP in most cases. A bad choice of access point can dramatically hurt performance in practice.

\begin{figure}[t]
	\centering
	\includegraphics[width=\textwidth]{figures/applications/ap_sel_rel_diff.pdf}
	\caption{\label{fig:ap_sel_rel_diff}For each client, the relative difference in throughput over access points.}
\end{figure}

\begin{figure}[t]
	\centering
	\includegraphics[width=\textwidth]{figures/applications/ap_sel_tpt_diff.pdf}
	\caption{\label{fig:ap_sel_tpt_diff}For each client, the absolute difference throughput loss over access points.}
\end{figure}

\subsection{Access point selection performance}
As presented above, \algref{alg:ap_sel_basic} shows the framework for selecting access points typically used today, with Packet SNR (\algref{alg:snr_link_metric}) usually used as the metric of comparison between access points. I calculate the Effective SNR link metric as in \algref{alg:eff_snr_link_metric}---a simple instantiation of the procedure described in \chapref{chap:model}. 

\begin{figure}[p]
	\centering
	\includegraphics[width=\textwidth]{figures/applications/ap_sel_ratio_opt.pdf}
	\caption{\label{fig:ap_sel_ratio_opt}AP selection using Packet SNR or Effective SNR compared to Optimal.}
\end{figure}

\begin{figure}[p]
	\centering
	\includegraphics[width=\textwidth]{figures/applications/ap_sel_diff_opt.pdf}
	\caption{\label{fig:ap_sel_delta_opt}The difference in throughput using APs selected by Packet SNR or Effective SNR compared to Optimal.}
\end{figure}

\begin{figure}[p]
	\centering
	\includegraphics[width=\textwidth]{figures/applications/ap_sel_ratio.pdf}
	\caption{\label{fig:ap_sel_ratio}The relative throughput selecting APs by Packet SNR or by Effective SNR\@.}
\end{figure}

\figref{fig:ap_sel_ratio_opt} shows the performance of Packet SNR and Effective SNR-based access point selection algorithms relative to the access point with the best downlink throughput, and \figref{ap_sel_delta_opt} shows the absolute performance loss of the suboptimal choices. We see that both algorithms perform well, choosing an optimal access point in most cases. I attribute this overall good performance to the fact that the potential access points are spread across a large testbed, exhibiting a wide range of SNRs. This large geographic spread means that there may be one or a few nearby access points that offer a clear best choice, and Packet SNR---which is correlated with distance---can correctly identify good choices.

Although both algorithms perform well, Effective SNR works better. The Effective SNR algorithm finds the best access point for 82\% of clients, versus 60\% for Packet SNR. Put another way, Effective SNR makes a suboptimal choice half as often as Packet SNR, a strong advantage. And considering only the suboptimal choices, those made by Effective SNR are better: 3/4 of incorrect choices are within 80\% of optimal, versus only half when using Packet SNR. These benefits translate to raw bitrate as well: access points selected by Effective SNR links are within 10\Mbps of optimal for 90\% of cases, but only 70\% of selections meet this criterion when using Packet SNR. For Packet SNR, the 90th percentile performance loss is 43\Mbps, versus 9\Mbps with Effective SNR. These results show that the Effective SNR-based access point selection algorithm works well and makes better choices than one based on Packet SNR.

Of course, though Effective SNR is statistically better than Packet SNR over the testbed does not mean it works better in all cases. \figref{fig:ap_sel_ratio} shows the ratio of the downlink throughput for the access points chosen by each algorithm for the same client. They choose the same access point (or access points that perform equivalently well) for 57\% of the clients, Effective SNR makes a better choice in 33\% of cases, and Packet SNR is better for the remaining 10\%. Effective SNR is not always better, but is better by a ratio of 3:1 for cases when the algorithms differ. The graph also that when Packet SNR makes a better choice, Effective SNR comes close---within 2/3 for 9 in 10 of these cases---and often improves on the choice from Packet SNR by a larger margin.

\xxx{do I talk about why Packet SNR can be better here? (1) Single CSI probe isn't as accurate as multiple probes, whereas single RSSI probe averages noise out across carriers and is generally fine. (2) Model's assumptions broken, i.e., bad subchannels coded nearby. (3) other reasons?}

\heading{Summary.} When selecting access points, both Packet SNR and Effective SNR make good choices; each algorithm selects the fastest access point in most cases. However, the ability of Effective SNR to capture channel effects leads to better choices more often, and generally closer-to-optimal performance when it makes a choice incorrectly.

\xxx{Another interesting experiment would be to randomly assign APs to have 1, 2, or 3 transmit antennas. RSSI would see roughly the same SNR, but ESNR can capture the effects of heterogeneity. (Is this unfair? One could design RSSI-based algs that take streams into account; just don't think there are any yet.)}

%\subsection{Evaluation methodology}
%Compare the following strategies:
%\begin{itemize}
%\item first AP seen
%\item max RSSI
%\item max CSI predicted rate
%\item max measured rate
%\end{itemize}


%%%%%%%%%%%%%%%%%%%%%%%%%%%%%%%%%%%%%%%%%%%%%%%%%%%%%%%%%%%%%%%%%%%%%%%%%%%%%%%%%%%%%%%%%%%%%%%%%%%%%%%%%%%%%%%%%%%%%%%%%%%%%%%%%%%%%%%%%
\section{Channel selection}\label{sec:esnr_chansel}
The goal of a \emph{channel selection} algorithm is to quickly choose the best operating frequency for a pair of nodes to communicate. In this section, I define the ``best'' channel to be the channel that provides the highest throughput in the absence of interferers.

%%%%%%%%%%%%%%%%%%%%%%%%%%%%%%%%%%%%%%%%%%%%%%%%%%%%%%%%%%%%%%%%%%%%%%%%%%%%%%%%%%%%%%%%%%%%%%%%%%%%%%%%%%%%%%%%%%%%%%%%%%%%%%%%%%%%%%%%%
\subsection{Characterization of 802.11 channels}
To start my investigation of channel selection algorithms, I first measured how the operating frequency affects 802.11n links in practice.

\subsubsection{Measurements}
\label{sec:chan_sel_data}
The \term{iwl5300} devices in the testbed can operate on 35 channels, each 20\MHz wide, defined by IEEE 802.11 standard. Of these 35 channels, 11 overlap in the 83-\MHz wide unlicensed 2.4\GHz band, and the remaining 24 non-overlapping channels are spread across three non-contiguous bands between 5.170\GHz and 5.835\GHz.

For the results presented in this section, I measured the physical RF channel (using both RSS and CSI) and actual packet delivery using each of the 24 IEEE 802.11n modulation and coding schemes (MCS~0--MCS~23) that use  1-, 2-, or 3-streams. I measured these data for all links between 24 testbed nodes at UW\@. Each sender transmits packets with random payloads, sending a total of 2400 packets by interleaving 100 packets from each of the 1-, 2-, or 3-stream 802.11n rates. Each receiver uses 3 antennas for spatial diversity and/or spatial multiplexing. From these data, we can compute the expected throughput $T$ for a particular link $\ell$ and MCS~$m$ as
\begin{equation}
	\label{eq:prr_throughput}
	T(\ell,m) = \frac{P(\ell,m)}{100} \cdot B(m),
\end{equation}
where $P(\ell,m)$ is the number of packets delivered on link $\ell$ at MCS~$m$ and $B(m)$ is the raw bitrate of MCS~$m$.

\subsubsection{Impact of operating channel}
To understand whether the choice of operating channel matters in practice, I measured the throughput for each link as a function of 802.11 channel. Since there is considerable Wi-Fi interference in our building on the entire 2.4\GHz band and on many of the 5\GHz channels, I use a subset of data taken on the 11 channels in the 5\GHz band that the testbed nodes can use and are not shared with a UW CSE or nearby UW access point.

\begin{figure}[tp]
	\centering
	\includegraphics[width=0.6\textwidth]{figures/esnr/rel_diff.pdf}
	\caption{\label{fig:rel_diff}The relative difference in throughput over 802.11n channels.}
\end{figure}

\begin{figure}[tp]
	\centering
	\includegraphics[width=0.6\textwidth]{figures/esnr/tpt_diff.pdf}
	\caption{\label{fig:tpt_diff}The absolute difference in throughput over 802.11n channels.}
\end{figure}

\figref{fig:rel_diff} and \figref{fig:tpt_diff} show how the throughput of the worst, median, and average channels compares to the best channel (with the highest throughput) for these links. \figref{fig:rel_diff} shows the relative throughput as a fraction of the throughput of the best channel, and \figref{fig:tpt_diff} shows the absolute difference between channels in Mbps. These figures demonstrate that the choice of channel can dramatically impact performance. The worst channel delivers no packets for more than half of the links, and offers less than half of the best throughput for more than 80\% of the links. In absolute terms, this difference can be quite large: the worst channel is a median 50\Mbps worse than the best channel, and for a few links the difference is more than 100\Mbps. For these links, some channel will deliver no packets at all, while another delivers packets at nearly the maximum bitrate. An unlucky choice of channel can cripple performance and result in little to no throughput.

The median and average channels perform about equivalently in our testbed, and even these are significantly worse. These channels yield less than half the optimal throughput for 40\% of the links and for only 20\% of links do these come within 80\% of the best throughput. These figures show that for very few links do most channels perform about the same, and the median or average channel is 10--25\Mbps worse than the best channel for most links, and the gap is larger than 25\Mbps for 20\% of links.

\subsubsection{Conclusion}
From these results, I conclude that a strategy that picks a fixed channel or a random channel will perform significantly worse than a strategy that can identify channels that are likely to offer good performance. Having motivated the need for an accurate channel selection algorithm, I present and evaluate different channel selection strategies in the rest of this section.

%%%%%%%%%%%%%%%%%%%%%%%%%%%%%%%%%%%%%%%%%%%%%%%%%%%%%%%%%%%%%%%%%%%%%%%%%%%%%%%%%%%%%%%%%%%%%%%%%%%%%%%%%%%%%%%%%%%%%%%%%%%%%%%%%%%%%%%%%
\subsection{Channel selection algorithms}
\algref{alg:chan_sel_basic} is a template for a channel selection algorithm. It takes as input a list of channels to evaluate $C$, and a sender $s$ and receiver $r$ that together define a link. The two nodes hop across channels, using the \fcall{PredictChannelThroughput} function to evaluate the performance of the link on each. The algorithm tracks the best performing channel, labeled $c_{\max}$ and returns that value. (\fcall{ChannelSelection} can also return $\emptyset$ if all channels have zero predicted throughput, but we only consider links that can communicate on at least 1 channel.) Using this template, different channel selection algorithms can be instantiated by providing different implementations of \fcall{PredictChannelThroughput}.

%%%%%%%%%%%%%%%%%%%%%%%%%%%%%%%%%%%%%%%%%%%%%%%%
\begin{algorithm}[htp]
\caption{\label{alg:chan_sel_basic}\fcall{ChannelSelection($C, s, r$)}}
\begin{algorithmic}
\STATE $t_{\max}\gets 0\Mbps$
\STATE $c_{\max} \gets \emptyset$
\FORALL{$c \in C$}
\STATE Both $s$ and $r$ switch to channel $c$
\STATE $t \gets \fcall{PredictChannelThroughput($c, s, r$)}$
\IF{$t > t_{\max}$}
	\STATE $t_{\max} \gets t$
	\STATE $c_{\max} \gets c$
\ENDIF
\RETURN $c_{\max}$
\ENDFOR
\end{algorithmic}
\end{algorithm}
%%%%%%%%%%%%%%%%%%%%%%%%%%%%%%%%%%%%%%%%%%%%%%%%

I consider three different channel selection algorithms in this section. The first is \fcall{ChannelSelectionOPT}, an oracular channel selection algorithm that chooses the optimal channel. For implementation purposes, I instantiate \fcall{ChannelSelectionOPT} by \algref{alg:chan_sel_probe} (\fcall{ProbeChannelThroughput}), which probes all MCSes with MTU-sized packets to determine packet delivery and predicts throughput according to \eqref{eq:prr_throughput}.

%%%%%%%%%%%%%%%%%%%%%%%%%%%%%%%%%%%%%%%%%%%%%%%%
\begin{algorithm}[tp]
\caption{\label{alg:chan_sel_probe}\fcall{ProbeChannelThroughput($c, s, r$)}}
\begin{algorithmic}
\STATE $p_0,p_1,\dots,p_{23} \gets 0$
\STATE $N \gets \text{number of probes at each MCS}$
\FOR{$i = 1 \dots N$}
\FOR{$m = 0 \dots 23$}
\STATE $s$ sends one MTU-sized packet at MCS~$m$
\IF{$s$ receives an ACK from $r$}
\STATE $p_m \gets p_m + 1$
\ENDIF
\ENDFOR
\ENDFOR
\STATE $t_{\max}\gets \max \{p_m/N \cdot M(m)\} \text{ over all } m$ \hfill \COMMENT{\eqref{eq:prr_throughput}}
\RETURN $t_{\max}$
\end{algorithmic}
\end{algorithm}
%%%%%%%%%%%%%%%%%%%%%%%%%%%%%%%%%%%%%%%%%%%%%%%%
%%%%%%%%%%%%%%%%%%%%%%%%%%%%%%%%%%%%%%%%%%%%%%%%
\begin{algorithm}[tp]
\caption{\label{alg:chan_sel_rss}\fcall{PredictChannelThroughputRSS($c, s, r$)}}
\begin{algorithmic}
\STATE $s$ sends one packet with 0-byte payload at MCS~0
\STATE $r$ computes the RSS $R$ and returns it along with the ACK
\IF{$s$ receives an ACK from $r$}
\RETURN \fcall{RSSToThroughput($R$)}, $R$ \hfill \COMMENT{the RSS $R$ is used to break ties}
\ELSE
\RETURN 0
\ENDIF
\end{algorithmic}
\end{algorithm}
%%%%%%%%%%%%%%%%%%%%%%%%%%%%%%%%%%%%%%%%%%%%%%%%
%%%%%%%%%%%%%%%%%%%%%%%%%%%%%%%%%%%%%%%%%%%%%%%%
\begin{algorithm}[tp]
\caption{\label{alg:chan_sel_esnr}\fcall{PredictChannelThroughputESNR($c, s, r$)}}
\begin{algorithmic}
\FOR{$m \in \{16, 8, 0\}$}
\STATE $s$ sends one packet with 0-byte payload at MCS~$m$
\STATE $r$ computes the Effective SNR values $\rho_\text{eff}$ and returns them along with the ACK
\IF{$s$ receives an ACK from $r$}
	\RETURN \fcall{ESNRToThroughput}($\rho_\text{eff}$), $\rho_\text{eff}$ \hfill \COMMENT{$\rho_\text{eff}$ is used to break ties}
\ENDIF
\ENDFOR
\RETURN 0
\end{algorithmic}
\end{algorithm}
%%%%%%%%%%%%%%%%%%%%%%%%%%%%%%%%%%%%%%%%%%%%%%%%

The second algorithm chooses the channel based on RSS, presented in \algref{alg:chan_sel_rss}. In this case, the sender only needs to send a single probe packet (with no payload) in order to measure the RSS on the channel. Since \fcall{RSSToThroughput} is monotonically increasing in RSS, this algorithm is equivalent to selecting the channel with the maximum RSS\@.

The third algorithm chooses the channel based on the Effective SNR, presented in \algref{alg:chan_sel_esnr}. Here, the sender sends probes (with no payload) with decreasing numbers of spatial streams to enable the receiver to collect a maximal CSI measurement. The receiver then sends the computed Effective SNR values back to the sender, which uses them to predict the best rate.

I evaluate these three algorithms in the next section.

%%%%%%%%%%%%%%%%%%%%%%%%%%%%%%%%%%%%%%%%%%%%%%%%%%%%%%%%%%%%%%%%%%%%%%%%%%%%%%%%%%%%%%%%%%%%%%%%%%%%%%%%%%%%%%%%%%%%%%%%%%%%%%%%%%%%%%%%%
\subsection{Evaluation}
To evaluate the three channel selection algorithms I presented in the previous section, I use the measurements described in \secref{sec:chan_sel_data}. In the 24-node UW testbed, there are $24*23=552$ ordered pairs of nodes that can form unidirectional links. To consider only non-trivial cases of channel selection, I restrict my analysis in this section to the 201 of these 552 links can deliver packets on at least 3 different channels.

%%%%%%%%%%%%%%%%%%%%%%%%%%%%%%%%%%%%%%
\subsubsection{Channel selection accuracy}
\figref{fig:chan_sel_ratio_opt} shows the performance relative to the optimal algorithm when using Effective SNR and RSS to choose between channels. I plot the complementary CDF of the links, so that each $(x,y)$ point in the graph shows the fraction of links $y$ that achieved performance at least a fraction $x$ of the optimal algorithm. The line for an accurate algorithm would be located in the upper right corner of the graph: most links would achieve most of the best possible throughput. This figure shows that both algorithms perform well, though choosing channel based on Effective SNR is more accurate than based on RSS.

\begin{figure}[htp]
	\centering
	\includegraphics[width=0.6\textwidth]{figures/esnr/chan_sel_ratio_opt.png}
	\caption{\label{fig:chan_sel_ratio_opt}Channel selection algorithm performance relative to an optimal algorithm.}
\end{figure}

In this figure, Effective SNR chooses an optimal channel for 121 links (60\%), whereas RSS is optimal for only 73 links (36\%). The Effective SNR-based algorithm is within 90\% of optimal for 161 links (84\%), 80\% for 181 links (90\%), and 70\% for 191 links (95\%). In contrast, maximizing RSS only gets within 90\% of optimal for 123 links (61\%).

\begin{figure}[htp]
	\centering
	\includegraphics[width=0.6\textwidth]{figures/esnr/chan_sel_delta_opt.png}
	\caption{\label{fig:chan_sel_delta_opt}Channel selection algorithm performance loss from optimal algorithm.}
\end{figure}

Next, I compare the absolute performance loss of the channel selection algorithms in \figref{fig:chan_sel_delta_opt}. Here, each $(x,y)$ point represents the fraction of links $y$ that choose a channel within $y$\Mbps of the best channel with a particular algorithm. The area over the curves represents the performance lost by each algorithm; an accurate algorithm would be in the top left corner, losing little throughput for only a few links. This area is 2.9$\times$ larger for the RSS-based selection algorithm, showing that Effective SNR is significantly more accurate. This difference translates to about 10\Mbps more per link when selecting channels via the Effective SNR.

\begin{figure}[htp]
	\centering
	\includegraphics[width=0.6\textwidth]{figures/esnr/chan_sel_ratio.png}
	\caption{\label{fig:chan_sel_ratio}Channel selection performance with Effective SNR relative to RSS\@.}
\end{figure}

Finally, I examine the head-to-head performance of selecting channels via Effective SNR or RSS in \figref{fig:chan_sel_ratio}. For each link, I plot the ratio of the performance of the channel picked with the Effective SNR strategy to that of the channel chosen by maximizing RSS\@. If the two channels perform equally, the ratio will be 1; if the Effective SNR strategy chose a better channel the ratio will be larger. The algorithms choose channels with equal performance for 90 (45\%) of the 201 links, while the Effective SNR-based algorithm chooses a better channel for 88 (44\%) links and the RSS-based algorithm for the remaining 23 (11\%). Additionally, the gains from Effective SNR are much larger than its losses: the RSS strategy chooses a channel that performs 20\% better than the Effective SNR-selected channel for only 6/23 (26\%) links its channel is better, while Effective SNR chooses a 20\% better channel for 41/88 (47\%) of cases. In other words, the Effective SNR channel selection algorithm is more likely to pick a better channel by a factor of about 4 (88/23), and this difference is more likely to be significant by a factor of about 2 (47\%/26\%).

In conclusion, these results shows that both Effective SNR- and RSS-based channel selection strategies perform well for 3-stream IEEE 802.11n links. However, the Effective SNR channel selection strategy is significantly more accurate: it chooses an optimal channel for 66\% more links, it offers about 10\Mbps more per link when selecting suboptimal channels, and it is more likely to choose a better channel by a factor of 4.

%%%%%%%%%%%%%%%%%%%%%%%%%%%%%%%%%%%%%%
\subsubsection{Channel discrimination}
\begin{itemize}
\item with multiple antennas, all channels within a band are roughly equivalent RSS. in 5\GHz, can see variability plausibly attributable to the actual antenna response.

\begin{figure}[htp]
	\centering
	\includegraphics[width=\textwidth]{figures/esnr/rssi_vs_freq_24.png}
	\caption{\label{fig:rssi_vs_freq_24}RSS versus 2.4\GHz channel for 249 links. Solid line shows the median for each channel.}
\end{figure}
\begin{figure}[htp]
	\centering
	\includegraphics[width=\textwidth]{figures/esnr/rssi_vs_freq_5.png}
	\caption{\label{fig:rssi_vs_freq_5}RSS versus 5\GHz channel for 163 links. Solid line shows the median for each channel.}
\end{figure}
\begin{figure}[htp]
	\centering
	\includegraphics[width=\textwidth]{figures/esnr/rssi_freq_dev.png}
	\caption{\label{fig:rssi_freq_dev}Standard deviation and max-min spread of RSS across channels within the same band.}
\end{figure}

\item \figref{fig:rel_diff_draws} shows that, depending on how close to optimal you want to be, need to look at 5, 10, or 15 of the 24 channels in 5\GHz band. If we can evaluate the rate offered by a channel quickly, we can look at more channels in the same amount of time to pick the best rate.
\end{itemize}

\begin{figure}[htp]
	\centering
	\includegraphics[width=0.6\textwidth]{figures/esnr/rel_diff_draws.pdf}
	\caption{\label{fig:rel_diff_draws}The expected rate after choosing the best of $k$ 802.11n channels.}
\end{figure}

%%%%%%%%%%%%%%%%%%%%%%%%%%%%%%%%%%%%%%%%%%%%%%%%%%%%%%%%%%%%%%%%%%%%%%%%%%%%%%%%%%%%%%%%%%%%%%%%%%%%%%%%%%%%%%%%%%%%%%%%%%%%%%%%%%%%%%%%%
\subsection{Further Evaluations?}
\heading{Performance.}  given time to hop $\mathcal{O}$(1\ms), time to execute a CSI probe $\mathcal{O}$(500\us), and time to execute a rate probe (unknown), how many channels can we look at in $T$ time? Reference \figref{fig:rel_diff_draws} to see what fraction of optimal this enables.

\heading{Completion Time.} The difference between the Effective SNR and the SNR is a proxy for how ``good'' a channel is, based on how flat it is. Given that RSS is similar across channels, the flattest channel will likely offer the best rate. Can we use this to detect a good channel and stop looking early?
%%%%%%%%%%%%%%%%%%%%%%%%%%%%%%%%%%%%%%%%%%%%%%%%%%%%%%%%%%%%%%%%%%%%%%%%%%%%%%%%%%%%%%%%%%%%%%%%%%%%%%%%%%%%%%%%%%%%%%%%%%%%%%%%%%%%%%%%%
\section{Path selection}\label{sec:esnr_pathsel}
In the last section, we looked at the benefits of AP selection; we now consider the more general problem of path selection. While today's AP and WDS\footnote{Actually, a WDS client implicitly makes a routing decision when joining the network---which access point it chooses can make a large difference in its connection quality.} networks use tree-structured topologies and have only a single path between any two nodes, a future device-to-device wireless network may offer many paths along which packets can be routed. Research in multi-hop routing for wireless mesh networks~\cite{Bahl_repeater,Rodrig_thesis} has shown that the choice of path can effect a large difference in connection quality.

The practical state of the art in this area is the recent work by Bahl et al.~\cite{Bahl_repeater} on an opportunistic repeater scheme for 802.11a. In this design, when a client with a strong link detects rate anomaly~\cite{Heusse_RateAnomaly}---that is, that its throughput is hurt by a client with a weak link monopolizing airtime---the strong client evaluates whether relaying that client's packets would improve throughout for both. In certain scenarios, they showed that this could improve aggregate performance of the network by 50\%--200\%.

While this solution is practical and effective, the use of 802.11n networks significantly complicates the picture. First, the scheme of Bahl et al.\ uses a link's RSSI to select between the 8 available 802.11a rates. In contrast, as we have shown in \chapref{chap:delivery}, RSSI does not accurately predict the rate for 802.11n links, nor does it enable devices to choose between different MIMO modes. Bahl et al.\ used a homogenous network of single-antenna 802.11a chipsets; the set of devices in 802.11n networks includes those with differing numbers of antennas and asymmetric transmit/receive capabilities. While it is not clear how to handle these challenges via the RSSI, the Effective SNR offers the ability to overcome them. In this section, we evaluate the ability of Effective SNR to deliver the benefits of opportunistic repeaters in 802.11n networks.

Note that the problem of path selection does not differ significantly from that of AP selection, except that when choosing between repeaters (or a direct link) the entire path must be considered rather than merely the last hop.\footnote{For simplicity, we assume that the network diameter is small such that pipelining~\cite{Rodrig_thesis} is of limited benefit, and do not consider schemes that forward along multiple unreliable paths such as ExOR~\cite{Biswas_ExOR}.} Instead, in this section we focus on how 802.11n and heterogeneous devices change the opportunities available from relaying, and whether Effective SNR delivers these improvements.

\subsection{Measurements on 802.11a vs 802.11n}
Denote node in center of testbed as AP, and pick a channel. Consider nodes in decreasing order of RSS: have them associate to the network, then turn into repeaters from which the next node can choose. Assume optimal decisions are made at each step. Compare 1x1, 1x3, and 3x3 versions of this scenario.
\begin{itemize}
\item What is the distribution of distance (\#hops) from each node to AP? [How often is repeating used, and at what scale?]
\item What is the distribution of end-to-end tpt? Of the fraction of max (i.e., 65\Mbps or 195\Mbps)? Of the improvement? [This gets at whether the gains get larger or smaller with various device changes.]
\end{itemize}
Same scenario with randomly assigned 3x3, 2x3, and 1x3 devices. How does heterogeneity affect these results?

\subsection{Measurements of Effective SNR}
Perform the same experiments as described above, this time predicting rate by RSSI and then by Effective SNR\@. (Use this only for topology choice, but assume rate selection finds the correct rate.)
%%%%%%%%%%%%%%%%%%%%%%%%%%%%%%%%%%%%%%%%%%%%%%%%%%%%%%%%%%%%%%%%%%%%%%%%%%%%%%%%%%%%%%%%%%%%%%%%%%%%%%%%%%%%%%%%%%%%%%%%%%%%%%%%%%%%%%%%%
\section{Mobility classification}\label{sec:esnr_mobility}
The previous three applications in this chapter used the Effective SNR in algorithms that configure various network parameters. In this section, I use the Channel State Information (CSI) underlying the Effective SNR model to determine whether a wireless device is moving. Though this application does not directly use the Effective SNR, this primitive provides an important complement to the network-level configuration problems.

In wireless systems, simply knowing whether a device is mobile can improve performance and reliability. For example, recent work of Ravindranath et al.\ \cite{Ravindranath_SensorHints} demonstrated a system that improved 802.11a performance on a mobile phone by selecting between different bitrate adaptation algorithms based on whether the device was moving. When the device is static, they use algorithms that can conduct a fine-grained search of the rate space to choose the optimum bitrate. When the device is moving, they use an algorithm that performs a coarser search, but does a better job of tracking a moving optimum. In their experiments, the fine-grained algorithms performed 10\%--30\% better in static scenarios, while the coarse-grained algorithm performed 25\%--75\% better in mobile scenarios.

Detecting mobility can also be used to enhance reliability in networks that support dynamic topology, such as today's cellular phone networks, enterprise Wi-Fi wireless distribution systems (WDSes), and networks that support relaying mechanisms such as described above. By proactively looking for a better AP or relay when the device starts moving, service quality can be improved and downtime reduced. \xxx{find some references about cell handoff, WDS handoff, etc.}

The implementation by Ravindranath et al.\ detected mobility using the accelerometer in a mobile phone. While this technique is accurate and responsive, it has a few disadvantages. The use of an on-board sensor means that detection can only be performed by the mobile client, and thus requires protocol changes to communicate a device's mobile state to the other endpoint of the link: this solution is not backwards-compatible. Also, this technique can only be implemented on devices that have accelerometers, and requires that this sensor be powered on.

In this section, I explore whether it is possible to classify whether a device is mobile based solely on passively measured RF information. If successful, such an implementation would eliminate all of these drawbacks by requiring no extra hardware and supporting unilateral adoption by either endpoint of the link, including the static device. Ravindranath et al.\ made a preliminary attempt to classify mobility using RSSI, but were not successful. They list three challenges: (1) that RSSI is unstable even for static links in a quiet environment; (2) that RSSI varies by different amounts at different absolute signal strengths, and thus needs to be calibrated; and (3) that RSSI was extremely sensitive to movement in the environment and triggered many false hints. Here, I show that the CSI can overcome these challenges and provide a robust solution.

\subsection{Experimental setup}
I configured a SIMO experiment using a single-antenna laptop as the client device, and several of the testbed nodes as three-antenna monitors. The client sent 100,000 back-to-back short packets using \mcs{0} (1 stream, 6.5\Mbps), approximately one packet every 200\us for 20\s. In my initial data collection described here, I took four traces. Two of the traces were taken with a \emph{static} client in the UW CSE Networking Lab and students present, but not moving in the room. I then took a trace with \emph{environmental mobility} in which I left the client static, but waved my hand within a few centimeters of the antenna and then walked around the room and opened doors. Finally, I took a \emph{mobile device} trace in which I picked up the laptop and moved it around within a meter of its original location. Chronologically, the traces were taken in the order described within a 10-minute window, with the second static trace taken last. \xxx{These results are from only a single receiver and a single mobile experiment; I could look at more traces and conduct more experiments to flesh out the results and to address claim (2) above.}

\subsection{Classifying mobility with RSSI}
Ravindranath et al.\ argue that it is difficult to classify mobility using RSSI. To confirm that this is indeed the case, I analyzed RSSI variation over time for these four traces. \figref{fig:mobility_rssi} shows the RSSI in dBm measured by one receiver for each scenario. In each plot, the three lines each show the RSSI for one of the three receive antennas.

I note several interesting effects visible in these measurements. First, the RSSI is actually extremely stable in static scenarios. This deviation from the observations by Ravindranath et al.\ is likely attributable to the better calibration of the newer 802.11n hardware used, compared with older hardware used to run experiments with the MadWiFi driver.

Second, though RSSI does vary with environmental mobility, the variation is fairly small and mostly limited to the periods of activity directly next to the client. Later in the trace, when I moved across the room, the RSSI variation decreased to match the static scenario. It also appears that the variation is not completely correlated across antennas; in several parts of the trace (e.g., at the beginning and around 10\s--12\s) one or two antennas see variation in RSSI while the others do not. These periods of particle variation may be indicative of a static device with environmental movement.

Finally, the mobile trace exhibits the RSSI variation with the largest magnitude, and shows consistent variation throughout the trace and across all antennas. This is a dramatic outlier compared to the other traces, and strongly reflects the effects of movement.

Based on this visual evidence, I believe it likely that the static scenario can be identified using RSSI, and hypothesize that it may also be possible to distinguish between environmental and device mobility. However, I deferred exploring these possibility further because, as I will show next, the CSI can conclusively classify a device's activity into these three states.

\begin{figure}[htp]
	\centering
	\subfigure[Static Environment]{
		\includegraphics[width=0.48\textwidth]{figures/applications/time_vs_rss_static.pdf}%
	}\hfill%
	\subfigure[Static Environment, trial 2]{
		\includegraphics[width=0.48\textwidth]{figures/applications/time_vs_rss_static2.pdf}%
	}
	
	\subfigure[Environmental Mobility]{
		\includegraphics[width=0.48\textwidth]{figures/applications/time_vs_rss_enviro.pdf}%
	}\hfill%
	\subfigure[Mobile Device]{
		\includegraphics[width=0.48\textwidth]{figures/applications/time_vs_rss_mobile.pdf}%
	}
	\caption{\label{fig:mobility_rssi}RSSI variation in different mobility scenarios.}
\end{figure}

\subsection{Classifying mobility with CSI} Here, I examine the same four traces through the lens of the CSI.

To start, recall that the RSSI yields a single power measurement per sample, whereas the CSI gives a 3-D matrix of complex numbers that represent magnitude and phase on spatial paths and frequency. To measure the deviation in RSSI, we could simply look at its variation---e.g., absolute difference between samples, or windowed variance---over time, as I showed visually in \figref{fig:mobility_rssi}. In contrast, it is not obvious how to quantify the variation in CSI over time.

\subsubsection{Pearson correlation}
I chose a simple way to quantify the variation of CSI, by using the \define{Pearson correlation function} for each spatial path between a transmit-receive antenna pair. The Pearson correlation is the ``standard'' correlation function for two $n$-element vectors $\vec{x}$ and $\vec{y}$, and is defined as
\begin{equation}
\textit{corr}(\vec{x},\vec{y}) = \frac{\sum_{i=1}^n(x_i-\overline{x})(y_i-\overline{y})}{\sqrt{\sum_{i=1}^n(x_i-\overline{x})^2 \sum_{i=1}^n(y_i-\overline{y})^2}}.
\end{equation}
Here $\vec{x},\vec{y}$ are indexed by $i$ and have respective means $\overline{x}$ and $\overline{y}$.

To apply this to CSI, let $\vec{r}_{p,t}$ represent the magnitudes of the CSI coefficients across subcarriers for spatial path $p$ at time sample $t$. Then we can quantify the change between sample $t$ and sample $t+1$ by $\textit{corr}(\vec{r}_{p,t},\vec{r}_{p,(t+1)})$. The correlation will be close to 1 if the CSI matches across time, i.e., the channel is not changing, and closer to zero if the CSI samples vary greatly.

\subsubsection{Results}
I present these correlations over time for the four traces in \figref{fig:mobility_csi}. Again, each plot shows one line for each spatial path between the 1 transmit antenna and the 3 receive antennas. This figures show that the static traces have near-perfect correlation, the environmental mobility trace shows a little deviation, and the mobile trace varies wildly with correlations as low as 0.3. These plots confirm that the Pearson correlation can be used to accurately classify whether a device is moving, and whether the environment is changing.

\figref{fig:mobility_csi_cdf} shows the CDF of the correlation (combined across antennas) over time. Static traces never show a correlation below 0.98; the trace with environmental mobility never drops to 0.9, and about 3\% of the correlations in the mobile trace are below 0.9. Though the low-correlation outliers occur infrequently, the fact that they are distributed throughout the traces means a windowed thresholding will accurately be able to distinguish between these three states. Note that the mobility state of a device will change slowly---on the order of seconds or longer---and the outliers are frequent enough at this time scale to prevent false negatives.

%In wireless networks today, laptops tend to be ``portable, but not mobile''~\cite{Woodruff_portable}. That is, though they can move from location to location, laptops are infrequently used while actually in motion.

\begin{figure}[htp]
	\centering
	\subfigure[Static Environment]{
		\includegraphics[width=0.48\textwidth]{figures/applications/time_vs_csi_static.pdf}%
	}\hfill%
	\subfigure[Static Environment, trial 2]{
		\includegraphics[width=0.48\textwidth]{figures/applications/time_vs_csi_static2.pdf}%
	}
	
	\subfigure[Environmental Mobility]{
		\includegraphics[width=0.48\textwidth]{figures/applications/time_vs_csi_enviro.pdf}%
	}\hfill%
	\subfigure[Mobile Device]{
		\includegraphics[width=0.48\textwidth]{figures/applications/time_vs_csi_mobile.pdf}%
	}
	\caption{\label{fig:mobility_csi}CSI variation as measured by correlation in different mobility scenarios.}
\end{figure}
\begin{figure}[htp]
	\centering
	\includegraphics[width=\textwidth]{figures/esnr/mobility_csi_cdf.png}
	\caption{\label{fig:mobility_csi_cdf}CDF of CSI variation as measured by correlation in different mobility scenarios.}
\end{figure}


%%%%%%%%%%%%%%%%%%%%%%%%%%%%%%%%%%
\ifx\mainfile\undefined
%
% ==========   Bibliography   ==========
%
%\nocite{*}   % include everything in the uwthesis.bib file
\bibliographystyle{plain}
\bibliography{dhalperi_thesis}

\end{document}
\fi