%%%%%%%%%%%%%%%%%%%%%%%%%%%%%%%%%%%%%%%%%%%%%%%%%%%%%%%%%%%%%%%%%%%%%%%%%%%%%%%%%%%%%%%%%%%%%%%%%%%%%%%%%%%%%%%%%%%%%%%%%%%%%%%%%%%%%%%%%
\section{Mobility classification}\label{sec:esnr_mobility}
The previous three applications in this chapter used the Effective SNR in algorithms that configure various network parameters. In this section, I use the Channel State Information (CSI) underlying the Effective SNR model to determine whether a wireless device is moving. Though this application does not directly use the Effective SNR, this primitive provides an important complement to the network-level configuration problems.

In wireless systems, simply knowing whether a device is mobile can improve performance and reliability. For example, recent work of Ravindranath et al.\ \cite{Ravindranath_SensorHints} demonstrated a system that improved 802.11a performance on a mobile phone by selecting between different bitrate adaptation algorithms based on whether the device was moving. When the device is static, they use algorithms that can conduct a fine-grained search of the rate space to choose the optimum bitrate. When the device is moving, they use an algorithm that performs a coarser search, but does a better job of tracking a moving optimum. In their experiments, the fine-grained algorithms performed 10\%--30\% better in static scenarios, while the coarse-grained algorithm performed 25\%--75\% better in mobile scenarios.

Detecting mobility can also be used to enhance reliability in networks that support dynamic topology, such as today's cellular phone networks, enterprise Wi-Fi wireless distribution systems (WDSes), and networks that support relaying mechanisms such as described above. By proactively looking for a better AP or relay when the device starts moving, service quality can be improved and downtime reduced. \xxx{find some references about cell handoff, WDS handoff, etc.}

The implementation by Ravindranath et al.\ detected mobility using the accelerometer in a mobile phone. While this technique is accurate and responsive, it has a few disadvantages. The use of an on-board sensor means that detection can only be performed by the mobile client, and thus requires protocol changes to communicate a device's mobile state to the other endpoint of the link: this solution is not backwards-compatible. Also, this technique can only be implemented on devices that have accelerometers, and requires that this sensor be powered on.

In this section, I explore whether it is possible to classify whether a device is mobile based solely on passively measured RF information. If successful, such an implementation would eliminate all of these drawbacks by requiring no extra hardware and supporting unilateral adoption by either endpoint of the link, including the static device. Ravindranath et al.\ made a preliminary attempt to classify mobility using RSSI, but were not successful. They list three challenges: (1) that RSSI is unstable even for static links in a quiet environment; (2) that RSSI varies by different amounts at different absolute signal strengths, and thus needs to be calibrated; and (3) that RSSI was extremely sensitive to movement in the environment and triggered many false hints. Here, I show that the CSI can overcome these challenges and provide a robust solution.

\subsection{Experimental setup}
I configured a SIMO experiment using a single-antenna laptop as the client device, and several of the testbed nodes as three-antenna monitors. The client sent 100,000 back-to-back short packets using \mcs{0} (1 stream, 6.5\Mbps), approximately one packet every 200\us for 20\s. In my initial data collection described here, I took four traces. Two of the traces were taken with a \emph{static} client in the UW CSE Networking Lab and students present, but not moving in the room. I then took a trace with \emph{environmental mobility} in which I left the client static, but waved my hand within a few centimeters of the antenna and then walked around the room and opened doors. Finally, I took a \emph{mobile device} trace in which I picked up the laptop and moved it around within a meter of its original location. Chronologically, the traces were taken in the order described within a 10-minute window, with the second static trace taken last. \xxx{These results are from only a single receiver and a single mobile experiment; I could look at more traces and conduct more experiments to flesh out the results and to address claim (2) above.}

\subsection{Classifying mobility with RSSI}
Ravindranath et al.\ argue that it is difficult to classify mobility using RSSI. To confirm that this is indeed the case, I analyzed RSSI variation over time for these four traces. \figref{fig:mobility_rssi} shows the RSSI in dBm measured by one receiver for each scenario. In each plot, the three lines each show the RSSI for one of the three receive antennas.

I note several interesting effects visible in these measurements. First, the RSSI is actually extremely stable in static scenarios. This deviation from the observations by Ravindranath et al.\ is likely attributable to the better calibration of the newer 802.11n hardware used, compared with older hardware used to run experiments with the MadWiFi driver.

Second, though RSSI does vary with environmental mobility, the variation is fairly small and mostly limited to the periods of activity directly next to the client. Later in the trace, when I moved across the room, the RSSI variation decreased to match the static scenario. It also appears that the variation is not completely correlated across antennas; in several parts of the trace (e.g., at the beginning and around 10\s--12\s) one or two antennas see variation in RSSI while the others do not. These periods of particle variation may be indicative of a static device with environmental movement.

Finally, the mobile trace exhibits the RSSI variation with the largest magnitude, and shows consistent variation throughout the trace and across all antennas. This is a dramatic outlier compared to the other traces, and strongly reflects the effects of movement.

Based on this visual evidence, I believe it likely that the static scenario can be identified using RSSI, and hypothesize that it may also be possible to distinguish between environmental and device mobility. However, I deferred exploring these possibility further because, as I will show next, the CSI can conclusively classify a device's activity into these three states.

\begin{figure}[htp]
	\centering
	\subfigure[Static Environment]{
		\includegraphics[width=0.48\textwidth]{figures/applications/time_vs_rss_static.pdf}%
	}\hfill%
	\subfigure[Static Environment, trial 2]{
		\includegraphics[width=0.48\textwidth]{figures/applications/time_vs_rss_static2.pdf}%
	}
	
	\subfigure[Environmental Mobility]{
		\includegraphics[width=0.48\textwidth]{figures/applications/time_vs_rss_enviro.pdf}%
	}\hfill%
	\subfigure[Mobile Device]{
		\includegraphics[width=0.48\textwidth]{figures/applications/time_vs_rss_mobile.pdf}%
	}
	\caption{\label{fig:mobility_rssi}RSSI variation in different mobility scenarios.}
\end{figure}

\subsection{Classifying mobility with CSI} Here, I examine the same four traces through the lens of the CSI.

To start, recall that the RSSI yields a single power measurement per sample, whereas the CSI gives a 3-D matrix of complex numbers that represent magnitude and phase on spatial paths and frequency. To measure the deviation in RSSI, we could simply look at its variation---e.g., absolute difference between samples, or windowed variance---over time, as I showed visually in \figref{fig:mobility_rssi}. In contrast, it is not obvious how to quantify the variation in CSI over time.

\subsubsection{Pearson correlation}
I chose a simple way to quantify the variation of CSI, by using the \define{Pearson correlation function} for each spatial path between a transmit-receive antenna pair. The Pearson correlation is the ``standard'' correlation function for two $n$-element vectors $\vec{x}$ and $\vec{y}$, and is defined as
\begin{equation}
\textit{corr}(\vec{x},\vec{y}) = \frac{\sum_{i=1}^n(x_i-\overline{x})(y_i-\overline{y})}{\sqrt{\sum_{i=1}^n(x_i-\overline{x})^2 \sum_{i=1}^n(y_i-\overline{y})^2}}.
\end{equation}
Here $\vec{x},\vec{y}$ are indexed by $i$ and have respective means $\overline{x}$ and $\overline{y}$.

To apply this to CSI, let $\vec{r}_{p,t}$ represent the magnitudes of the CSI coefficients across subcarriers for spatial path $p$ at time sample $t$. Then we can quantify the change between sample $t$ and sample $t+1$ by $\textit{corr}(\vec{r}_{p,t},\vec{r}_{p,(t+1)})$. The correlation will be close to 1 if the CSI matches across time, i.e., the channel is not changing, and closer to zero if the CSI samples vary greatly.

\subsubsection{Results}
I present these correlations over time for the four traces in \figref{fig:mobility_csi}. Again, each plot shows one line for each spatial path between the 1 transmit antenna and the 3 receive antennas. This figures show that the static traces have near-perfect correlation, the environmental mobility trace shows a little deviation, and the mobile trace varies wildly with correlations as low as 0.3. These plots confirm that the Pearson correlation can be used to accurately classify whether a device is moving, and whether the environment is changing.

\figref{fig:mobility_csi_cdf} shows the CDF of the correlation (combined across antennas) over time. Static traces never show a correlation below 0.98; the trace with environmental mobility never drops to 0.9, and about 3\% of the correlations in the mobile trace are below 0.9. Though the low-correlation outliers occur infrequently, the fact that they are distributed throughout the traces means a windowed thresholding will accurately be able to distinguish between these three states. Note that the mobility state of a device will change slowly---on the order of seconds or longer---and the outliers are frequent enough at this time scale to prevent false negatives.

%In wireless networks today, laptops tend to be ``portable, but not mobile''~\cite{Woodruff_portable}. That is, though they can move from location to location, laptops are infrequently used while actually in motion.

\begin{figure}[htp]
	\centering
	\subfigure[Static Environment]{
		\includegraphics[width=0.48\textwidth]{figures/applications/time_vs_csi_static.pdf}%
	}\hfill%
	\subfigure[Static Environment, trial 2]{
		\includegraphics[width=0.48\textwidth]{figures/applications/time_vs_csi_static2.pdf}%
	}
	
	\subfigure[Environmental Mobility]{
		\includegraphics[width=0.48\textwidth]{figures/applications/time_vs_csi_enviro.pdf}%
	}\hfill%
	\subfigure[Mobile Device]{
		\includegraphics[width=0.48\textwidth]{figures/applications/time_vs_csi_mobile.pdf}%
	}
	\caption{\label{fig:mobility_csi}CSI variation as measured by correlation in different mobility scenarios.}
\end{figure}
\begin{figure}[htp]
	\centering
	\includegraphics[width=\textwidth]{figures/esnr/mobility_csi_cdf.png}
	\caption{\label{fig:mobility_csi_cdf}CDF of CSI variation as measured by correlation in different mobility scenarios.}
\end{figure}
