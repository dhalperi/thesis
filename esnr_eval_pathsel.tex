%%%%%%%%%%%%%%%%%%%%%%%%%%%%%%%%%%%%%%%%%%%%%%%%%%%%%%%%%%%%%%%%%%%%%%%%%%%%%%%%%%%%%%%%%%%%%%%%%%%%%%%%%%%%%%%%%%%%%%%%%%%%%%%%%%%%%%%%%
\section{Path selection}\label{sec:esnr_pathsel}
While today's access point and wireless distribution system (WDS) infrastructure networks use tree-structured topologies and have only a single path between any two nodes, a future device-to-device wireless network such as Wi-Fi Direct may offer many paths along which packets can be routed. Choosing which path in a multi-hop wireless network will provide the best throughput is the \define{path selection} problem, and it can be thought of as a generalization of the access point selection problem I described in \secref{sec:esnr_apsel}. Indeed, a client implicitly makes a routing decision when joining a WDS network---which access point it chooses can make a large difference in its connection quality to the root access point that serves Internet access. Research in multi-hop routing for wireless mesh networks~\cite{Bahl_repeater,Rodrig_thesis} has shown that the choice of path can effect a large difference in connection quality.

The practical state of the art in this area is the recent work by Bahl et al.~\cite{Bahl_repeater} on an opportunistic repeater scheme for 802.11a. In this design, when a client with a strong link detects rate anomaly~\cite{Heusse_RateAnomaly}---that is, that its throughput is hurt by a client with a weak link monopolizing airtime---the strong client evaluates whether relaying that client's packets would improve throughout for both. In certain scenarios, they showed that this could improve aggregate performance of the network by 50\%--200\%.

While this solution is practical and works well, the use of 802.11n networks significantly complicates the picture. First, the scheme of Bahl et al.\ uses a link's RSSI (as a measure of Packet SNR) to select between the 8 available 802.11a rates. In contrast, as I have shown in \chapref{chap:delivery}, Packet SNR does not accurately predict the rate for 802.11n links, nor does it enable devices to choose between different MIMO modes. Second, Bahl et al.\ used a homogenous network of single-antenna 802.11a chipsets; but the set of devices in 802.11n networks will be heterogeneous and support differing numbers of antennas and asymmetric transmit/receive capabilities. While it is not clear how to handle these challenges via the Packet SNR, the Effective SNR offers the ability to overcome them. In this section, I evaluate the ability of Effective SNR to deliver the benefits of opportunistic repeaters in 802.11n networks.

Note that the problem of path selection does not differ significantly from that of AP selection, except that when choosing between repeaters (or a direct link) the entire path must be considered rather than merely the last hop. For simplicity, I assume that the network diameter is small such that pipelining~\cite{Rodrig_thesis} is of limited benefit, and do not consider schemes that forward along multiple unreliable paths such as ExOR~\cite{Biswas_ExOR}. Instead, in this section I focus on how 802.11n and heterogeneous devices change the opportunities available from relaying, and whether Effective SNR delivers these improvements.

\subsection{Measurements on 802.11a vs 802.11n}
Denote node in center of testbed as AP, and pick a channel. Consider nodes in decreasing order of RSS: have them associate to the network, then turn into repeaters from which the next node can choose. Assume optimal decisions are made at each step. Compare 1x1, 1x3, and 3x3 versions of this scenario.
\begin{itemize}
\item What is the distribution of distance (\#hops) from each node to AP? [How often is repeating used, and at what scale?]
\item What is the distribution of end-to-end tpt? Of the fraction of max (i.e., 65\Mbps or 195\Mbps)? Of the improvement? [This gets at whether the gains get larger or smaller with various device changes.]
\end{itemize}
Same scenario with randomly assigned 3x3, 2x3, and 1x3 devices. How does heterogeneity affect these results?

\subsection{Measurements of Effective SNR}
Perform the same experiments as described above, this time predicting rate by RSSI and then by Effective SNR\@. (Use this only for topology choice, but assume rate selection finds the correct rate.)