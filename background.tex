\ifx\mainfile\undefined
%  ========================================================================
%  Copyright (c) 2006-2011 The University of Washington
%
%  Licensed under the Apache License, Version 2.0 (the "License");
%  you may not use this file except in compliance with the License.
%  You may obtain a copy of the License at
%
%      http://www.apache.org/licenses/LICENSE-2.0
%
%  Unless required by applicable law or agreed to in writing, software
%  distributed under the License is distributed on an "AS IS" BASIS,
%  WITHOUT WARRANTIES OR CONDITIONS OF ANY KIND, either express or implied.
%  See the License for the specific language governing permissions and
%  limitations under the License.
%  ========================================================================
%
 
\documentclass [11pt, twoside] {uwthesis}

\usepackage{color}
\usepackage{url}
\usepackage{amsmath}
\usepackage{amsfonts}
\usepackage[bookmarks,
	hidelinks,
	plainpages=false,
	pdfpagelabels,
	pagebackref=true,
            ]{hyperref}
\renewcommand*{\backref}[1]{}% for backref < 1.33 necessary
\renewcommand*{\backrefalt}[4]{%
  \ifcase #1 %
    (No citations.)%
  \or
    (Cited on page #2.)%
  \else
    (Cited on pages #2.)%
  \fi
}

\newcommand{\biburl}[1]{{\tt<}\url{#1}{\tt>}}

\hypersetup{%
pdfauthor = {Daniel Chaim Halperin},
pdftitle = {Simplifying the Configuration of 802.11 Wireless Networks with Effective SNR},
pdfsubject = {Ph.D. Dissertation},
pdfkeywords = {},
pdfcreator = {University of Washington, Computer Science and Engineering},
pdfproducer = {},
bookmarksopen = {true},
pdfpagelayout = {TwoColumnRight},
}

\usepackage{footnotebackref}
%%%%%%%%%%%%%%%%%%%%%%%%%%%%%%%%%%%%%%%%%%%%%%%%%%%%%%
%%%        Formatting sections                     %%%
%%%%%%%%%%%%%%%%%%%%%%%%%%%%%%%%%%%%%%%%%%%%%%%%%%%%%%
\newcommand{\algref}[1]{Algorithm~\ref{#1}}
\newcommand{\chapref}[1]{Chapter~\ref{#1}}
\renewcommand{\eqref}[1]{Equation~\ref{#1}}
\newcommand{\figref}[1]{Figure~\ref{#1}}
\newcommand{\secref}[1]{\S\ref{#1}}
\newcommand{\tabref}[1]{Table~\ref{#1}}
\newcommand{\heading}[1]{\vspace{4pt}\noindent\textbf{#1}}
\newcommand{\topheading}[1]{\noindent\textbf{#1}}
\newcommand{\noheading}[0]{\vspace{4pt}\noindent}

%%%%%%%%%%%%%%%%%%%%%%%%%%%%%%%%%%%%%%%%%%%%%%%%%%%%%%
%%%        XXX and other warnings                  %%%
%%%%%%%%%%%%%%%%%%%%%%%%%%%%%%%%%%%%%%%%%%%%%%%%%%%%%%
\newcommand{\xxx}[1]{\textit{\color{red}XXX #1}}

%%%%%%%%%%%%%%%%%%%%%%%%%%%%%%%%%%%%%%%%%%%%%%%%%%%%%%
%%%        Units                                   %%%
%%%%%%%%%%%%%%%%%%%%%%%%%%%%%%%%%%%%%%%%%%%%%%%%%%%%%%
\usepackage{xspace}
\newcommand{\unitsep}{\texorpdfstring{\,}{ }}
\def\unit#1{% from: http://www.tex.ac.uk/cgi-bin/texfaq2html?label=csname "Defining a macro from an argument"
  \expandafter\def\csname #1\endcsname{\unitsep\text{#1}\xspace}%
}
\def\varunit#1#2{% from: http://www.tex.ac.uk/cgi-bin/texfaq2html?label=csname "Defining a macro from an argument"
  \expandafter\def\csname #1\endcsname{\unitsep\text{#2}\xspace}%
}
\unit{GHz}
\unit{MHz}
\unit{kHz}
\unit{Gbps}
\unit{Mbps}
\unit{KB}
\unit{dB}
\unit{dBi}
\unit{dBm}
\unit{W}
\unit{mW}
\varunit{uW}{$\mu$W}
\unit{ms}
\varunit{us}{$\mu$s}
\unit{h}
\unit{m}
\unit{s}
\unit{km}
\unit{cm}
\unit{mm}
\varunit{mmsq}{mm$^\text{2}$}
\varunit{insq}{in$^\text{2}$}
\newcommand{\degree}{\ensuremath{^\circ}\xspace}
\newcommand{\degrees}{\degree}
%%%%%%%%%%%%%%%%%%%%%%%%%%%%%%%%%%%%%%%%%%%%%%%%%%%%%%%%%%%%%%%%%%%%%%%%%%%%%%%%%%%%%%
% Euler for math | Palatino for rm | Helvetica for ss | Courier for tt
%
% From: http://www.tug.org/mactex/fonts/LaTeX_Preamble-Font_Choices.html
%%%%%%%%%%%%%%%%%%%%%%%%%%%%%%%%%%%%%%%%%%%%%%%%%%%%%%%%%%%%%%%%%%%%%%%%%%%%%%%%%%%%%%
\renewcommand{\rmdefault}{ppl} % rm
\usepackage[scaled]{helvet} % ss
\usepackage{courier} % tt
\usepackage{eulervm} % a better implementation of the euler package (not in gwTeX)
\normalfont
\usepackage[T1]{fontenc}
%%%%%%%%%%%%%%%%%%%%%%%%%%%%%%%%%%%%%%%%%%%%%%%%%%%%%%%%%%%%%%%%%%%%%%%%%%%%%%%%%%%%%%

%%%%%%%%%%%%%%%%%%%%%%%%%%%%%%%%%%%%%%%%%%%%%%%%%%%%%%
%%%        Figures                                 %%%
%%%%%%%%%%%%%%%%%%%%%%%%%%%%%%%%%%%%%%%%%%%%%%%%%%%%%%
\usepackage{graphicx}
% Caption package both lets you set the spacing between figure and caption
% and also makes the \figref{} point to the right place.
\usepackage[font=bf,aboveskip=6pt,belowskip=-4mm]{caption}
% Allow subfigures, make them bold
\usepackage[bf,BF,small]{subfigure}
% List of figures
\setcounter{lofdepth}{2}  % Print the chapter and sections to the lot

%%%%%%%%%%%%%%%%%%%%%%%%%%%%%%%%%%%%%%%%%%%%%%%%%%%%%%
%%%        Lists with reduced spacing              %%%
%%%%%%%%%%%%%%%%%%%%%%%%%%%%%%%%%%%%%%%%%%%%%%%%%%%%%%
\usepackage{enumitem}

%%%%%%%%%%%%%%%%%%%%%%%%%%%%%%%%%%%%%%%%%%%%%%%%%%%%%%
%%%        Fancy tables                            %%%
%%%%%%%%%%%%%%%%%%%%%%%%%%%%%%%%%%%%%%%%%%%%%%%%%%%%%%
\usepackage{tabulary}
\usepackage{booktabs}

%%%%%%%%%%%%%%%%%%%%%%%%%%%%%%%%%%%%%%%%%%%%%%%%%%%%%%
%%%        Formatting techniques/tools/etc.        %%%
%%%%%%%%%%%%%%%%%%%%%%%%%%%%%%%%%%%%%%%%%%%%%%%%%%%%%%
\newcommand{\term}[1]{\texttt{#1}}

\begin{document}
 
\textpages
\setcounter{chapter}{1} % Set to n-1!
\fi
%%%%%%%%%%%%%%%%%%%%%%%%%%%%%%%%%%

\cleardoublepage
\chapter{Background}
\label{chap:background}

\begin{table}
\centering
%\footnotesize
\begin{tabular}{cccc}
\toprule
MCS & Modulation & Coding Rate & Data Rate (Mbps) \\
\midrule
0 & BPSK & 1/2 & 6.5 \\
1 & QPSK & 1/2 & 13.0\\
2 & QPSK & 3/4 & 19.5\\
3 & 16-QAM & 1/2 & 26.0\\
4 & 16-QAM & 3/4 & 39.0\\
5 & 64-QAM & 2/3 & 52.0\\
6 & 64-QAM & 3/4 & 58.5\\
7 & 64-QAM & 5/6 & 65.0\\
\bottomrule
\end{tabular}
\caption[The 802.11n single-stream rates.]{\label{tab:siso_mcs} The single-stream 802.11n modulation and coding schemes. The first seven rates correspond to 802.11a/g rates (excluding 9\Mbps) with four added OFDM subcarriers; the highest data rate uses a new 5/6-rate code. The data rates are given for 20\MHz channels with 4\ms symbols.}
\end{table}

\section{Wireless Communication}
\begin{itemize}
\item Fundamental: radio waves, with a carrier frequency $f_c$. Values of $f_c$ relevant to 802.11n are 2.4\GHz and 5\GHz.
\item Data is modulated onto a radio wave by transforming the wave. Simplest way to do this might be sending 1/0 with wave on or off. The rate at which we change what we're sending is called the baud or the symbol rate, and determines the bandwidth of the channel $B$ measured in Hertz (Hz).
\item To send data wirelessly, the transmitter emits a modulated radio wave called the \define{signal}. This signal radiates from the transmitter's antenna and propagates over the air to the receiver's antenna. The quantity $S$ denotes the power of the signal at the receive antenna, and is usually measured in decibels relative to 1~milliwatt, or dBm.
\item The received signal is corrupted by broad-spectrum electromagnetic noise generated by electrons (or other charged particles) that move around inside the conductive material, typically metal, that forms the antenna. This corruption is sometimes called \define{Johnson-Nyquist noise} after its identification in 1927 by Johnson~\cite{Johnson_noise} and explanation in 1928 by Nyquist~\cite{Nyquist_noise}, but is more commonly known as \define{thermal noise}.

Thermal noise is modeled as a complex Gaussian with average \define{noise power} $N$ (in Watts) equal to
\begin{equation}
N = kTB,
\end{equation}
where $k\approx1.38\times10^{-23}$ (in Joules/kelvin) is Boltzmann's constant, $T$ is the temperature (in kelvins), and $B$ is the bandwidth as described above.
%For a device at room temperature ($\approx$ 293\K), we can compute $N$ in dBm using the approximation
%\begin{equation}
%N \text{ (dBm)} = -174 + 10\log_{10}(B).
%\end{equation}
In the context of 802.11, Wi-Fi links typically use bandwidths of 20\MHz or 40\MHz, which correspond to thermal noise levels of $-$101\dBm and $-$98\dBm at room temperature. In practice, we also assume a 10\dB--15\dB \define{noise figure}, a quantity that estimates additional noise added by imperfect analog hardware used in receiver processing.

\item Shannon-Hartley Theorem. The seminal works of Ralph Hartley~\cite{Hartley_law} and Claude Shannon~\cite{Shannon_coding,Shannon_capacity}, proved that the \define{capacity} of a channel---i.e., the maximum data rate $R$ at which the transmitter and receiver can communicate---is determined by the channel's bandwidth ($B$ as above) and its \define{signal-to-noise ratio}. The signal-to-noise ratio, or \define{SNR}, is denoted by $\rho$:
\begin{equation}
\rho = \frac{S}{N}.
\end{equation}
The Shannon-Hartley Theorem~\cite{Shannon_capacity} establishes what is called the \define{Shannon capacity} to be
\begin{equation}
R = B\log(1+\rho).
\end{equation}

\item The above binary modulation system is a scheme called On-Off Keying~(OOK). Could do more bits conveyed by multiple power levels, called Amplitude Modulation~(AM). It turns out that amplitude is only one dimension of light; you can also vary the phase. This gives rise to more efficient schemes such as Phase-Shift Keying~(PSK) or Quadrature Amplitude Modulation~(QAM) which is AM in two dimensions; this can be equivalently represented as a complex number Amplitude and Phase. There are many more, including another fundamental modulation called FM that we won't discuss. PSK and QAM are the relevant modulations for 802.11n.
\item hi
\end{itemize}


\section{The IEEE 802.11n standard}

%%%%%%%%%%%%%%%%%%%%%%%%%%%%%%%%%%
\ifx\mainfile\undefined
%
% ==========   Bibliography   ==========
%
%\nocite{*}   % include everything in the uwthesis.bib file
\bibliographystyle{plain}
\bibliography{dhalperi_thesis}

\end{document}
\fi
