\abstract{
Advances in the price, performance, and power consumption of Wi-Fi (IEEE 802.11) technology have led to the adoption of wireless functionality in diverse consumer electronics. These trends have enabled an exciting vision of rich wireless applications that combine the unique features of different devices for a better user experience. To meet the needs of these applications, a wireless network must be configured well to provide good performance at the physical layer. But because of wireless technology and usage trends, finding these configurations is an increasingly challenging problem.

Wireless configuration objectives range from simply choosing the fastest way to encode data on a single wireless link to the global optimization of many interacting parameters over multiple sets of communicating devices. As more links are involved, as technology advances (e.g., the adoption of OFDM and MIMO techniques in Wi-Fi), and as devices are used in changing wireless channels, the size of the configuration space grows. Thus algorithms must find good operating points among a growing number of options.

The heart of every configuration algorithm is evaluating of the performance of a wireless link in a particular operating point. For example, if we know the performance of all three links between a source, a destination, and a potential relay, we can easily determine whether or not using the relay will improve aggregate throughput. Unfortunately, the two standard approaches to this task fall short. One approach uses aggregate signal strength statistics to estimate performance, but these do not yield accurate predictions of performance. Instead, the approach used in practice measures performance by actually trying the possible configurations. This procedure takes a long time to converge and hence is ill-suited to large configuration spaces, multiple devices, or changing channels, all of which are trends today. As a result, the complexity of practical configuration algorithms is dominated by optimizing this performance estimation step.

In this thesis, I develop a comprehensive way to rapidly and accurately predict the performance of every operating point in a large configuration space. I devise a simple but powerful model that uses a single low-level channel measurement and extrapolates over a wide configuration space. My work makes the most complex step of today's configuration algorithms---estimating the effectiveness of a particular configuration---trivial, achieving better performance in practice and enabling the practical solution of larger problems.
}