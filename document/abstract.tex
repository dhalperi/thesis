\abstract{
Wireless networking technology has become fast, cheap, and low power, and is now being adopted in consumer electronics such as smartphones, printers, speakers, video cameras, televisions, and DVD players. Because of its rapid adoption in a diverse set of devices, Wi-Fi is poised at the heart of the next networking revolution: the combining of these diverse consumer devices to build rich applications that leverage each device's unique features.

Despite these great technology, standardization, and adoption trends, one major factor challenges these future wireless applications. Realizing Wi-Fi's significant potential for speed, capacity, and reliability requires that the network be configured to support the set of devices in the network and how they are used. The issue is that the underlying Wi-Fi technologies and network architectures have become rather complex, and how to configure and control them has become a significant decision problem without a simple, comprehensive solution.

The standard algorithms used to configure networks take a ``try-it-and-see'' approach to choosing parameters. These approaches react slowly to changing channels and generally cannot control multiple parameters---such as wireless bitrate and network topology---concurrently. But these limitations are precisely located where wireless applications are heading, as devices are used in tandem and used while mobile. Instead, Wi-Fi systems need a better way to optimize the operating point of a link and of the network that incorporates key configuration factors and can rapidly respond to changing conditions.

In this thesis, I develop a comprehensive way to configure wireless networks using low-level RF channel measurements with a simple but powerful model that can predict the performance of every operating point in the entire configuration space. This provides a simple, fast, accurate way to configure the network and completely replaces a broad class of complex configuration algorithms.

By using a single channel measurement and extrapolating over a wide configuration space, my approach is considerably more practical than probing everything. By using low-level RF information, my approach is considerably more accurate than approaches that only use high-level signal strength information. Thus my approach represents a great tradeoff between these two extremes, maintaining the flexibility and accuracy of probe-based approaches while achieving the simplicity and low overhead of the latter.

To evaluate my work, I implement my model using a state-of-the-art commodity 802.11n wireless device and evaluate its use in a variety of applications over real links. I find that when my model is integrated into wireless network configuration algorithms, the choices made lead to good performance in practice, and that my techniques can solve joint parameter optimization problems. Together, these show that my model unifies the decision-making components of wireless network configuration algorithms into a single comprehensive framework that is practical and provides good performance.
}