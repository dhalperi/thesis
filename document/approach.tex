\ifx\mainfile\undefined
\input{chapter_head}
\setcounter{chapter}{2} % Set to n-1!
\fi
%%%%%%%%%%%%%%%%%%%%%%%%%%%%%%%%%%

\cleardoublepage
\chapter{Problem and Approach}
\label{chap:problem}
\label{chap:approach}

The problem I study in this thesis is how to inform configuration decisions for wireless networks. I begin this chapter by presenting some of the primary problems in this space.

Next, I discuss how we handle these challenges today. There are two primary classes of techniques: (1) \emph{statistics-based} schemes which only use packet reception or loss as an indicator of the suitability of a particular configuration for a specific channel, and (2) \emph{channel-based} schemes which use measurements of the RF channel to predict packet delivery. Generally, packet delivery or loss is too specific, because a packet probe in one configuration says little about whether another configuration---say, at a different rate or antenna mode----will work similarly. This means that many configurations must be tested. On the other hand, RF channel measurements offer the potential for predicting the performance of a configuration without testing it. However, the aggregate data such as RSSI recorded today and the way they are applied only coarsely reflect the true behavior of wireless operation over the underlying RF channel. Thus RF channel measurements are not presently accurate enough to provide good predictions.

My hypothesis is that it is possible to rapidly and accurately predict how well different configurations of MIMO and OFDM wireless links will perform in practice, using a small set of wireless channel measurements. To do so, I develop a comprehensive system that uses low-level RF channel measurements in conjunction with a simple but powerful model to predict the performance of every operating point in the 802.11n configuration space. 

%According to the standards makers, one of the intended purposes of the RSSI is ``to aid in link optimization algorithms such as roaming decisions''~\cite[\S 19.9.5.10]{80211}.

%\section{Wireless Configuration Problems}
%In this section, I describe the configuration problems that arise in modern wireless links and wireless networks, made concrete in the context of multi-antenna 802.11n.
%
%%%%%%%%%%%%%%%%%%%%%%%%%%%%%%%%%%%%%%%%%%%%%%%%%%%%%%%%%%%%%%%%%%%%%%%%%%%%%%%%%%%%%%%%%%%%%%%%%%%%%%
\section{Problem: Rate Control for a Single Link}
\begin{figure}[t]
	\centering
	\includegraphics[width=\textwidth]{figures/approach/rate_configs.pdf}
	\caption[The rate-related 802.11n configurations that use three antennas]{\label{fig:rate_configs}The different rate-related 802.11n configurations that use three antennas and 802.11n physical layer enhancements.}
\end{figure}

The problem of rate control is to find a rate configuration that can successfully deliver packets while maximizing performance.  With 802.11n and its modern multi-antenna and physical layer techniques, this problem has become significantly more complex. To illustrate this, \figref{fig:rate_configs} shows the available rate configurations in 802.11n for a device with three antennas. These configurations use the eight modulation and coding scheme (MCS) combinations described in \tabref{tab:siso_mcs} and the 802.11n enhancements shown in \tabref{tab:11n_enhancements}.

At the bottom of the figure, the SIMO line shows the eight single-stream configurations, which provide rates ranging from 6.5\Mbps to 65\Mbps. These are precisely the eight choices for rate that algorithms controlling a legacy 802.11a/g system must choose from.

This space expands by a factor of 12 when using 802.11n with three antennas. Adding a second (MIMO2) and third (MIMO3) spatial stream increases the maximum rate to 195\Mbps, for a total of 24 different configurations. For each of these configurations, 802.11n adds the optional use of double-width (40\MHz, HT40) channels that raises the maximum rate to 405\Mbps with 48 choices. Finally, a physical layer tweak to a shorter OFDM guard interval (SGI) adds another $\approx 11\%$ and pushes the fastest configuration to 450\Mbps among 96 possibilities.

Looking forward, though three antennas are common today 802.11n can support up to four. The next amendment (802.11ac) will add two new single-stream rates using the 256-QAM modulation, plus channels of up to 160\MHz bandwidth. Combined, 802.11ac will comprise 320 configurations for rate alone---a factor of 40 more than 802.11a. Note that this analysis ignores related configurations such as antenna selection and beamforming, which exacerbates the problem. For the foreseeable future, the dramatic expansion in the rate space will continue as wireless technology evolves.

Now having defined the basic problem of rate control, in the next two sections I describe the statistics-based and the channel-based approaches to solving it.

%%%%%%%%%%%%%%%%%%%%%%%%%%%%%%%%%%%%%%%%%%%%%%%%%%%%%%%%%%%%%%%%%%%%%%%%%%%
\section{Existing Statistics-based Approaches}
The majority of rate control algorithms today rely on packet loss statistics to adapt the operating rate. These algorithms the number of losses as a signal of link quality. With too many losses, the channel is too poor to support the current rate, and the system should fall back to a lower rate. Conversely, when a link experiences a very small number of losses using its current rate, it sends some \emph{packet probes} at a new faster rate, and switches to the faster rate if those probes succeed. \figref{fig:search_11a} shows a typical 802.11a adaptation search pattern, where each box corresponds to an 802.11a rate and the arrows show the faster (solid) and slower (dashed) rates that might be probed.

The first algorithms (ARF~\cite{Kamerman_ARF} and AARF~\cite{Lacage_AARF}) would only switch between a rate and the next fastest or slowest; later implementations look up to two rates ahead~\cite{Bicket_SampleRate}. The dominant 802.11a rate adaptation algorithm used today is minstrel~\cite{minstrel}, which performs intelligent (biased) sampling of \emph{all} rates to keep up-to-date estimates of the global rate space and can thus take discontinuous jumps. Recent revisions to these algorithms have focused on better handling of corner cases, such as improving performance when there is interference via adaptive control over RTS/CTS~\cite{minstrel,Wong_RRAA}.

\begin{figure}[t]
      \centering
      \includegraphics[scale=0.4]{figures/approach/search_11a.pdf}
      \caption[Rate adaptation search pattern for 802.11a]{\label{fig:search_11a}A typical rate adaptation search pattern for 802.11a. When the current 18\Mbps rate works well, a rate control algorithm might probe one or two faster rates (solid lines). If the current rate results in many losses, the algorithm may fall back to the slower 12\Mbps rate (dashed line).}
\end{figure}

Some algorithms have been proposed to use bit error rate statistics instead of loss rate statistics to adapt rate. SoftRate~\cite{Vutukuru_SoftRate} estimates the bit error rate using a soft-output Viterbi decoder for error correction, and Chen et al.~\cite{Chen_EEC} designed a coding scheme called Error Estimating Coding (EEC) to enable accurate BER estimation at a higher layer.

\subsection{Complication: Multi-dimensional Search Space} 
%The explosion in the sheer number of configurations outlined in the prior section greatly complicates the rate configuration space for 802.11n, however this is only part of the problem.
All of the statistics-based approaches, which walk up or down the list of rates based on whether the current rate works well, implicitly rely on the following basic assumption (outlined by Vutukuru et al.~\cite{Vutukuru_SoftRate}):
\begin{center}
\textbf{Assumption:} \emph{BER is a monotonically increasing function of the bit rate.}
\end{center}
But 802.11n rate configurations are \emph{non-monotonic}. That is, it is not necessarily true that faster configurations are generally less likely to work than slower ones. This violates the axiom of these statistics-based approaches, and hence the \emph{multi-dimensional} search space must be treated as such. I explain why in the following example.

\figref{fig:rate_table_2d} shows three plausible \define{rate maps} for 3-antenna 802.11n links. In these rate maps, each row represents a different number of spatial streams, and each column represents a different MCS. A cell is shaded if a link can reliably deliver packets using that rate at that number of streams. The black box corresponds to \mcs{12}---2 streams at 39\Mbps each---which is the highest working 2-stream rate for these three hypothetical links.

\begin{figure}[t]
      \centering
      \includegraphics[width=\textwidth]{figures/approach/rate_table_2d.pdf}
      \caption[Three different rate maps for 802.11n links]{\label{fig:rate_table_2d}Three different rate maps for 802.11n links. On these links, \mcs{12} (the black box) is the highest reliable 2-stream rate, and gray boxes indicate other reliable transmit configurations. A (\emph{left}): The worst possible situation in which no 3-stream rates and no higher single-stream configurations work. B (\emph{middle}): The best case in which all single stream rates work and all 3-stream rates work up to 39~Mbps each. C (\emph{right}): An average case in which the set of reliable rates decreases as more spatial streams are used.}
\end{figure}
\begin{figure}[t]
      \centering
      \includegraphics[width=\textwidth]{figures/approach/rate_table_1d.pdf}
      \caption[Rate maps for the links in \figref{fig:rate_table_2d} mapped into one dimension]{\label{fig:rate_table_1d}Rate maps for the links in \figref{fig:rate_table_2d} now mapped into one dimension and sorted by total link speed (Mbps).
      %Ties (e.g., 1x39 and 3x13) are broken such that configurations with fewer streams come first.
      In this view, the strict monotonicity assumed by rate adaptation algorithms is violated, and the violations occur in a channel-dependent way. Thus, 802.11n rate adaptation requires optimization along multiple dimensions.}
\end{figure}

\begin{figure}[t]
      \centering
      \includegraphics[width=0.7\textwidth]{figures/approach/rate_maps.pdf}
      \caption[Rate maps for real 802.11n wireless links]{\label{fig:rate_maps}Rate maps for 166 wireless links in my 802.11n testbeds show that with MIMO, the rate space is not monotonic in practice.}
\end{figure}

In 802.11n, each of the scenarios (A), (B), and (C) illustrated in \figref{fig:rate_table_2d} is possible. In particular, if the link can reliably deliver packets using \mcs{12} then it is likely that \mcs{4}---the same encoding, but fewer streams---works as well. The same holds for \mcs{11}, since it uses the same number of streams but a less dense encoding. Similar logic implies all the shaded cells in link (A), which represents the most conservative situation in which \mcs{12} works well. Conversely, link (B) exhibits the best corresponding situation; higher-encoding single-stream configurations may also deliver most packets, and there may be little penalty from using 3 streams, thus resulting in the same set of 3-stream links working. Finally, link (C) exhibits an average case, in which lower rates must be used as the number of spatial streams increases.

The key is that each of these three links is plausible, which means that the search space for rate is non-monotonic in 802.11n. In \figref{fig:rate_table_1d}, I have redrawn the rate maps for these three links along a single-dimension, sorted by data rate. Ties in pure data rate---e.g., \mcs{1} vs \mcs{8}, both at 13\Mbps---are broken such that fewer streams is lower in the search. Here we can see that for all three links there exist higher rates that work well where lower rates do not. I also plot the measured rate maps for 166 wireless testbeds links in \figref{fig:rate_maps}. These results show that with MIMO, 802.11n rates are not monotonic in practice. Thus a rate configuration algorithm for a multi-antenna link needs to consider a multi-dimensional search space.

\subsection{Current 802.11n Statistics-based Approaches}
\begin{figure}[t]
      \centering
      \includegraphics[width=\textwidth]{figures/approach/search_11n.pdf}
      \caption[Rate adaptation search pattern for 802.11n]{\label{fig:search_11n}A typical rate adaptation search pattern for 802.11n. The current rate configuration is \mcs{4}, which uses one 39\Mbps stream. There are three different immediately faster configurations, indicated by the solid lines.
      %: \mcs{5} uses one stream at 52\Mbps, \mcs{11} uses two streams at 26\Mbps each, and \mcs{18} 3 streams at 19.5\Mbps each.
      There are also 5 rate configurations as potential fallback rates, indicated by the dashed lines, that do not offer better bitrates but may be faster if \mcs{4} experiences loss. These configurations offer different tradeoffs between the number of spatial streams and the density of the modulations, and all are valid options for the next choice of rate.}
\end{figure}
\figref{fig:search_11n} illustrates the multi-dimensional search challenge with a concrete example. Each row corresponds to a transmit configuration with one, two, or three spatial streams. The eight boxes correspond to the eight 802.11n MCS combinations, placed in columns that reflect the aggregate link speed.

As we saw in \figref{fig:search_11a}, the single-dimensional search algorithm might try one or two rates higher, and during periods of loss it might fall back to the next lowest rate. In contrast, when increasing 802.11n rate from a single stream at \mcs{4} (39\Mbps), the configurations \mcs{5} (52\Mbps), \mcs{11} (MIMO2-52\Mbps), and \mcs{18} (MIMO3-58.5\Mbps) are all transmit configurations with better link speed, and each might work or not work depending on the channel. When \mcs{4} experiences loss, there are five choices of fallback configuration. This includes the higher-stream \mcs{10} (MIMO2-39\Mbps) and \mcs{17} (MIMO3-39\Mbps) which both have the same link speed and might work, as they use more robust modulation and coding combinations (but require good separation between streams).

MiRA~\cite{Pefkianakis_MiRA} is a recent research algorithm that implements a version of this multi-probing scheme, as does the minstrel\_ht~\cite{minstrel_ht} algorithm used by the Linux kernel to do 802.11n rate selection in practice. A third approach is used by Intel's iwlwifi driver~\cite{iwlwifi}, which uses an 802.11a-like algorithm to select between MCS using a fixed number of streams, and adjusts the number of streams at coarse intervals.

\subsection{State of the Art in Statistics-based Approaches}
The dominant algorithms used in the Linux kernel today are minstrel~\cite{minstrel} (for 802.11a/g) and minstrel\_ht~\cite{minstrel_ht} (for 802.11n).  In general, statistics-based schemes provide good performance for static links in which the devices do not move and the surrounding environment does not change. For such cases, the algorithms may be inefficient at first but will converge to a good operating point. The challenge, as pointed out by several works~\cite{Holland_RBAR,Judd_CHARM,Vutukuru_SoftRate}, is that---depending on the speed at which devices move or the environment changes---these algorithms may be slow to react to varying conditions in mobile links environments, resulting in significant performance degradation in these cases. In \chapref{chap:rate}, I evaluate loss-based rate selection algorithms for 802.11n and confirm that the increased size of the search space and the larger number of rates that must be probed do indeed result in poor performance in fast changing channels.

SoftRate~\cite{Vutukuru_SoftRate} and EEC~\cite{Chen_EEC} are the newest BER-based adaptation algorithms. Both algorithms provide faster adaptation than their loss-based counterparts because by using the BER they can distinguish between a rate that is barely working with marginal BER (in which case the next fastest rate will not work) and a rate that has a lot of headroom (in which case it is worth probing the next fastest rate). These algorithms perform well at shifting up and down within a monotonic rate space of 802.11a/g; however, their BER estimations do not apply across the orthogonal dimensions such as multiple spatial streams of 802.11n. To handle 802.11n, these algorithms would need to be amended to do multi-dimensional search as well.

%%%%%%%%%%%%%%%%%%%%%%%%%%%%%%%%%%%%%%%%%%%%%%%%%%%%%%%%%%%%%%%%%%%%%%%%%%%%%%%%%%%%%%%%%%%%%%%%%%%%%%
\section{Channel-based Approaches}
The second class of approaches to configuring rate use channel information to guide rate selection or adaptation.
As I described in \chapref{chap:background} (\figref{fig:mod_ber_snr}), textbook analyses of modulation schemes give delivery probability for a single signal in terms of the signal-to-noise (SNR) ratio~\cite{Goldsmith}.
These theoretical models hold for narrowband channels with additive white Gaussian noise. They predict a sharp transition region of 1--2\dB over which a link changes from extremely lossy to highly reliable. This feature in theory makes the SNR a valuable indicator of performance.

This gives rise to a simple SNR-based configuration scheme, at least for selecting rate: Upon receiving a packet, a device can use the measured RSSI to compute the \define{Packet SNR} and predict the fastest rate supported. It can then feed this information back to the transmitter, which will use the newly selected rate for subsequent transmissions. This approach was explored in simulation by Holland et al.~\cite{Holland_RBAR} with an algorithm called RBAR, and shown to work well.

\subsection{Complication: Packet Delivery versus SNR in Practice}
Though SNR-based rate control algorithms may work well in simulation, subsequent practical work found that the Packet SNR computed from RSSI was unreliable~\cite{Aguayo_Roofnet, Reis_interference, Zhao_sensys03}. In very early devices, the RSSI was found to vary wildly over time or device temperature, providing unreliable thresholds; this was corrected by calibration in later devices (e.g., confirmed by Zhang et al.~\cite{Zhang_SNRguided} and by my measurements). Reis et al.~\cite{Reis_interference} found that RSSI estimates were corrupted by interfering transmissions. Finally, even in the absence of these latter effects, several studies found that the same RSSI value gives dramatically different performance for different links.

To understand which effects still hold for 802.11n hardware, I generated performance curves using an Intel Wireless Wi-Fi Link 5300 a/g/n wireless network card (I describe my experimental platform in \chapref{chap:tool}). I connected two network cards together via a wire, and configured them to operate in a mode that uses a single antenna to transmit or receive. Using an inline variable attenuator I varied the amount of power received, and for each power level I sent around 1,000 packets using each of the eight 802.11n single-stream rates (\tabref{tab:siso_mcs}) and measured the fraction of delivered packets, the \define{packet reception ratio (PRR)}. With these measurements, I plotted the PRR against the link's SNR (computed from RSSI measurements at the receiver), and present the result in \figref{fig:snr_prr_attenuator}. 

\begin{figure}[t!]
	\centering
%	\includegraphics[width=\textwidth,viewport=13 0 364 204,clip]{figures/esnr/embed_attenuator_snr_prr.pdf}
	\includegraphics[width=\textwidth]{figures/approach/snr_prr_atten.pdf}
	\caption[SNR vs. PRR for a wired 802.11n link]{\label{fig:snr_prr_attenuator}A wired 802.11n link with variable attenuation has a predictable relationship between SNR and packet reception rate (PRR) and clear separation between rates.}
\end{figure}
\begin{figure}[t!]
	\centering
%	\includegraphics[width=\textwidth,viewport=2 0 217 124,clip]{figures/esnr/embed_scatterplot_meas_snr_small.pdf}
	\includegraphics[width=\textwidth]{figures/approach/snr_prr_scatter.pdf}
	\caption[SNR vs. PRR for many wireless 802.11n channels]{\label{fig:snr_prr_26_65} Over real wireless channels in my testbeds, the transition region varies by 10\dB or more. The wireless channel loses the clear separation between rates. Only three rates are shown for legibility.}%
\end{figure}

This figure shows a characteristic sharp transition region between SNR values at which the link goes from lossy to working, 2\dB at low modulations up to 4\dB for the fastest 65\Mbps rate. There is also a clear separation between rates: At a given SNR value, it is clear which rate should be used. This wired link provides a good approximation of a theoretical narrowband channel despite the relatively wide 20\MHz channel, the use of 56 OFDM subcarriers, coding and other bit-level operations. This is the behavior we would want from a link metric in order to predict packet delivery.

In contrast, packet delivery over real wireless channels does not exhibit the same picture. \figref{fig:snr_prr_26_65} shows the measured PRR versus SNR for three sample rates (6.5\Mbps, 26\Mbps, and 65\Mbps) over all wireless links in two wireless testbeds, using the same 802.11n NICs. The SNR of the transition regions can exceed 10\dB, so that some links easily work for a given SNR and others do not. There is no longer clear separation between rates. This is consistent with the measurements from prior work mentioned above~\cite{Aguayo_Roofnet, Judd_CHARM, Reis_interference, Zhang_SNRguided, Zhao_sensys03}.

\subsection{State of the Art in SNR-based Approaches}
Although prior studies and my measurements showed that Packet SNR does not accurately predict delivery across links, they also found that for a particular link a higher SNR generally has higher packet delivery for a given rate~\cite{Aguayo_Roofnet,Judd_CHARM,Zhang_SNRguided}. Consequently, there are two algorithms, SGRA~\cite{Zhang_SNRguided} and CHARM~\cite{Judd_CHARM}, that use SNR feedback from the receiver in conjunction with packet loss statistics in order to learn the relationship between SNR and packet delivery online. Like statistics-based approaches, these algorithms work well for static links. Additionally, they provide good performance for fixed devices in mobile environments~\cite{Judd_CHARM}, because the learned relationship between SNR and PER is only slightly affected by moving objects and the learned calibration is generally valid. However, successive measurements by Vutukuru et al.~\cite{Vutukuru_SoftRate} showed that CHARM tends to under-select in mobile links because it is unable to adapt its thresholds quickly enough to respond to the changing channel.

\subsection{Complication: High-level Measurement of Low-level Subchannel Effects}
I listed above several reasons that Packet SNR calculated from RSSI has historically been a poor predictor of performance. In the modern era of calibrated hardware, however, measurements no longer vary significantly with changing temperature or power level, or across devices. Instead, the dominant factor is likely to be the use of OFDM, and the presence of frequency-selective fading in the RF channel.

\begin{figure}[t]
  \centering
%  \includegraphics[width=\columnwidth,viewport=2 9 185 108,clip]{figures/esnr/embed_fsf-shape-two-links.pdf}
  \includegraphics[width=0.8\textwidth]{figures/approach/fsf_shape.pdf}
  \caption[Channel gains on four links that perform about equally well at 52\Mbps]{Channel gains on four links that perform about equally well at 52\Mbps. The more faded links require larger RSSIs (i.e., more transmit power) to achieve similar PRRs.}
  \label{fig:example_fsf_shape}
  % information for the links used to make above plot: 
  %srcs = [1 10 3 3];
  %dests = [9 11 2 5];
  %txpowers = [-4 20 28 20];

  % reference numbers from expt-8
  %prr = [80 83 78 74];
  %rss = [16.5 30.2 27.1 18.2];
\end{figure}

To illustrate this fact, I chose four representative links in my 802.11n testbed. These four links have SNRs ranging from 16\dB to 30\dB and yet they each perform similarly, delivering around 80\% of packets sent using single-stream \mcs{6} (52\Mbps). \figref{fig:example_fsf_shape} shows the packet reception rates and SNRs for these four links; it also includes the SNRs of the individual OFDM subcarriers for each links.

With this detailed picture, we can see that multipath causes some subcarriers to work markedly better than others although all use the same modulation and coding. These channel details, and not simply the overall signal strength as given by SNR, affect packet delivery. The fading profiles vary significantly across the four links. One distribution is quite flat across the subcarriers, while the other three exhibit frequency-selective fading of varying degrees. Two of the links have two deeply-faded subcarriers that are more than 20\dB down from the peak.

Because of these different fading profiles, these links harness the received power with different efficiencies.
The more faded links are more likely to have errors that must be repaired with coding, and they require extra transmit power to compensate. Thus, while the performance is roughly the same, the most frequency-selective link needs a much higher overall packet SNR~(30.2\dB) than the frequency-flat link~(16.5\dB). This difference of almost 14\dB (more than $20\times$) highlights why Packet SNR based on RSSI does not reliably predict performance.

To exacerbate this issue, 802.11n adds MIMO techniques to the network. During a multi-stream transmission, the receiver still records only one RSSI value per antenna. This RSSI (and the resulting SNR) reflects the total received power combined across all subcarriers and all spatial streams. The total power received will vary with the number of streams, and the actual performance of the link will depend on how well this total power is balanced across spatial streams and how well the receiver can separate the two spatial streams. Thus in 802.11n Packet SNR is likely to be even less accurate when predicting link performance.

\subsection{Approach using Low-level RF Measurements}
\label{sec:accurate}
AccuRate~\cite{Sen_AccuRate} takes an alternative approach to using physical layer information to predict performance. Instead of measuring information about the \emph{signal power}, AccuRate measures the \emph{error vectors} (described in \chapref{chap:background}) of the received symbols when demodulating a packet. To make predictions about bit error rate, AccuRate then replays those same error vectors to a physical layer simulator, which models the reception of a packet using each of the different rates and selects the fastest successfully received packet. Though it would be impractical to implement a full physical layer simulator for each received packet, AccuRate was shown to be significantly more accurate than SNR-based algorithms with performance comparable to SoftRate. At the same time, AccuRate suffers from the same 802.11n-related flaws as the remaining algorithms: The magnitude of the error vectors will change depending on different numbers of spatial streams or channel widths or the use of a short guard interval, and AccuRate can handle none of these cases without implementing a multi-dimensional search.

%%%%%%%%%%%%%%%%%%%%%%%%%%%%%%%%%%%%%%%%%%%%%%%%%%%%%%%%%%%%%%%%%%%%%%%%%%%%%%%%%%%%%%%%%%%%%%%%%%%%%%
\section{Further Wireless Configuration Problems}
\label{sec:problems_11n}
In the previous section, I outlined the rate configuration problem, the current approaches, and the multi-dimensional aspects of OFDM and MIMO technologies that make them less effective. In this section, I briefly mention several other configuration problems and how they are solved in Wi-Fi networks today. (For a deeper discussion of each, see \chapref{chap:related}.) Together, these problems show the richness of the wireless network configuration space. Today, they are handled separately, if at all; in future wireless networks we will want the ability to configure them all concurrently.

\subsection{Antenna Selection}
A transmitter sending fewer spatial streams than it has transmit antennas may wish to use only a subset of the available antennas to save power or pick the best subset to improve performance. Among production 802.11n drivers in Linux, only Intel~\cite{iwlwifi} has such an algorithm; within a fixed number of streams it occasionally probes the different transmit antenna configurations. In this algorithm, switching between transmit antenna sets occurs on timescales of seconds.

In the analogous scenario on the receive side, some chips automatically use only a subset of receive antennas when receiving a packet instead of fully maximizing available spatial diversity. These algorithms typically use Packet SNR measurements of the antennas to determine when additional antennas will add little gain. (Of course, this determination can be erroneous in the face of subchannel fading effects.)

\subsection{Channel Width Selection}
When both devices in an link support multiple channel widths, they may obtain better performance or power savings depending on the bandwidth they choose. Wider channels generally offer higher ideal Shannon Capacities, but since the total power is constrained this is not necessarily true at low SNR. Links may also wish to use smaller channels to avoid in badly faded OFDM subcarriers.

SampleWidth~\cite{Chandra_SampleWidth} uses a probing algorithm to determine which width gives the best performance, aiming to achieve spectral isolation from interferers. For choosing between 20\MHz and 40\MHz bands, the algorithms in Linux tend to integrate channel width selection into the rate selection algorithm. When both widths are available, Intel's driver, which coarsely switches between different numbers of spatial streams, doubles the set of modes it probes by instantiating one copy for each potential bandwidth.

\subsection{Transmit Power Control}
Some devices have the ability to adapt their transmit power levels to save energy or to reduce interference with other nodes. In practice, however, all drivers seem to aim to optimize performance of their link, and simply use the maximum output power. The effects of transmit power control on a link are unpredictable (because RSSI does not capture OFDM fading), so research proposals to achieve control transmit power typically rely on sampling performance and/or interference at various power levels.

\subsection{Access Point Selection}
When clients select which access point to connect to today, a number of factors are important, including both the quality of the link to the access point and also the load on that access point. Clients today typically choose access points solely based on the measured Packet SNR, on the assumption that it is a good proxy for downlink performance. Some proposed enterprise AP systems (e.g., DenseAP~\cite{Murty_DenseAP}) can take load into account as well.

One interesting note is that today's algorithms are not heterogeneity-aware. Consider a choice of two APs: (1) with 30\dB SNR and 3 antennas, and (2) a closer AP with 35\dB SNR but only 2 antennas. Today's clients will choose AP (2) with the larger SNR, even though the first will likely provide better bandwidth depending on subchannel fading effects.

\subsection{Channel Selection}
Channel selection is the problem of choosing the best operating frequency for a pair (or set) of wireless devices. This is not a runtime choice in today's access point networks, because the frequency is chosen by the AP. However, channel selection is likely to be an important problem in device-to-device networks, e.g., Wi-Fi Direct. (I note that algorithms have been proposed for the related problem of distributed access point channel assignment to manage load and interference~(e.g., \cite{Akella_Chan}).)

\subsection{Multi-Hop Routing}
Multi-hop paths will be needed in device-to-device networks, but are not needed in today's AP networks. Work in this space uses a combination of probing, Packet SNR-based heuristics, and state space reduction (e.g., assuming single-antenna devices and a single fixed rate network-wide). One more practical recent proposal by Bahl et al.~\cite{Bahl_repeater} uses relays to address the rate anomaly problem in Wi-Fi access point networks, and uses Packet SNR to predict bitrate and calculate how well paths work. The performance of these solutions depends on the accuracy of these heuristics, which have not yet been adapted for heterogeneous devices or been made MIMO-aware.

\subsection{Spatial Reuse}
Spatial reuse is the problem of managing concurrent transmissions that occupy the same frequency. In today's access point networks, algorithms typically aim to have only one transmission at a time and adaptively turn on RTS/CTS when necessary to eliminate hidden terminals that hurt performance. This effectively avoids spatial reuse entirely.

CMAP~\cite{Vutukuru_CMAP} is a state-of-the-art algorithm to promote spatial reuse that determines whether two links can operate concurrently on the same frequency. However, CMAP only works by fixing the entire network to homogeneous single-antenna devices using the same rate, and even so it requires a complex distributed probing algorithm in a static environment to achieve good performance.

\subsection{Beamforming}
Finally, there is the problem of beamforming, that is, for a transmitter to shape the combined signal sent out its antennas so that the spatial paths best combine at the receiver. This is not yet used in 802.11n, so I discuss the theoretical work and challenges to practical implementations.

Theoretical gains from beamforming are evaluated by measuring the increase in Shannon capacity using ideal hardware and optimal receiver, rather than by the practical constraints of real hardware. I do not know of any practical systems for Wi-Fi that use this type of beamforming, but such a system would need a way to ground the output of the theoretical model to evaluate how the link would work in practice. Rather than an abstract complaint, there are real issues such as the tension between the optimal ``water-filling'' algorithms for beamforming~\cite{Tse} (which allocate power unequally across antennas or subcarriers) and practical hardware constraints such as amplifier peak-to-average-power limits.

In practice, I imagine that this grounding would be done by probing the link performance, as has been the case with analog beamforming strategies~\cite{Liu_DIRC} that operate in a fundamentally different way than 802.11n-like beamforming.

\subsection{Summary}
I have described several configuration problems for wireless networks. In Wi-Fi today, these tasks are complicated to implement and have not been adapted to new technologies like the use of multiple antennas in 802.11n. In practice, these challenges mean that these tasks are often simplified drastically or avoided entirely. In my thesis, I aim to make these configuration problems practical solve.

%%%%%%%%%%%%%%%%%%%%%%%%%%%%%%%%%%%%%%%%%%%%%%%%%%%%%%%%%%%%%%%%%%%%%%%%%%%%%%%%%%%%%%%%%%%%%%%%%%%%%%
\section{My Approach: An Effective SNR-based model for Wi-Fi}
My hypothesis is that \emph{it is possible to rapidly and accurately predict how well different configurations of MIMO and OFDM wireless links will perform in practice, using a small set of wireless channel measurements}. I develop a framework to evaluate how well a particular physical layer configuration works that is flexible enough to handle all the problems discussed in this section. My system uses low-level RF channel measurements in conjunction with a simple but powerful model that can predict the performance of each operating point in a large physical-layer configuration space.

In particular, I develop a practical methodology that uses low-level measurements of the RF channel and the concept of an Effective SNR~\cite{Nanda_EffectiveSNR} to predict performance for wireless channels that use modern physical layer technologies such as OFDM, multiple antennas, variable channel widths, and spatial diversity and multiplexing. I also explain how to apply this prediction engine to a wide variety of link and network configuration problems such as those described in this chapter. To demonstrate that this methodology is practical, I prototype a working system in the context of IEEE 802.11n, which is the dominant consumer wireless networking technology today and includes these state-of-the-art RF techniques. Finally, to show that my model works, I use my prototype to evaluate the accuracy of the choices made using my techniques. I find that my model accurately predicts packet delivery over hundreds of indoor wireless links in two environments, and that this level of accuracy is sufficient to lead to good configurations for many link and network problems. Because channel measurements can be obtained rapidly and Effective SNR uses lightweight computation, 

\subsection{Effective SNR-based Model}
The central component in my thesis is a model for packet delivery that uses low-level RF measurements called CSI (see \secref{sec:csi} below) to predict packet loss over real wireless channels. To be useful, this model must accurately predict the packet delivery probability for a given physical layer configuration operating over a given channel. It must also simple and practical, so that it can be readily deployed, and cover a wide range of physical layer configurations, so that it can be applied in many settings and for many tasks.

In this thesis, I scope my model to devices that use MIMO and OFDM, which captures the fundamental technological primitives for many current and future networks. In particular, the scope of my model is 802.11n including all the enhancements described in \secref{sec:background_80211n}. My model is based on the concept of Effective $E_b/N_0$ developed by Nanda and Rege~\cite{Nanda_EffectiveSNR}, and described as follows.
 
\begin{figure}[t!]
	\centering
	\includegraphics[width=\textwidth]{figures/approach/esnr_intuitive.pdf}
	\caption[Simplified overview of an RF link operating over multiple subchannels]{\label{fig:esnr_intuitive}Simplified overview of an RF link operating over multiple subchannels.}
\end{figure}

\figref{fig:esnr_intuitive} shows a simplified overview of a link operating over an RF channel that has multiple subchannels, such as MIMO spatial paths or OFDM subcarriers. The transmitter applies error correction to the original data packet, and then processes the coded bitstream and maps the resulting symbols onto the multiple subchannels. The receiver processes the noisy signal to recover the (potentially errored) coded bitstream, and then uses error correction to attempt to recover the original data bits.

The key hypothesis introduced by Nanda and Rege is that error correction---in conjunction with mechanisms like frequency- and spatial-aware interleaving in 802.11n---works to spread the errors caused by faded subchannels across the entire channel. If this assumption holds, the link can be modeled as performing with an aggregate error rate equal to the average error rate across subchannels. This average bit error rate is called the \define{Effective BER} of the channel, and from it we can compute the \define{Effective SNR} of the channel. Since the four links displayed in \figref{fig:example_fsf_shape} have similar error performance, they should have similar Effective SNRs. Then the Effective SNR can be used as a metric of link quality, and hence provide accurate estimates of packet delivery.

My model is based on Nanda and Rege's work, which was framed in the context of time-varying channels for CDMA. I extend their work to handle OFDM and MIMO subchannels. I also include practical considerations such as implementation constraints in real hardware and the effects of operating over real channels. I also present the first experimental evaluation of Effective SNR as applied to Wi-Fi.

\subsection{Model Input: Fine-grained RF Measurements}
\label{sec:csi}
As described above, the input to my Effective SNR-based system is a set of low-level RF channel measurements. The particular measurements I use in this thesis are called Channel State Information (CSI). For an OFDM link, the CSI comprises the channel gain coefficient (amplitude and phase,\footnote{Note that in \figref{fig:example_fsf_shape}, I plot only the amplitude as the phase offset does not affect packet delivery for a SISO link (assuming it is properly equalized by the receiver).} see \chapref{chap:background}) for each OFDM subcarrier. For an $N$x$M$ MIMO link, the CSI is an $N$x$M$ matrix where each entry reflects the channel gain coefficient from one transmit antenna to one receive antenna. For a MIMO-OFDM link such as in 802.11n, the CSI comprises a three-dimensional $N$x$M$x$S$ matrix that reflects the $N$x$M$ MIMO link for each of $S$ subcarriers.

A single comprehensive CSI measurement captures the low-level channel details that enable my model to calculate the Effective SNR for a wide configuration space. I next summarize how these measurements are used.

\subsection{Model Output, and how to Apply it}
I describe the model and how it is used in complete detail in the next chapter, but the basic structure of the model is simple: Given (1) a current CSI measurement of the RF channel between transmitter and receiver, and (2) a target physical layer configuration of the transmit and receive NICs, it predicts how well that link will deliver packets in that configuration.

This simple decision primitive integrates easily into higher-layer optimization protocols. These include solutions to all of the problems mentioned in this chapter, such as selecting the best rate, number of spatial streams, or transmit antenna set; whether to use 20\MHz or the entire 40\MHz channel; or choosing the lowest transmit power at which the link supports a particular rate. The key is that my model makes the most complicated step of those protocols---evaluating how well a link will work in a particular configuration---trivial.

%Note that I do not try to make predictions in the transition region during which a link changes from lossy to reliable. Predictions there are likely to be variable, and simply knowing when the link starts to work is useful information in practice. For the model output, I define that the link will work, i.e., will reliably deliver packets, if the model predicts $\geq$90\% packet reception rate. As we will see in this thesis, this system provides good performance across a range of wireless configuration problems.


%steps are (1) comprehensively measuring CSI for the RF channel including all subchannels, (2) adapting that CSI measurement to model the transmit configuration problem under investigation, (3) modeling receiver behavior to come up with the post-processing SNRs of the individual subchannels, and (4) computing the bit error rates of the individual subchannels from their SNRs and then calculating the Effective BER and finally the Effective SNR.

\subsection{Problems Addressed}
\begin{table}[htp]
	\centering
	\begin{tabular}{lc}
	\toprule
		Application of Effective SNR & Described in \\
	\midrule
		Bitrate/MCS selection & \chapref{chap:delivery}, \chapref{chap:rate}\\
		Channel width selection & \chapref{chap:delivery}, \chapref{chap:rate}\\
		Antenna selection & \chapref{chap:delivery}, \chapref{chap:rate}\\
		Transmit power control & \chapref{chap:delivery}\\
		Channel selection & \chapref{chap:applications}\\
		AP selection & \chapref{chap:applications}\\
		Multi-hop path selection & \chapref{chap:applications}\\
		Interference planning/Spatial reuse & \emph{Future work}\\
		Partial packet recovery/FEC & Bhartia et al.~\cite{Bhartia_FreqDiv}\\
		Beamforming & \emph{Future work}\\
		Multicast rate selection & \emph{Future work}\\
	\bottomrule
	\end{tabular}
	\caption[A variety of applications of Effective SNR]{\label{tab:esnr_uses}A variety of applications of Effective SNR. I describe and evaluate how to solve many of these problems with Effective SNR in this thesis, some have been addressed by other researchers using my research platform, and I leave some problems for future work.}
\end{table}

\tabref{tab:esnr_uses} shows a list of several potential applications of Effective SNR. These cover all the problems described above and range from optimizing various parameters of a single Wi-Fi link, such as the MCS or antenna set used, to coordinating many nodes in a dense wireless network. Additionally, I identify applications that can be implemented by looking at other aspects of the Channel State Information in \tabref{tab:csi_uses}. These provide useful primitives that can enable systems to adapt behavior based on the location and movement of the user.

Combined, I believe these applications form the critical building blocks for configuring dense future wireless networks like Wi-Fi Direct. In particular, my Effective SNR model provides the information needed to select rates or configure the network topology, among other things. The CSI can be used to supplement these schemes, particularly by using mobility classification to determine when a device starts to move and trigger reconfiguration of the wireless network in response. I implement and evaluate many of these applications in the rest of this thesis, several have been investigated by other researchers in follow-on work, and some are left for future research.

\begin{table}[tp]
	\centering
	\begin{tabular}{lc}
	\toprule
		Application of CSI & Described in \\
	\midrule
		Mobility classification & \chapref{chap:applications}\\
	    Guard interval selection & \emph{Future work}\\
		Indoor localization & FILA~\cite{Wu_FILA}, PinLoc~\cite{Sen_PinLoc}, SpinLoc~\cite{Sen_SpinLoc} \\ 
	\bottomrule
	\end{tabular}
	\caption[A variety of applications of Channel State Information]{\label{tab:csi_uses}A variety of applications of Channel State Information. I describe and evaluate how to classify mobility in this thesis. Some problems have been addressed by other researchers using my research platform. I leave guard interval selection for future work.}
\end{table}

\section{Summary}
In this chapter I have presented a detailed overview of wireless link and network configuration problems, and the statistics-based and channel-based approaches used today. Generally, packet loss statistics are too specific, applying only to a single or a few configurations. These approaches therefore require packet probes of many different operating points, and are slow to converge in changing channels. Conversely, channel measurements used previously have been too general, not capturing the low-level details of the channel. Thus they do not provide accurate predictions and cannot be used to select operating configurations in practice.

I then described my channel-based approach, which uses low-level channel measurements of the MIMO and OFDM subchannels in conjunction with an Effective SNR-based model to provide a way to predict performance over the broad configuration space. In the next chapter, I flesh out this model and how to use it, and argue that it is indeed flexible and has low overhead. In the remainder of the thesis, I will show that my model makes accurate predictions that lead to good choices of operating points in practice for many of the problems I described in this chapter.

%%%%%%%%%%%%%%%%%%%%%%%%%%%%%%%%%%
\ifx\mainfile\undefined
%
% ==========   Bibliography   ==========
%
%\nocite{*}   % include everything in the uwthesis.bib file
\bibliographystyle{plain}
\bibliography{dhalperi_thesis}

\end{document}
\fi
