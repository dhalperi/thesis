\section{Introduction}
\label{sec:intro}
%%Introduction to the area and motivation for the problem(s) you propose to solve.  Include a proposed principal hypothesis that the thesis will demonstrate, and a synopsis of the likely contributions of the work.

Computer scientists have long envisioned a networked home, in which services and functionality offered by different devices can be easily shared and combined to create a new style of rich applications. Though this vision dates back decades, it has recently been brought closer to reality by the sharp growth in the number and diversity of network-enabled consumer devices. An ABI Research report from late 2010 forecast the shipment of 1~billion IEEE 802.11~\cite{80211} (Wi-Fi) chipsets in 2011, with more than half of these chipsets used in consumer electronics, handsets, and other mobile devices. Wi-Fi's popularity can be attributed to its low cost, small physical footprint, and dramatically increased speeds that reach up to 600\Mbps in 802.11n~\cite{80211n}. The flexibility brought about by eliminating the need for wires makes Wi-Fi a natural choice for home applications.

In this thesis, we aim to understand how to architect a wireless network to support these rich home applications. To begin, we note three important networking trends in this new space to which an architecture must adapt. The first trend is that home wireless networks will become increasingly \emph{dense}. Whereas today the use of Wi-Fi scales in the number of occupants---one access point plus some laptops, tablets, and smartphones---tomorrow, it will scale with the number of objects. Rather than a handful of wireless devices per home, there will be five, ten, or hundreds of devices per room.

Second, consumer devices will be increasingly \emph{heterogeneous} in their wireless capabilities. The number of antennas a device has determines its maximum rate (for 802.11n MIMO), and the available spectrum depends on whether it has a single-band or dual-band radio. These factors are determined independently by manufacturers as a function of size, workload, and profit margin. Small devices like the $\approx$1\insq Eye-Fi SD cards~\cite{eyefi} may not fit multiple antennas; an HDTV has room for many antennas and the demand for high speeds; and printers are sold so cheaply that they may only support 2.4\GHz bands and single antenna systems despite their large form factor. Devices may also take on different roles in the network depending on whether they are fixed or mobile, or using battery or wall power.

Finally, the defining characteristic of this new generation of applications is that any functionality offered by any device, such as input, output, actuation, computation, storage, or data, should be accessible via the network by any other device. Thus, communication patterns are tending towards \emph{device-to-device} and away from today's primarily device-to-Internet workload.\footnote{Though there is also a shift towards the Cloud, applications will still include content generated in the home, such as live video, and can be more interactive using local resources, such as computation~\cite{satya_cloudlets}.} Access to the Internet is no longer the primary purpose of wireless connectivity; instead, the Internet is now only one service of many offered in the home network. 

These three characteristics are fundamentally mismatched with the ``access point'' (AP) model used predominantly in home wireless networks today.
In the access point model, all communication in the network transits through the AP\@, requiring two transmissions per packet for any device-to-device links.
%This can hurt performance as there is no mechanism to short-circuit the AP for device-to-device communication, even though clients can halve transmission count by sending directly, not to mention possibly using faster rates or more streams.
Also, all devices must use the AP's channel and thus concurrent flows all operate in the same contention domain. This leads to performance artifacts such as the well-known ``rate anomaly'' problem if one client has a weak link, or a high likelihood of collisions if there are many active clients.

%When close to each other and far from the AP, devices can use faster rates, and in the case of MIMO, potentially more spatial streams. And by removing the AP as intermediary, devices are no longer constrained to all use the same frequency. A device-device pair can shift to a different channel that might have better (frequency-selective) fading and/or to operate in a different carrier sense contention domain from other ongoing communications.
 
We contend that for the wireless home vision to come to fruition, the underlying network must embrace and take advantage of these trends to become more \emph{efficient}. An efficient network will utilize all available resources, including both the dense set of devices and the available spectrum. This will enable it to concurrently support as many demanding applications as possible.

Researchers have proposed many mechanisms to improve performance that relax the AP model. For instance, a system could use multiple channels to alleviate contention, could use short-circuit the AP or use a different intermediary node to improve throughput, or could operate non-interfering, spatially separated links concurrently. The challenge with these mechanisms is that by adding flexibility to the network, there are many more configurations between which the system must choose. In the AP model, a node must choose when to send a packet, with which antennas, and at what rate. This has already proved a difficult task, and added flexibility makes it worse. As a result, most of this work fixes many of the search dimensions---most commonly by ignoring rate selection---and mechanisms are rarely (if ever) composed. Thus, most of these solutions are not currently suited for practical use.

At the same time, a wireless home network must be \emph{reliable}, as reliability represents both one key component of user satisfaction and one key challenge to adoption~\cite{edwards_challenges}. A network network should be robust even as people and objects move around or leave the home, devices turn off or switch between wall and battery power, and new application workloads add stress to the network. These requirements make it even more important for a flexible system to converge on an efficient operating point---in the complete search space---quickly and accurately.

It is our hypothesis that a wireless network can be built that meets these needs. There are extant techniques in the research literature that can likely be combined to make a flexible, efficient network architecture that will support rich wireless applications. The main challenge

%The earliest ``smart home'' visions covered mainly sensing and actuation---lighting or temperature could automatically be adapted to the location and preferences of users---but a transition to a media-centric 

The principle hypothesis we propose to demonstrate in the thesis is: \emph{It is possible to create a wireless system to support rich home applications that uses practical, distributed mechanisms to efficiently utilize available devices and spectrum resources and maintain reliable communication in the face of mobility and churn.}

The likely contributions of this thesis include:
\begin{itemize}
	\item A tool that provides fine-grained measurements of RF channels in the form of 802.11n Channel State Information (CSI).
	\item An algorithm that uses the CSI to make accurate, real-time predictions about 802.11 packet delivery for wireless links across a wide range of configurations.
	\item Distributed protocols and algorithms for a flexible network that efficiently and rapidly adapts the communication topology and spectral use in response to changing workload, environment, and devices.
	\item The public, open source release of the code and data for tools, algorithms, and experiments used in this work.
\end{itemize}
The remainder of this generals document is organized as follows. In \secref{sec:related}, we begin by discussing prior work related to our proposed hypothesis. Next, \secref{sec:preliminary} introduces our preliminary work on the building blocks for our proposed system. We describe our proposed work in \secref{sec:proposed}, and how we will evaluate our hypothesis in \secref{sec:methodology}. We give a tentative schedule of work in \secref{sec:timeline}. Finally, \secref{sec:conclusion} concludes this thesis proposal and discusses several additional questions that could be answered in future work.

%
%Commodity wireless LAN technology has improved dramatically in the past decade. The low cost, small physical footprint, and minimal power consumption have led to the increasingly pervasive adoption of IEEE~802.11~\cite{80211} (Wi-Fi) in a diverse set of consumer devices. This ubiquity of Wi-Fi, in symbiosis with dramatically increased speeds---from 2\Mbps to 600\Mbps between 1997 (802.11) and 2009 (802.11n~\cite{80211n})---makes it suitable to replace wires in a wide set of applications, and to enable many new ones. This has spawned a number of academic and industrial research and standardization efforts into innovative home applications~\cite{dixon_homeos,microsoft_home,wireless_hd,wifi_direct,xxx}.
%
%A key idea for these visions is that any functionality offered by any device, such as input, output, actuation, computation, storage, or data, should be accessible via the network by any other device. This is not a new idea, indeed it dates back at least to the HomeRF project~\cite{nagus_homerf} that predates Wi-Fi. However, only recently has the market achieved the numbers and diversity of low cost, network-enabled devices necessary to develop rich applications.
%%This idea manifests in applications such as Apple's AirPlay, which enables streaming music from any computer, iPad, iPhone, or iPad to a wireless speaker, 
%
%The hallmark of these new applications is that they involve device-to-device communication within the home;
%%, rather than the device-to-Internet connections typical of the standard 802.11 ``access point'' (AP) model.
%access to the Internet is no longer the primary purpose of wireless connectivity. Instead, the Internet is now only one service of many offered in the home network. Yet, home networks predominantly use the 802.11 infrastructure model of operation that centralizes all communication in the network at the access point (AP). This decision creates a mismatch between the application workloads and the underlying network topology that can limit performance, flexibility, and reliability.
%%Yet, this assumption that Internet access is the main reason for the network is a fundamental part of home network design, and can limit application performance, flexibility, and reliability. I propose to explore how this shift in workload offers an opportunity for revisiting and improving the topology of home networks to better meet the needs of future applications.
%
%In the access point model, all packets in the network are sent via the AP\@. This includes transmissions between wireless clients; 802.11 provides no mechanism to short-circuit the AP when clients are within range. Sending directly, clients can halve transmission count by sending each packet only once. When close to each other and far from the AP, devices can use faster rates, and in the case of MIMO, potentially more spatial streams. And by removing the AP as intermediary, devices are no longer constrained to all use the same frequency. A device-device pair can shift to a different channel that might have better (frequency-selective) fading and/or to operate in a different carrier sense contention domain from other ongoing communications.
%
%%The 802.11 access point mode uses a star topology with the AP at the hub and clients as leaves. In these networks, the AP centralizes control and communication: clients gain admission to the network by associating
%%(and perhaps authenticating)
%%to the AP, and this link forms their only connection to the network. 
%%Packets sent between clients must transit through the client-AP links.
%%Packets sent between clients transit through the AP
%% two distant clients may not be able to communicate, and 802.11 provides no mechanism to short-circuit the AP when they can. 
%%Finally, the access point also buffers packets intended for battery-powered clients that have gone to sleep to save energy.
%
%%This design for a wireless network is simple, but limiting in more diverse workloads. The AP's operating frequency constrains all devices to use a single channel, but .
%
%
%
%%By removing the need for a wired connection, Wi-Fi eases deployment, adds convenience, and enables much smaller devices, such as the $\approx$1\insq Eye-Fi~\cite{eyefi} SD cards that share photos over a network from inside a digital camera.
%%makes Wi-Fi an attractive technology for use in new home applications.
%%has led to a spate of proposed applications that combine multiple of these devices.
%%Examples of these applications include wireless system backup, wireless streaming of high definition video and audio from a Blu-ray player to an HD TV and surround sound speakers, and even pervasive interactive interfaces to the home via wireless cameras and projectors in the Microsoft Home~\cite{microsoft_home}.
%
%%These devices vary widely: they may be mobile or static, large or small, plugged in or battery powered; and the way these characteristics manifest can change over time depending on how devices are used. I summarize some representative commercial Wi-Fi devices in \tabref{tab:wifi-devices}.
%
%%\begin{table*}[ht]
%%\centering
%%\begin{tabulary}{\textwidth}{cCc}
%%\toprule
%%\bf Device & \bf Description & \bf Characteristics \\
%%\midrule
%%Eye-Fi~\cite{eyefi}  & Wi-Fi SD card used to transfer photos from digital camera & Mobile, small, battery power  \\
%%WiDi~\cite{widi} receiver~\cite{widi-rx}  & Laptop screen sharing with HD TV & Static, large, plugged in  \\
%%WiDi~\cite{widi} transmitter  & Built into laptops with Intel Wi-Fi & Mobile$^*$, large, battery power$^*$  \\
%%Printer, scanner, fax & Transfer documents to/from printers & Static, large, plugged in \\
%%AirTunes~\cite{airtunes} & Streaming audio & Static, small, plugged in \\
%%Laptop & Flexible computing device & Mobile, large, battery power \\
%%Tablet/Netbook & Flexible computing device & Mobile, medium, battery power \\
%%Smartphone & Handheld applications & Mobile, small, battery power \\
%%Smartphone & Handheld applications & Mobile, small, battery power \\
%%\bottomrule
%%\end{tabulary}
%%\label{tab:wifi-devices}
%%\caption{Some Wi-Fi enabled consumer devices.}
%%\end{table*}
%
%%This set spans not just mobile personal computers such as laptops, tablets, and smartphones, but also fixed-function devices such as printers, cameras, media storage devices like cards, 
