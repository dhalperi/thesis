\section{Proposed work}
\label{sec:proposed}
%\emph{\color{red}Proposed work section that describes the main challenges and innovations that are needed to accomplish the proposed hypothesis, with as much description as possible of how you will tackle each one.}

%In \secref{sec:preliminary}, we described our tool to measure the 802.11n channel state information (CSI), and how we have used the CSI to compute effective SNR and predict packet delivery for wireless links. In this section, we describe an architecture for a 

In this section, we describe our proposed architecture for home wireless networks, and outline the mechanisms we believe will enable it to efficiently and reliably support rich, demanding applications. We begin by presenting our proposed design.

%\subsection{Design Goals}
%When considering a next-generation architecture for home wireless networks, we first examine the benefits of the access point model used today. The major advantage of an access point network is its simplicity: by adopting the Ethernet switch and/or IP router models and nearly completely ignoring the broadcast nature of wireless networks, individual nodes make few decisions. As the network is a fully connected tree, routing is trivial. Because the network is at most a few hops wide, simple CSMA/CA works in the common case to manage access to the medium, and nodes can enable simple mechanisms such as RTS/CTS can be enabled when they detects a hidden terminal~\cite{wong_rraa} if one exists.
%
%Here are our design goals for a more flexible wireless home network.
%\begin{enumerate}[leftmargin=0em,itemindent=2em,itemsep=0pt,topsep=3pt,labelsep=0.5em]
%	\item First, the network should be \emph{fully connected}. Consider a strawman solution in which each application starts a different network on a different channel, and nodes cannot participate in multiple networks simultaneously. This trivially achieves spectral efficiency, inherits the simplicity and reliability of the AP model, and promotes application isolation. However,  it limits nodes to using a single application at a time, requires that applications be enumerated ahead of time,
%\end{enumerate}

\subsection{Wireless Network Architecture}
When considering a new architecture for home wireless networks, we first note the benefits of the access point model used today. The major advantage of an access point network is its simplicity: by adopting the Ethernet switch and/or IP router models and nearly completely ignoring the broadcast nature of wireless networks, individual nodes make few decisions. As the network is a fully connected tree, routing is trivial. Routes are learned in the same manner as wired Ethernets, by inspecting the source addresses of received packets. Because the network is at most a few hops wide, simple CSMA/CA works in the common case to manage access to the medium, and nodes can enable simple mechanisms such as RTS/CTS can be enabled when they detects a hidden terminal~\cite{wong_rraa} if one exists.

We would like our network to replicate many of these benefits, with added flexibility. Thus, we explain our design by mimicking an AP network, and then relax the design to be more flexible.

\subsubsection{Assumptions}
We make a few simplifying assumptions about the devices in the network. First, we assume there is a single specialized device, which we call the \emph{coordinator}, that takes in many of the roles that the access point does in Wi-Fi networks. In particular, the coordinator is the root of the communication topology, is always on, and operates on a fixed channel (likely, but not necessarily, in the 2.4\GHz band). We assume that the coordinator is a relatively powerful device: it has many antennas and supports fast rates, and is physically centrally located in the home. We also assume that every device in the network can associate to the coordinator at some minimal rate; given the size and scale of most homes, and the range-extending techniques enabled by multiple antennas in 802.11n, this is a (mostly) reasonable assumption.

The coordinator need not take on all the roles of the AP, however. In particular, the coordinator need not offer egress from the wireless LAN, or host a DHCP server. However, we assume that there is a device on the network that provides each of these services, if necessary.

We also assume that all devices on the network support the mechanisms we describe, and defer handling of legacy devices and backwards compatibility for later.

\subsubsection{Clients and Repeaters}
\label{sec:client_repeater}
In addition to the coordinator, there are two more roles that devices will play in the network, also inspired by roles of devices in AP networks. In our network architecture, all non-coordinator nodes are clients. Clients act as they do in 802.11, associating with the coordinator and perhaps using power-saving techniques when inactive and/or on battery power.
%As in MultiNet and Wi-Fi Direct, clients in our network architecture may associate with multiple nodes and operate on multiple channels when beneficial to do so. We will describe these mechanisms below.

Some clients in the network may additionally act as \emph{repeaters} and relay packets between other devices in the network. Unlike in standard access point networks, any client may choose to be a repeater, not just a special-purpose device purchased to extend the network. The decision to become a repeater is a local choice, but we envision that repeaters will typically be capable nodes with multiple antennas that are also connected to a large power source. The former constraint implies that their repeater functionality is likely to improve rates for some clients, and the latter that the repeater can tolerate the added energy consumption that comes from relaying transmissions for others. As in AP networks, repeaters broadcast beacons advertising their presence and respond to probe requests.

\subsubsection{Adding Flexibility via Relaxed Association}
\label{sec:flexible}
The notion of association is crucial to the functioning of AP networks. Indeed, we observe here that we can recreate all the above constraints of an AP network simply by scrutinizing the notion of association between two wireless devices. When one device associates to another, this creates a \emph{parent}-\emph{child} relationship between them, and implicitly defines the tree along which traffic is sent. Second, the notion of association constrains the client to the multicast/broadcast transmission schedule of the parent. Third, association defines the operating channel of the client, which adopts the channel that its parent uses.

\heading{Flexible Topology.}
To add flexibility to the network's topology, we relax the definition of association.
%In particular, we separate the notion of a node's \emph{parent} from the its primary \emph{operating channel}.
First, we enable a client to be associated to multiple devices in the same network, but mandate exactly one of these is the clients's parent. We call the non-parent devices to which a client is associated its \emph{aunts}, and the client the \emph{nephew}. When a node associates to a new aunt, this creates a new link between these two devices and short-circuits the higher parts of the tree. In the routing table, each endpoint adds a single static route across the link for the other; traffic to other destinations on the network is unaffected by the new link. In particular, default routes in the network still follow the tree generated by parent-child links, and this new link can create no loops in the network.

Note that one implication of adding only a single static route, in combination with Ethernet's destination-based routing, is that children of the nephew do not use the nephew-aunt link when talking to children of the aunt. However, because all nodes begin within range of the coordinator, network topologies are likely to be small, in practice probably limited to depth at most 2 below the root. We believe that this inefficiency is limited to a few unlikely scenarios, and more than justified by the routing simplicity it provides.

In contrast with the approach described here, related works such as MultiNet~\cite{chandra_multinet}, FatVAP~\cite{kandula_fatvap}, and Wi-Fi Direct~\cite{wifi_direct} all only handle the case where the networks are logically separate.

\heading{Multi-channel Operation.}
Having described how we can support a flexible topology and still route packets, we now discuss how multi-channel operation works. In our network, the coordinator and all repeaters have a primary channel, which need not be the same as that of its parent. For instance, a repeater may be participating in a nephew-aunt conversion with high traffic demand on a different channel than the coordinator, but still route packets for other endpoints through the coordinator. A repeater may only send beacons or respond to probe requests on its primary channel. Clients need not have a primary channel, and instead can choose to communicate with whichever parent or aunt they choose.

Note that scheduling is more difficult in a multi-channel network. When the aunt's primary channel is different from its parent's, or its nephew's parent's, their communication schedule needs to take into account the times that one node or the either must be on a different channel. Repeater beacons must include not just scheduling information to tell children when it transmits buffered multicast/broadcast packets, but also the schedule of the repeater's absence from its primary channel. In some cases, there may be little overlapping time in the schedule, limiting the gains from this particular nephew-aunt link. 

Note that in this network architecture, all children of any repeater whose primary channel is different than that of its parent must buffer packets for the repeater during the its absence. In contrast, in the access point model in which APs and repeaters all operate on the same channel, parents are always listening and clients can push traffic to their parent whenever the CSMA/CA algorithm allows. This also implies that extra latency due to these scheduling delays may be incurred compared to access point networks. These are the tradeoffs for the increased flexibility and capacity that multi-channel operation allows.

%In today's access point networks, there are two types of nodes in the network: the \emph{core}, and the \emph{clients}. The core consists of the access point and repeaters. The devices in the core operate on the same channel and are always on. They are responsible for informing clients of the network's presence by sending beacons and responding to probe requests, forwarding and/or buffering traffic destined for associated clients, and communicating the schedule of the network via timing information (about beacon intervals) which they share in beacon packets. The access point serves as the the root of the core, and repeaters associate to it or to each other. Clients associate to a single core node, which they send to and receive from all their traffic. Clients are responsible for following the schedule dictated by the beacons, in particular, they must wake up at the right time in order to hear any multicast traffic that is destined for them.

%\heading{Flexibility via Relaxed Association.} The notion of association is crucial to the functioning of AP networks. Indeed, we observe here that we can recreate all the above constraints of an AP network simply by scrutinizing the notion of association between two wireless devices. When one device associates to another, this creates a parent-child relationship between them, and implicitly defines the tree along which traffic is sent. Second, the notion of association constrains the client to the multicast/broadcast transmission schedule of the parent. Third, association defines the operating channel of the client, which adopts the channel that its parent uses.

%To add flexibility to the network, we relax the definition of association. In particular, we separate the notion of a node's \emph{parent} from the its primary \emph{operating channel}. First, we enable a node to be associated to multiple core devices, but exactly one of these is the node's parent. We call the non-parent devices to which a node is associated its \emph{aunt}, and the node the \emph{nephew}. When a node associates to a new aunt, this creates a new link between these two devices and short-circuits the higher parts of the tree. In the routing table, each endpoint adds a single static route across the link for the other; traffic to other destinations on the network is unaffected by the new link.

\heading{Changing Parents.}
In some cases, a node may wish to switch parents. This can happen due to mobility, if the node moves away from its parent; when a better repeater is activated, e.g., by a laptop being plugged into a docking station and enabling repeater mode; or due to topology churn such as during a node failure, environmental changes, shifting workloads, or the laptop described above being unplugged and disabling its repeater functionality. When a node changes parents, this is handled in the same manner as Wi-Fi roaming in today's networks; after associating with the new parent, the node sends a broadcast packet such as a gratuitous ARP to the multicast tree in order to force nodes to update their routes. Some packets destined for the node may be lost during the transition, as well as multicast/broadcast packets that the new parent sent prior to association, but these same problems afflict Wi-Fi roaming and are easily solved by higher-layer algorithms.

%\heading{Federated Tree Structure.}
%Keep one device as coordinator, on 2.4\GHz band. Centralized place to keep logs. Other powered, many-antenna devices can act as localized repeaters similar to ``range extenders'' in today's networks. Coordinators take on many functionalities of APs today: broadcast/multicast traffic, and buffering, for clients. Disadvantage: multiple broadcasts. Advantage: can do multicast rate selection. Another advantage: improved power consumption for clients that use a nearer repeater with better rates, also fewer other clients to listen to. Both possible in today's networks but require user to purchase dedicated hardware; here any device can choose to be a coordinator and provide these benefits.

%Believe that this will typically have at most 2 levels of coordinators.

%Generalized model of that provided in Wi-Fi Direct Spec, which has some support in current hardware; key is that as in 802.11 standard there is no guide as to how to make any decisions.

%\heading{State.}
%As in AP networks, coordinators share local state via beacons/heartbeats. This includes setting up a periodic time (and channel) to meet to share multicast/broadcast packets (typically every 200--400\ms in Wi-Fi today).

\subsubsection{Challenges}
This completes our description of the proposed wireless network architecture. However, note that while the basic mechanisms are now defined, algorithms to make decisions about topology or frequency use are still undefined. How should a node joining the network select a parent from the available coordinator and repeaters? How should two distant nodes that wish to communicate select a good repeater to relay transmissions between them? How should a potential nephew and aunt connection choose the best operating frequency for the link? We address these questions in the next subsection.

\subsection{Improved Association}
In a dense network, a new client may need to select its parent from many available repeaters in addition to the network coordinator. In AP and Wireless Distribution System~(\secref{sec:wds}) networks today, clients typically choose the node whose probe response has the highest SNR\@. In dual-band networks, some devices may prefer a 5\GHz AP with slightly lower SNR, as long as it exceeds a minimum threshold, based on the optimistic assumption that interference is lower in the 5\GHz band. Here, we describe a procedure that uses the Effective SNR to improve this decision process and select a good parent.

Normally, a client scanning for a network cycles through the available channels, sends a probe request at the lowest rate (including a single stream and 20\MHz channels), and all APs or repeaters in range respond. We propose that the client instead send multiple probes that use the lowest 6.5\Mbps rate, but vary the number of antennas and channel width in decreasing order. In this way, the coordinator and all repeaters that measure CSI from the probes can compute the Effective SNR for the link. The probe responses can now include the computed Effective SNR to better inform the client's choice. The response can also include the predicted uplink rate, and repeaters can also include their uplink and downlink\footnote{In 802.11n, uplinks and downlinks can be truly asymmetric, e.g., if one node has many more antennas than the other. Also, some devices such as the \term{iwl5100} can only send a single stream but can receive two.} Estimated Transmission Times~\cite{draves_ett} to the coordinator so that the client can pick the best \emph{relay}, not just the best \emph{node}.

Finally, if the client includes its transmit power level in the probe request (or if the responder makes a conservative estimate), then the responder can combine this information with the CSI measured from the probe to compute the Effective SNR for the downlink. It can then send the probe response at a faster rate than the base rate and reduce the overhead of the probe response.

\subsection{Finding Alternate Paths}
When initiating a connection for a high-bandwidth workload, or after detecting via queueing that the application wants to send more traffic to a particular destination than the current route supports~\cite{bahl_repeater}, a node can initiate a search for an alternate path.

\heading{Short-Circuiting Paths.} The first way to do this might be to try and short-circuit the existing route. The sender can send a probe packet along the path to the destination.\footnote{This is the egress of the network if the destination is outside the LAN.} As the probe propagates along the path, each intermediate node can append their MAC address, primary operating channel, and uplink and downlink rates; in other words, the probe collects the salient entries for the path vector. After gathering this information, either source or destination can probe nodes along the path for CSI, use Effective SNR to compute the uplink and downlink rates, and initiate a nephew-aunt connection when beneficial to improve performance by short-circuiting the path. We implicitly include the establishment of a direct link between the nodes in the process.

\heading{Finding a New Intermediary.} When the short-circuiting approach fails or in insufficient, the endpoints of a link can also agree to look for a new intermediary. In this case, each node sends a broadcast request to gather a list of repeaters and their schedules and primary operating channels. They can then probe those repeaters with compatible schedules, gather CSI, compute Effective SNR, and choose a suitable relay if one exists. If so, the clients can each establish a nephew-aunt connection with the repeater, each also add a second new static route to send packets destined for the other device through the relay. For fault tolerance, if the relay connection becomes unreliable, packets can always fall back to the default route through the tree.

\heading{Alleviating Rate Anomaly.} In the same vein, we can leverage all of the mechanisms from the opportunistic repeater work of Bahl, et al.~\cite{bahl_repeater} to automatically recognize rate anomaly when multiple nodes on the same channel communicate with the same destination. As in their system, negatively impacted nodes can volunteer to serve as repeaters, when doing so will improve their own performance. That work used RSSI to predict the performance of the weaker link if it used the repeater; Effective SNR can likely make this computation significantly more accurate. The weaker node could also perform multicast rate selection to maximize the combined performance in the case that either the repeater or the destination receive packet. Also, Bahl, et al.\ mentioned that the repeater could shift the workload to a different channel, but did not address the work any further. As we shall see in the next subsection, Effective SNR can help here as well.

\subsection{Channel Selection}
In a variety of scenarios, two nodes may wish to find a different operating channel. For example, they might aim to find one that permits a faster link, or to avoid interference with an ongoing heavy workload. The standard approach in this case would be to synchronously hop through channels, using SNR as an indicator of link quality. To be more accurate, the search could sample some rates, but this takes more time. Obviously, the Effective SNR should be helpful in this case to enable measurements that are just as quick to obtain as RSSI and as accurate as sampling.

However, we have two additional insights to help improve this search. First, we measured most links in our testbed to have to have approximately the same RSSI across channels within each frequency band. This is attributable to the diversity gain from multiple receive antennas in 802.11n, which reduces the likelihood of a deep fade on a particular channel. The major difference between channels is then attributable to the individual details of the subcarriers themselves. As discussed in \secref{sec:preliminary}, the more faded the subcarriers of a link, the worse its performance relative to what might be indicated by its RSSI-based packet SNR\@. We believe it likely that we can look at the distribution of subcarrier fades computed from the CSI, for instance by comparing the 25th percentile subcarrier SNR and comparing this to the packet SNR, to determine whether the current channel is likely to be better or worse than another channel within the same band. Indeed, we use an approach along these lines instead of Effective SNR in an initial version of our preliminary work.

The second insight is that the same NICs operating on the 5\GHz band consistently exhibit about 14\dB weaker SNR than on the 2.4\GHz band. This is attributable to the inefficiencies of dual-band antennas (that often are about 3\dBi worse in 5\GHz, per endpoint) and a 7\dB--8\dB loss because of the center frequency difference between the two bands. We also measured this difference to be consistent across links in our testbed measurements. Links can leverage this knowledge to choose which band to operate in; strong links in particular, which would like work well in 5\GHz, should abdicate the valuable 2.4\GHz spectrum and leave it for more distant endpoints.

\subsection{Mobility Detection}
To conclude our discussion of proposed mechanisms, we consider the problem of detecting when a wireless device is mobile. Ravindranath et al.~\cite{ravindranath_sensorhints} shows that this can be valuable both for adapting the behavior of the underlying rate control algorithm, and for triggering the pre-emptive computation of alternate paths or relays. Having this same functionality for mobile devices in our network could greatly improve reliability by reducing outages while nodes are mobile. In their work, Ravindranath, et al.\ used sensor hints, such as accelerometers in mobile phones to detect mobility. They found that RSSI changes too slowly to be useful, and also returns false positives when there is movement through the environment instead. We believe that the using the CSI instead of the RSSI will overcome these deficiencies. When a device starts moving, fading on subcarriers and spatial paths changes on the order of wavelengths, up to 12\cm for Wi-Fi, so CSI can likely detect mobility quickly. Secondly, we believe that temporal correlation on the CSI will reveal when the device itself is moving (little to no long-term correlation) or the environment is changing instead (sporadically strong correlations). Finally, we propose the use of a new ``sensor hint'' in addition to those used in the prior work: whether the Wi-Fi device is plugged into an external power source or not.

\heading{Summary.} In this section, we have outlined an architecture for a flexible wireless network and proposed four different algorithms that leverage the 802.11n Channel State Information and the Effective SNR to aid basic decisions and detection. While there are many other applications of Effective SNR that we could explore, these four seem the most immediately valuable mechanisms in the home application scenario. In the next section, we will discuss how we will evaluate our proposed work.

\if 0
\heading{Mechanisms:}
\begin{itemize}
\item capacity and isolation: operation on multiple channels via, e.g., psm
\item flexible topology: multihop communication in wireless link
\item responsiveness: mobility detection with CSI
\item discovery: capability announcement/broadcast
\end{itemize}

\heading{Capabilities:}
\begin{itemize}
\item Admission to Wi-Fi network
\item Relay service
\item Packet Buffer
\item Fixed (e.g., same channel, always on)
\item Power state\\
\hrule
\item DNS/DHCP
\item Access to Internet
\item Media (Video/Audio/Pictures/etc. Source/Sink)
\item Input device (M, KB, Touch, Sensors)
\item Intermediary (e.g., transcoder)
\end{itemize}

\heading{Algorithms:}
\begin{itemize}
\item allocation of devices/links to channels, esp. between bands
\item decision to switch relays/etc.
\item link rate/latency/etc. estimation: what/how to feed up to applications
\item assignment of apps to protocols
\end{itemize}

\heading{Apps:}
\begin{itemize}
\item Use phone as interface for instant replay streaming from DVR
\end{itemize}
\fi