\ifx\mainfile\undefined
\input{chapter_head}
\setcounter{chapter}{9} % Set to n-1!
\fi
%%%%%%%%%%%%%%%%%%%%%%%%%%%%%%%%%%

\cleardoublepage
\chapter{Conclusions and Future Work}
\label{chap:conclusion}

Modern Wi-Fi (IEEE 802.11n) devices can provide flexible, portable, high-performance connectivity at low cost. This unprecedented functionality is poised to enable a new class of rich applications built by combining functionality from many devices. The key missing component is a network connectivity layer that ``just works'', providing good performance overall and quickly adapting to changing application demands and mobile wireless environments.

There are two components to such a network layer. The first is a protocol to connect the devices logically. For 802.11n, this is readily available in the form of Wi-Fi Direct~\cite{wifi_direct}, a recently standardized specification for building wireless peer-to-peer networks targeted at these applications. Support for Wi-Fi Direct is actively being developed in major consumer operating systems including Linux, Mac OS X, Windows, and Android.

The focus of my thesis is on the second key component: A way to configure the physical layer parameters and network topology to best meet application needs. The nature of modern wireless technology and heterogeneity of wireless devices combined with the inherent multi-device coordination makes this a hard problem. There is a large number of possible configurations from which the network must choose a good operating point.

In this thesis, I have demonstrated that my Effective SNR model provides a practical mechanism to quickly evaluate how well configurations work, which can be used to efficiently configure the physical layer. I have also shown that it can flexibly handle a wide variety of applications and parameters that no prior approaches considered in tandem.

In this chapter I summarize my thesis and its contributions and present next steps for this work.

\section{Thesis and Contributions}
My thesis is that \emph{it is possible to rapidly and accurately predict how well different configurations of MIMO and OFDM wireless links will perform in practice, using a small set of wireless channel measurements}. I demonstrated this thesis by building an Effective SNR-based model for wireless networks and evaluating it in the context of IEEE 802.11n.

My Effective SNR model can evaluate a particular physical-layer configuration using a simple interface. The model takes as input a single MIMO and OFDM channel measurement, a target set of transmitter and receiver device configurations, and produces an estimate of how well that combination will deliver packets. This flexible API can express a wide class of configuration tasks, as illustrated by the algorithms I presented in \chapref{chap:delivery}, \chapref{chap:rate}, and \chapref{chap:applications}. In particular, I applied my Effective SNR model to rate selection, antenna selection, joint transmit power and rate control, access point selection, channel selection, and relay selection. I also showed how to use CSI to determine whether a wireless device is mobile.

I demonstrated that this model is practical; I presented data on its low computational overhead and implemented it on commodity wireless hardware. My model uses measurements already taken by devices in order to receive packets, and can compute its output in much less time (4\us) than it takes to transmit a packet (at least 48\us). In most cases, only a few bytes that represent application decisions need to be exchanged, such as a receiver feeding back a particular requested rate to a transmitter. My detailed prototype evaluation of the model in a wide variety of applications provides experimental proof that the model is practical and accurate for real devices operating in real wireless channels.% I used this prototype to apply model to a wide range of applications, and showed that it has good performance that extends to large search spaces and fast mobile channels.

\noheading My specific contributions are as follows:

\begin{itemize}[leftmargin=0.5cm,parsep=1ex,itemsep=1ex,topsep=1ex]
\item I developed a model that accurately predicts the error performance of different MIMO and OFDM configurations on wireless channels. This model is flexible to support a wide variety of transmitter and receiver device capabilities, device implementations, and applications. I presented an implementation of my model using a commodity 802.11n wireless device that demonstrates its feasibility in practice and handles the practical considerations of operation over real links using real, non-ideal hardware. This includes a detailed experimental evaluation of my system that shows that this model accurately predicts packet delivery over real 802.11n wireless links in practice.

\item I detailed how to use this model in a system that can solve a large number and variety of configuration problems similar to those described in \secref{sec:intro_problem}. I evaluated this system in the context of a wide variety of 802.11n configuration problems.
%In these evaluation, I quantify the performance gains when using my Effective SNR metric over versions that use the RSSI-based SNR channel measurements available today.
These evaluations show that the predictions output by my model lead to good application performance in practice, and exceed the performance of prior probe-based and RSSI-based approaches.

\item As part of my thesis I have produced an 802.11n research platform based on open-source Linux kernel drivers, open-source application code, and commodity Intel 802.11n devices using closed-source firmware that I customized.
%that uses commodity 802.11n wireless devices to measure the 802.11n CSI for the wireless channel, and use this tool to apply my model to real measured 802.11n channels.
I have released this tool publicly, and at the time of writing it is in use at 23 universities, research labs, and corporations.
\end{itemize}

\section{Future Work}
In this section, I consider paths for future research. First, I think Effective SNR provides a path to building a high-performance networking layer that works well, and would like to pursue this. Second, I present three further configuration problems that I believe Effective SNR can help solve, but which I have not yet validated.

\subsection{Using Effective SNR in Practical Systems and Protocols}
My thesis suggests that it will be possible to build a high-performance networking layer that ``just works''. While my practical Effective SNR model is a step in that direction, much work remains to be done. I proposed several different ways that my model could be used, e.g., via computation at the receiver side or transmitter side of a link, but there is still a large step to reducing it to practice.

One place to start is by integrating my model into control algorithms for Wi-Fi Direct networks. Open source Wi-Fi Direct implementations are close to being refined enough to support experiments. Building a working combined system would provide invaluable practical experience and research lessons. One key problem includes identifying the cases where Effective SNR results in poor choices (e.g., the few cases where Effective SNR chooses poorly performing access points) and working around them to provide good performance in practice. I discuss a few more important tasks below.

\subsection{Spatial Reuse}
Coping with interference is an important problem in wireless networks, and one that I have not yet solved. An important problem for further research is understanding and managing spatial reuse, that is, understanding when multiple transmitters can send concurrently on the same spectrum. This problem is the subject of great importance in today's AP wireless networks, as systems become increasingly dense. This problem will likely be ameliorated to some extent in future wireless networks that can take advantage of multiple channels, but interference will always be a primary limiting factor in scaling networks.

Existing work has shown large gains from spatial reuse, especially in the area of eliminating corner cases of hidden terminals that can degrade link performance to almost nothing. Today's solutions use expensive distributed coordination mechanisms (i.e., RTS/CTS~\cite{Karn_MACA}) with large overheads that are often disabled in practice, and today's research proposals on spatial reuse for Wi-Fi~\cite{Shrivastava_CENTAUR,Vutukuru_CMAP} have simply fixed the entire network to a single rate during experiments because of the large search space.

Work in communications theory has defined an Effective SINR notion that extends the Effective SNR to take interference into account. These models typically assume that interference, like noise, is modeled by the AWGN model. However, I believe that it will not be this simple in practice. One reason is that interference is not really equivalent to noise, but is frequency- and spatially-selective as well. Second, how to extract good performance from real Wi-Fi devices in persistent interference is not well-understood. Extending my model to support a practical version of this notion would likely be useful.

\subsection{Saving Energy with Effective SNR}
Effective SNR could be highly integrated into the development of better methods to manage the power consumption of battery-operated devices. In particular, clients could select access points or relays with the express aim of minimizing wake time. By choosing a close relay that uses fast rates, a client can spend less time awake. By disabling receive antennas on the mobile device and using advanced mechanisms such as beamforming on the transmitter, the client can make further power savings. I highlighted the importance of these 802.11n parameters in an earlier measurement study~\cite{Halperin_Power}, but have not shown how to optimize them. I believe that my Effective SNR model can be used in conjunction with power-aware metric functions such as that described in \secref{sec:metric_functions}, but this solution has not yet been shown to work well in practice.

\subsection{Practical Benefits of Beamforming}
Transmit beamforming is a well-studied area of research in communications theory. Most theoretical systems aim to optimize Shannon capacity assuming ideal hardware. In this case, they use the well known water-filling algorithm~\cite[p. 183]{Tse} to allocate power across subchannels proportional to SNR. However, this approach may be ineffective in practice because real hardware does not support these idealized requirements.

Real transmit hardware can only support signals with a particular dynamic range, and so cannot perfectly support water-filling. Secondly, in practical systems like 802.11, different subchannels are modulated identically and thus cannot make good use of the asymmetric power across subchannels. Work is needed to understand the practical constraints of real hardware, and then to design a beamforming algorithm for 802.11n that limits the space of beamforming matrices to account for these factors. I believe this problem is interesting in its own right, and that my Effective SNR model can be used to evaluate allocations of power with the goal of minimizing bit errors and finding the best working modulation and coding scheme across all subchannels.


\section{Summary}
Wireless systems such as 802.11 are configured today using probing for rate adaptation and for a host of other applications. This probing is the standard strategy because communications-theoretic approaches to configuring network parameters are considered too inaccurate to work well. However, in my thesis I have shown that it is indeed possible to connect theory back to real wireless systems operating over real wireless channels. I have presented an Effective SNR model for wireless systems that use modern physical layer techniques like MIMO and OFDM, and I have shown that it works well for IEEE 802.11n. Going forward, I hope and expect that Effective SNR will be integrated into the control plane for future wireless networks and help enable the next generation of device-to-device wireless applications.

%%%%%%%%%%%%%%%%%%%%%%%%%%%%%%%%%%
\ifx\mainfile\undefined
%
% ==========   Bibliography   ==========
%
%\nocite{*}   % include everything in the uwthesis.bib file
\bibliographystyle{plain}
\bibliography{dhalperi_thesis}

\end{document}
\fi

