\ifx\mainfile\undefined
\input{chapter_head}
\setcounter{chapter}{8} % Set to n-1!
\fi
%%%%%%%%%%%%%%%%%%%%%%%%%%%%%%%%%%

\cleardoublepage
\chapter{Related Work}
\label{chap:related}

In this chapter, I place my research in the context of existing work on wireless technology and wireless systems.

\section{Understanding real 802.11 wireless channels}
A number of studies investigate the performance characteristics of 802.11. Initial studies of 802.11b~\cite{Aguayo_Roofnet,Reis_interference} found RSSI to be a weak predictor of packet delivery that improved when receivers were calibrated for thermal conditions and manufacturing variability. Today's NICs have intense calibration procedures that mostly eliminate these issues. However, the variation across links with 802.11a/g/n OFDM comes from frequency-selective fading~\cite{Lampe_adaptation, Tse}, which does not affect spread-spectrum modulations in 802.11b.

\section{Theoretical analysis}
Much theoretical work on OFDM with convolutional coding starts with Effective BER or SNR~\cite{Nanda_EffectiveSNR} and adds simulated faded channels to build closed-form expressions for error rates under coding~\cite{Tobagi_ofdm,Nortel_3g,Tralli_convolutional}. Effective SNR has also been extended to MIMO-OFDM~\cite{Liu_EESM,Martorell_11n}. Our model is related, but simpler: we eschew simulating complex, implementation-dependent coding effects in favor of using fixed, per-rate thresholds. We convert CSI to Effective SNR in a way that better matches the equal modulation and power allocation used by 802.11n and offer a better API for practical use. Most importantly, we experimentally evaluate our model for 802.11 NICs and real RF channels; we are not aware of other work on 802.11 that uses Effective SNR measures outside of simulation or analysis.

\section{Rate adaptation}
Many rate adaptation algorithms have been proposed that use packet delivery statistics~\cite{Bicket_SampleRate,Wong_RRAA}, RSSI-based packet SNR~\cite{Camp_rateadapt,Judd_CHARM}, or symbol-level details of packet reception~\cite{Sen_AccuRate,Vutukuru_SoftRate} to adapt to varying channel conditions. Some proposals require custom hardware~\cite{Camp_rateadapt} and may drastically change the fundamentals of the communication~\cite{Rahul_FARA}. These methods do not extend to 802.11n and do not address related factors, e.g., transmit power.

Compared to SoftRate's~\cite{Vutukuru_SoftRate} use of BER estimates, our Effective SNR metric is more general. With a single CSI measurement, we can extrapolate performance in a wide space of rates, spatial streams, antenna selections, channel widths, and transmit power levels. We have also shown that Effective SNR can be implemented on commodity NICs and evaluated it over real wireless channels with mobile and fixed clients. Like RRAA~\cite{Wong_RRAA} and SoftRate, Effective SNR helps to distinguish collisions from channel induced packet loss; with accurate predictions of interference-free packet delivery there is no need to adapt rate in response to loss.

Finally, Effective SNR could inform and improve schemes that combine transmission with more efficient channel-dependent coding~\cite{Lin_ZipTX} or partially-correct ARQ schemes~\cite{Jamieson_PPR}. Our deeper understanding of fading should also aid attempts to use the faster OFDM rates in challenging outdoor mobile environments~\cite{Eriksson_Cabernet}. %that have previously been hampered. % by an inability to explain or predict performance in a reasonable way.

\section{Transmit power control}
Existing proposals for transmit power control require complex probing and adaptation mechanisms~\cite{Monks_PowerMAC,Ramachandran_Symphony,Son_PowerStudy}. Our %evaluation highlights 
example in \chapref{chap:delivery} suggests 
that, with a good predictive model, we can directly and confidently select a reduced transmit power without degrading link performance. 

%%%%%%%%%%%%%%%%%%%%%%%%%%%%%%%%%%
\ifx\mainfile\undefined
%
% ==========   Bibliography   ==========
%
%\nocite{*}   % include everything in the uwthesis.bib file
\bibliographystyle{plain}
\bibliography{dhalperi_thesis}

\end{document}
\fi
