\ifx\mainfile\undefined
%  ========================================================================
%  Copyright (c) 2006-2011 The University of Washington
%
%  Licensed under the Apache License, Version 2.0 (the "License");
%  you may not use this file except in compliance with the License.
%  You may obtain a copy of the License at
%
%      http://www.apache.org/licenses/LICENSE-2.0
%
%  Unless required by applicable law or agreed to in writing, software
%  distributed under the License is distributed on an "AS IS" BASIS,
%  WITHOUT WARRANTIES OR CONDITIONS OF ANY KIND, either express or implied.
%  See the License for the specific language governing permissions and
%  limitations under the License.
%  ========================================================================
%
 
\documentclass [11pt, twoside] {uwthesis}

\usepackage{color}
\usepackage{url}
\usepackage{amsmath}
\usepackage{amsfonts}
\usepackage[bookmarks,
	hidelinks,
	plainpages=false,
	pdfpagelabels,
	pagebackref=true,
            ]{hyperref}
\renewcommand*{\backref}[1]{}% for backref < 1.33 necessary
\renewcommand*{\backrefalt}[4]{%
  \ifcase #1 %
    (No citations.)%
  \or
    (Cited on page #2.)%
  \else
    (Cited on pages #2.)%
  \fi
}

\newcommand{\biburl}[1]{{\tt<}\url{#1}{\tt>}}

\hypersetup{%
pdfauthor = {Daniel Chaim Halperin},
pdftitle = {Simplifying the Configuration of 802.11 Wireless Networks with Effective SNR},
pdfsubject = {Ph.D. Dissertation},
pdfkeywords = {},
pdfcreator = {University of Washington, Computer Science and Engineering},
pdfproducer = {},
bookmarksopen = {true},
pdfpagelayout = {TwoColumnRight},
}

\usepackage{footnotebackref}
%%%%%%%%%%%%%%%%%%%%%%%%%%%%%%%%%%%%%%%%%%%%%%%%%%%%%%
%%%        Formatting sections                     %%%
%%%%%%%%%%%%%%%%%%%%%%%%%%%%%%%%%%%%%%%%%%%%%%%%%%%%%%
\newcommand{\algref}[1]{Algorithm~\ref{#1}}
\newcommand{\chapref}[1]{Chapter~\ref{#1}}
\renewcommand{\eqref}[1]{Equation~\ref{#1}}
\newcommand{\figref}[1]{Figure~\ref{#1}}
\newcommand{\secref}[1]{\S\ref{#1}}
\newcommand{\tabref}[1]{Table~\ref{#1}}
\newcommand{\heading}[1]{\vspace{4pt}\noindent\textbf{#1}}
\newcommand{\topheading}[1]{\noindent\textbf{#1}}
\newcommand{\noheading}[0]{\vspace{4pt}\noindent}

%%%%%%%%%%%%%%%%%%%%%%%%%%%%%%%%%%%%%%%%%%%%%%%%%%%%%%
%%%        XXX and other warnings                  %%%
%%%%%%%%%%%%%%%%%%%%%%%%%%%%%%%%%%%%%%%%%%%%%%%%%%%%%%
\newcommand{\xxx}[1]{\textit{\color{red}XXX #1}}

%%%%%%%%%%%%%%%%%%%%%%%%%%%%%%%%%%%%%%%%%%%%%%%%%%%%%%
%%%        Units                                   %%%
%%%%%%%%%%%%%%%%%%%%%%%%%%%%%%%%%%%%%%%%%%%%%%%%%%%%%%
\usepackage{xspace}
\newcommand{\unitsep}{\texorpdfstring{\,}{ }}
\def\unit#1{% from: http://www.tex.ac.uk/cgi-bin/texfaq2html?label=csname "Defining a macro from an argument"
  \expandafter\def\csname #1\endcsname{\unitsep\text{#1}\xspace}%
}
\def\varunit#1#2{% from: http://www.tex.ac.uk/cgi-bin/texfaq2html?label=csname "Defining a macro from an argument"
  \expandafter\def\csname #1\endcsname{\unitsep\text{#2}\xspace}%
}
\unit{GHz}
\unit{MHz}
\unit{kHz}
\unit{Gbps}
\unit{Mbps}
\unit{KB}
\unit{dB}
\unit{dBi}
\unit{dBm}
\unit{W}
\unit{mW}
\varunit{uW}{$\mu$W}
\unit{ms}
\varunit{us}{$\mu$s}
\unit{h}
\unit{m}
\unit{s}
\unit{km}
\unit{cm}
\unit{mm}
\varunit{mmsq}{mm$^\text{2}$}
\varunit{insq}{in$^\text{2}$}
\newcommand{\degree}{\ensuremath{^\circ}\xspace}
\newcommand{\degrees}{\degree}
%%%%%%%%%%%%%%%%%%%%%%%%%%%%%%%%%%%%%%%%%%%%%%%%%%%%%%%%%%%%%%%%%%%%%%%%%%%%%%%%%%%%%%
% Euler for math | Palatino for rm | Helvetica for ss | Courier for tt
%
% From: http://www.tug.org/mactex/fonts/LaTeX_Preamble-Font_Choices.html
%%%%%%%%%%%%%%%%%%%%%%%%%%%%%%%%%%%%%%%%%%%%%%%%%%%%%%%%%%%%%%%%%%%%%%%%%%%%%%%%%%%%%%
\renewcommand{\rmdefault}{ppl} % rm
\usepackage[scaled]{helvet} % ss
\usepackage{courier} % tt
\usepackage{eulervm} % a better implementation of the euler package (not in gwTeX)
\normalfont
\usepackage[T1]{fontenc}
%%%%%%%%%%%%%%%%%%%%%%%%%%%%%%%%%%%%%%%%%%%%%%%%%%%%%%%%%%%%%%%%%%%%%%%%%%%%%%%%%%%%%%

%%%%%%%%%%%%%%%%%%%%%%%%%%%%%%%%%%%%%%%%%%%%%%%%%%%%%%
%%%        Figures                                 %%%
%%%%%%%%%%%%%%%%%%%%%%%%%%%%%%%%%%%%%%%%%%%%%%%%%%%%%%
\usepackage{graphicx}
% Caption package both lets you set the spacing between figure and caption
% and also makes the \figref{} point to the right place.
\usepackage[font=bf,aboveskip=6pt,belowskip=-4mm]{caption}
% Allow subfigures, make them bold
\usepackage[bf,BF,small]{subfigure}
% List of figures
\setcounter{lofdepth}{2}  % Print the chapter and sections to the lot

%%%%%%%%%%%%%%%%%%%%%%%%%%%%%%%%%%%%%%%%%%%%%%%%%%%%%%
%%%        Lists with reduced spacing              %%%
%%%%%%%%%%%%%%%%%%%%%%%%%%%%%%%%%%%%%%%%%%%%%%%%%%%%%%
\usepackage{enumitem}

%%%%%%%%%%%%%%%%%%%%%%%%%%%%%%%%%%%%%%%%%%%%%%%%%%%%%%
%%%        Fancy tables                            %%%
%%%%%%%%%%%%%%%%%%%%%%%%%%%%%%%%%%%%%%%%%%%%%%%%%%%%%%
\usepackage{tabulary}
\usepackage{booktabs}

%%%%%%%%%%%%%%%%%%%%%%%%%%%%%%%%%%%%%%%%%%%%%%%%%%%%%%
%%%        Formatting techniques/tools/etc.        %%%
%%%%%%%%%%%%%%%%%%%%%%%%%%%%%%%%%%%%%%%%%%%%%%%%%%%%%%
\newcommand{\term}[1]{\texttt{#1}}

\begin{document}
 
\textpages
\setcounter{chapter}{0} % Set to n-1!
\fi
%%%%%%%%%%%%%%%%%%%%%%%%%%%%%%%%%%

\cleardoublepage
\chapter{Introduction}
\label{chap:intro}

Wireless communication forms the dominant consumer-facing networking technology in the world. Billions of people worldwide use cellular phones to provide global voice communication service, and in the past decade smartphones have begun to provide digital access to the Internet while mobile. These innovations are revolutionizing the way people interact with their devices, with each other, and with the world.

Cellular data networks provide mobile connectivity to the Internet over a long range and at relatively low rates. Complementing these, Wi-Fi (IEEE~802.11~\cite{80211}) is used today in locations such as caf\'{e}s, shopping malls, corporate offices, and homes to provide connectivity at much higher rates over much shorter range. With its low cost, small physical footprint, and dramatically increased speeds (up to 600\Mbps in IEEE~802.11n~\cite{80211n}), Wi-Fi is not limited to traditional computing devices such as laptop and desktop computers, but is also being adopted by consumer electronics such as televisions and DVD players. An ABI Research report~\cite{ABI_Research_2010} from late 2010 forecast that more than half of the 1 billion Wi-Fi chipsets shipped in 2011 would be used in consumer electronics.
%, Wi-Fi today connects devices such as laptops, desktops, tablets, and phones to the Internet at high-speed.
%The latest revision to Wi-Fi, called IEEE~802.11n~\cite{80211n}, enables individual devices to transmit data as fast as 600\Mbps.

Due to its dramatic improvement in recent years and rapid adoption in a diverse set of devices, Wi-Fi is thus at the heart of the next networking revolution: the combining of these diverse consumer devices to build rich applications that leverage each device's unique features rather than simply interacting with the Internet at large. A new protocol, called Wi-Fi Direct~\cite{wifi_direct} was standardized in late 2010 to enable Wi-Fi devices to build networks that better match their applications, and has seen great uptake. Another ABI Research study~\cite{ABI_Research_2011}, conducted in late 2011, measured a 50\% annual growth rate of Wi-Fi direct, and predicts that there will be 2~billion Wi-Fi Direct-enabled devices by 2016.

%The next networking revolution will be for local wireless networks, in the enterprise and especially in the home.
%%, short-range, high-rate local area networks in the
%% enterprise and the
%%home that complement cellular networks.
%, these indoor networks will be used to build rich device-to-device applications in a comparably small area and at much higher rates. Wi-Fi (IEEE 802.11) is the dominant consumer wireless technology, and 
%By both dwarfing the speeds of commercial cellular networks and enabling much denser deployments of many devices that use unlicensed spectrum, Wi-Fi is on the cusp of enabling a new generation of rich device-to-device applications.
%Already, consumer electronics have begun gaining Wi-Fi functionality with dramatic uptake: a 2011 ABI Research report measured a 50\% annual growth rate of devices that support Wi-Fi Direct~\cite{wifi_direct} (a new standard protocol for device-to-device networks) and forecasts that there will be 2~billion Wi-Fi Direct-enabled devices by 2016.
%
%With this shift, the decades-old ``Internet of Things'' vision is becoming a reality, and researchers have begun building systems such as HomeOS~\cite{Dixon_HomeOS} to design, implement, and program applications that run across the devices in a home.
%Now that network-ready consumer devices are available, protocols to connect them have been developed, and systems to control them and help them interoperate are being researched, we have most of the pieces to build the home networks that will support these rich new applications. The last networking component, identified in studies of users who have deployed inchoate versions of home networking systems~\cite{Brush_HomeAutomation}, is a reliable, high-bandwidth wireless network to keep these home devices connected.

%This change is happening fast: an ABI Research report from late 2010 forecast the shipment of 1~billion Wi-Fi (IEEE 802.11~\cite{80211}) chipsets in 2011, with more than half of these chipsets used in consumer electronics, handsets, and other mobile devices. 

Despite these great technology, standardization, and adoption trends towards this potentially bright future, one major factor challenges these systems. While these networks have significant potential for speed, capacity, and reliability, realizing these benefits requires that the network is configured in an efficient and effective way, and that it can adapt rapidly to changing conditions. The issue is that the underlying Wi-Fi technologies and network architectures have become rather complex, and how to configure and control them has become a significant decision problem without a simple, comprehensive solution.

In this thesis, I develop a practical solution to this problem. In particular, I develop a comprehensive metric to inform these complex decision problems using information available in commercial commodity Wi-Fi hardware. Throughout the rest of this chapter, I first explain in further detail the problem (\secref{sec:intro_problem}), then present my approach to solving it (\secref{sec:intro_approach}). I conclude this chapter by discussing the contributions of my work (\secref{sec:intro_contributions}) and the organization of the rest of this thesis (\secref{sec:intro_organization}).

\section{The Problem}
\label{sec:intro_problem}

\section{My Approach}
\label{sec:intro_approach}

\section{Contributions}
\label{sec:intro_contributions}

\section{Organization of This Thesis}
\label{sec:intro_organization}

%%%%%%%%%%%%%%%%%%%%%%%%%%%%%%%%%%
\ifx\mainfile\undefined
%
% ==========   Bibliography   ==========
%
%\nocite{*}   % include everything in the uwthesis.bib file
\bibliographystyle{plain}
\bibliography{dhalperi_thesis}

\end{document}
\fi
