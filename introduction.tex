\ifx\mainfile\undefined
%  ========================================================================
%  Copyright (c) 2006-2011 The University of Washington
%
%  Licensed under the Apache License, Version 2.0 (the "License");
%  you may not use this file except in compliance with the License.
%  You may obtain a copy of the License at
%
%      http://www.apache.org/licenses/LICENSE-2.0
%
%  Unless required by applicable law or agreed to in writing, software
%  distributed under the License is distributed on an "AS IS" BASIS,
%  WITHOUT WARRANTIES OR CONDITIONS OF ANY KIND, either express or implied.
%  See the License for the specific language governing permissions and
%  limitations under the License.
%  ========================================================================
%
 
\documentclass [11pt, twoside] {uwthesis}

\usepackage{color}
\usepackage{url}
\usepackage{amsmath}
\usepackage{amsfonts}
\usepackage[bookmarks,
	hidelinks,
	plainpages=false,
	pdfpagelabels,
	pagebackref=true,
            ]{hyperref}
\renewcommand*{\backref}[1]{}% for backref < 1.33 necessary
\renewcommand*{\backrefalt}[4]{%
  \ifcase #1 %
    (No citations.)%
  \or
    (Cited on page #2.)%
  \else
    (Cited on pages #2.)%
  \fi
}

\newcommand{\biburl}[1]{{\tt<}\url{#1}{\tt>}}

\hypersetup{%
pdfauthor = {Daniel Chaim Halperin},
pdftitle = {Simplifying the Configuration of 802.11 Wireless Networks with Effective SNR},
pdfsubject = {Ph.D. Dissertation},
pdfkeywords = {},
pdfcreator = {University of Washington, Computer Science and Engineering},
pdfproducer = {},
bookmarksopen = {true},
pdfpagelayout = {TwoColumnRight},
}

\usepackage{footnotebackref}
%%%%%%%%%%%%%%%%%%%%%%%%%%%%%%%%%%%%%%%%%%%%%%%%%%%%%%
%%%        Formatting sections                     %%%
%%%%%%%%%%%%%%%%%%%%%%%%%%%%%%%%%%%%%%%%%%%%%%%%%%%%%%
\newcommand{\algref}[1]{Algorithm~\ref{#1}}
\newcommand{\chapref}[1]{Chapter~\ref{#1}}
\renewcommand{\eqref}[1]{Equation~\ref{#1}}
\newcommand{\figref}[1]{Figure~\ref{#1}}
\newcommand{\secref}[1]{\S\ref{#1}}
\newcommand{\tabref}[1]{Table~\ref{#1}}
\newcommand{\heading}[1]{\vspace{4pt}\noindent\textbf{#1}}
\newcommand{\topheading}[1]{\noindent\textbf{#1}}
\newcommand{\noheading}[0]{\vspace{4pt}\noindent}

%%%%%%%%%%%%%%%%%%%%%%%%%%%%%%%%%%%%%%%%%%%%%%%%%%%%%%
%%%        XXX and other warnings                  %%%
%%%%%%%%%%%%%%%%%%%%%%%%%%%%%%%%%%%%%%%%%%%%%%%%%%%%%%
\newcommand{\xxx}[1]{\textit{\color{red}XXX #1}}

%%%%%%%%%%%%%%%%%%%%%%%%%%%%%%%%%%%%%%%%%%%%%%%%%%%%%%
%%%        Units                                   %%%
%%%%%%%%%%%%%%%%%%%%%%%%%%%%%%%%%%%%%%%%%%%%%%%%%%%%%%
\usepackage{xspace}
\newcommand{\unitsep}{\texorpdfstring{\,}{ }}
\def\unit#1{% from: http://www.tex.ac.uk/cgi-bin/texfaq2html?label=csname "Defining a macro from an argument"
  \expandafter\def\csname #1\endcsname{\unitsep\text{#1}\xspace}%
}
\def\varunit#1#2{% from: http://www.tex.ac.uk/cgi-bin/texfaq2html?label=csname "Defining a macro from an argument"
  \expandafter\def\csname #1\endcsname{\unitsep\text{#2}\xspace}%
}
\unit{GHz}
\unit{MHz}
\unit{kHz}
\unit{Gbps}
\unit{Mbps}
\unit{KB}
\unit{dB}
\unit{dBi}
\unit{dBm}
\unit{W}
\unit{mW}
\varunit{uW}{$\mu$W}
\unit{ms}
\varunit{us}{$\mu$s}
\unit{h}
\unit{m}
\unit{s}
\unit{km}
\unit{cm}
\unit{mm}
\varunit{mmsq}{mm$^\text{2}$}
\varunit{insq}{in$^\text{2}$}
\newcommand{\degree}{\ensuremath{^\circ}\xspace}
\newcommand{\degrees}{\degree}
%%%%%%%%%%%%%%%%%%%%%%%%%%%%%%%%%%%%%%%%%%%%%%%%%%%%%%%%%%%%%%%%%%%%%%%%%%%%%%%%%%%%%%
% Euler for math | Palatino for rm | Helvetica for ss | Courier for tt
%
% From: http://www.tug.org/mactex/fonts/LaTeX_Preamble-Font_Choices.html
%%%%%%%%%%%%%%%%%%%%%%%%%%%%%%%%%%%%%%%%%%%%%%%%%%%%%%%%%%%%%%%%%%%%%%%%%%%%%%%%%%%%%%
\renewcommand{\rmdefault}{ppl} % rm
\usepackage[scaled]{helvet} % ss
\usepackage{courier} % tt
\usepackage{eulervm} % a better implementation of the euler package (not in gwTeX)
\normalfont
\usepackage[T1]{fontenc}
%%%%%%%%%%%%%%%%%%%%%%%%%%%%%%%%%%%%%%%%%%%%%%%%%%%%%%%%%%%%%%%%%%%%%%%%%%%%%%%%%%%%%%

%%%%%%%%%%%%%%%%%%%%%%%%%%%%%%%%%%%%%%%%%%%%%%%%%%%%%%
%%%        Figures                                 %%%
%%%%%%%%%%%%%%%%%%%%%%%%%%%%%%%%%%%%%%%%%%%%%%%%%%%%%%
\usepackage{graphicx}
% Caption package both lets you set the spacing between figure and caption
% and also makes the \figref{} point to the right place.
\usepackage[font=bf,aboveskip=6pt,belowskip=-4mm]{caption}
% Allow subfigures, make them bold
\usepackage[bf,BF,small]{subfigure}
% List of figures
\setcounter{lofdepth}{2}  % Print the chapter and sections to the lot

%%%%%%%%%%%%%%%%%%%%%%%%%%%%%%%%%%%%%%%%%%%%%%%%%%%%%%
%%%        Lists with reduced spacing              %%%
%%%%%%%%%%%%%%%%%%%%%%%%%%%%%%%%%%%%%%%%%%%%%%%%%%%%%%
\usepackage{enumitem}

%%%%%%%%%%%%%%%%%%%%%%%%%%%%%%%%%%%%%%%%%%%%%%%%%%%%%%
%%%        Fancy tables                            %%%
%%%%%%%%%%%%%%%%%%%%%%%%%%%%%%%%%%%%%%%%%%%%%%%%%%%%%%
\usepackage{tabulary}
\usepackage{booktabs}

%%%%%%%%%%%%%%%%%%%%%%%%%%%%%%%%%%%%%%%%%%%%%%%%%%%%%%
%%%        Formatting techniques/tools/etc.        %%%
%%%%%%%%%%%%%%%%%%%%%%%%%%%%%%%%%%%%%%%%%%%%%%%%%%%%%%
\newcommand{\term}[1]{\texttt{#1}}

\begin{document}
 
\textpages
\setcounter{chapter}{0} % Set to n-1!
\fi
%%%%%%%%%%%%%%%%%%%%%%%%%%%%%%%%%%

\cleardoublepage
\chapter{Introduction}
\label{chap:intro}

Computer scientists have long envisioned a networked home, in which services and functionality offered by different devices can be easily shared and combined to create a new style of rich applications. Though this vision dates back decades, it has recently been brought closer to reality by the sharp growth in the number and diversity of network-enabled consumer devices. A key component of this change has been the availability of cheap wireless connectivity to enable flexible and convenient deployment of devices.

The leading consumer wireless technology today---and for the foreseeable future---is IEEE 802.11~\cite{80211}, also called Wi-Fi. An ABI Research report from late 2010 forecast the shipment of 1~billion Wi-Fi (IEEE 802.11~\cite{80211}) chipsets in 2011, with more than half of these chipsets used in consumer electronics, handsets, and other mobile devices. Wi-Fi's popularity can be attributed to its low cost, small physical footprint, and dramatically increased speeds that can now reach up to 600\Mbps with the 2009 update to IEEE 802.11n~\cite{80211n}. These attributes bring the vision of rich home applications close to being realized.

However, one major impediment stands in the way of this vision: the Wi-Fi networks in use today are not architected to support these applications. In particular, three trends of the future wireless home will push today's home networks beyond their limits:

\begin{enumerate}
\item Home wireless networks will become increasingly \emph{dense}. Whereas today the use of Wi-Fi scales in the number of occupants---one access point plus some laptops, tablets, and smartphones---tomorrow, it will scale with the number of objects. Rather than a handful of wireless devices per home, there will be five, ten, or hundreds of devices per room.
%
\item Consumer devices will be increasingly \emph{heterogeneous} in their wireless capabilities. For instance, the number of antennas a device has determines its maximum rate (for IEEE 802.11n), and the available spectrum depends on whether it has a single-band or dual-band radio. These factors are determined independently by manufacturers as a function of size, workload, and profit margin. In three contrasting examples: (1) Small devices like the $\approx$1\insq Eye-Fi SD cards~\cite{EyeFi} may not fit multiple antennas; (2) an HDTV has room for many antennas and the demand for high speeds; and (3) printers are sold at such low margins that they may only support 2.4\GHz bands and single antenna systems despite their large form factor. Devices may also take on different roles in the network depending on whether they are fixed or mobile, or using battery or wall power.
%
\item Communication patterns are tending towards \emph{device-to-device}. The defining characteristic of this new generation of applications is that any functionality offered by any device, such as input, output, actuation, computation, storage, or data, should be accessible via the network by any other device. Access to the Internet is no longer the primary purpose of wireless connectivity; the Internet is now only one service of many.%
\footnote{Though there is also a shift towards the Cloud, applications will still include content generated in the home, such as live video, and can be more interactive using local resources, such as computation~\cite{Satya_Cloudlets}.}
\end{enumerate}

To scale in order to handle these trends, a wireless network should make efficient use of all of its resources. For instance, 

The ``access point'' (AP) model used predominantly in home wireless networks today will simply not scale in the face of these trends. AP networks are designed to permit only a single ongoing transmission network-wide, 

%%%%%%%%%%%%%%%%%%%%%%%%%%%%%%%%%%
\ifx\mainfile\undefined
%
% ==========   Bibliography   ==========
%
%\nocite{*}   % include everything in the uwthesis.bib file
\bibliographystyle{plain}
\bibliography{dhalperi_thesis}

\end{document}
\fi
