\section{Dataset}
To thoroughly evaluate these applications in indoor Wi-Fi networks, I took comprehensive RF measurements in between static 24 devices in my testbed at the University of Washington. 
The IWL5300 devices in the testbed can operate on 35 channels, each 20\MHz wide, defined by IEEE 802.11 standard. Of these 35 channels, 11 overlap in the 83-MHz wide unlicensed 2.4\GHz band, and the remaining 24 non-overlapping channels are spread across three non-contiguous bands between 5.170\GHz and 5.835\GHz.

For the results presented in this section, I measured the physical RF channel (using both RSS and CSI) and actual packet delivery using each of the 24 IEEE 802.11n modulation and coding schemes (\mcs{0}--\mcs{23}) that use  1-, 2-, or 3-streams. I measured these data for all links between 24 testbed nodes at UW\@. Each sender transmits packets with random payloads, sending a total of 2400 packets by interleaving 100 packets from each of the 1-, 2-, or 3-stream 802.11n rates. Each receiver uses 3 antennas for spatial diversity and/or spatial multiplexing. From these data, we can compute the measured throughput $T$ for a particular link~$\ell$ and \mcs{$m$} as
\begin{equation}
	\label{eq:prr_throughput}
	T(\ell,m) = \frac{P(\ell,m)}{100} \cdot B(m),
\end{equation}
where $P(\ell,m)$ is the number of packets delivered on link $\ell$ at \mcs{$m$} and $B(m)$ is the raw bitrate of \mcs{$m$}.