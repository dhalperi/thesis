\ifx\mainfile\undefined
\input{chapter_head}
\setcounter{chapter}{2} % Set to n-1!
\fi
%%%%%%%%%%%%%%%%%%%%%%%%%%%%%%%%%%

\cleardoublepage
\chapter{Problem and Approach}
\label{chap:problem}
\label{chap:approach}

According to the standards makers, one of the intended purposes of the RSSI is ``to aid in link optimization algorithms such as roaming decisions''~\cite[\S 19.9.5.10]{80211}.

\section{Motivation: Inadequacy of RSSI}
\begin{figure}[t]
	\centering
		\subfigure[A wired 802.11n link with variable attenuation has a predictable relationship between SNR and packet reception rate (PRR) and clear separation between rates.]{
			\label{fig:snr_prr_attenuator}
			\includegraphics[width=\textwidth,viewport=13 0 364 204,clip]{figures/esnr/embed_attenuator_snr_prr.pdf}
		}

      	\subfigure[Over real wireless channels in our testbeds, the transition region varies up to 10\dB. This loses the clear separation between rates (and so only three rates are shown for legibility).]{
			\label{fig:snr_prr_26_65}
			\includegraphics[width=\textwidth,viewport=2 0 217 124,clip]{figures/esnr/embed_scatterplot_meas_snr_small.pdf}
		}
	\caption[Packet delivery over wired and real 802.11 channels]{\label{fig:rssi_predictions}Measured (single antenna) 802.11n packet delivery over wired and real channels.}%
\end{figure}
\topheading{Packet Delivery versus RSSI/SNR\@.}
Textbook analyses of modulation schemes give delivery probability for a single signal in terms of the signal-to-noise (SNR) ratio~\cite{Goldsmith}, %The SNR is defined as the ratio of the signal power to the thermal noise power, typically expressed on a logarithmic scale in decibels, i.e., SNR = $10\log_{10}(S/N)$. 
typically expressed on a log scale in decibels.
This model holds for narrowband channels with additive white Gaussian noise. It predicts a sharp transition region of 1--2\dB over which a link changes from extremely lossy to highly reliable. This makes the SNR a valuable indicator of performance.

We generated performance curves using SNR for the \program{iwl5300} over a simple wired link with a variable attenuator and for a single transmit and receive antenna. The result is shown for all single antenna 802.11n rates in \figref{fig:snr_prr_attenuator}. 
We observe a characteristic sharp transition region for packet reception rate (PRR) versus SNR\@. This is despite the relatively wide 20\MHz channel, 56 OFDM subcarriers, coding and other bit-level operations. This is the behavior we want from a link metric in order to predict packet delivery.

In contrast, packet delivery over real wireless channels does not exhibit the same picture. \figref{fig:snr_prr_26_65} shows the measured PRR versus SNR for three sample rates (6.5, 26, and 65\Mbps) over all wireless links in our testbeds, using the same 802.11n NICs. The SNR of the transition regions can exceed 10\dB, so that some links easily work for a given SNR and others do not. There is no longer clear separation between rates. This is consistent with other reported measurements that show RSSI does not predict packet delivery for real links~\cite{aguayo_roofnet, Reis_sigcomm06, snr_infocom08, zhao_sensys03}.

\begin{figure}[t]
  \centering
  \includegraphics[width=\columnwidth,viewport=2 9 185 108,clip]{figures/esnr/embed_fsf-shape-two-links.pdf}
% viewport=2 10 170 108
  \caption{Channel gains on four links that perform about equally well at 52\Mbps. The more faded links require larger RSSIs (i.e., more transmit power) to achieve similar PRRs.}
  \label{fig:example_fsf_shape}
  % information for the links used to make above plot: 
  %srcs = [1 10 3 3];
  %dests = [9 11 2 5];
  %txpowers = [-4 20 28 20];

  % reference numbers from expt-8
  %prr = [80 83 78 74];
  %rss = [16.5 30.2 27.1 18.2];
\end{figure}

\heading{Example: RSSI vs CSI\@.}
To contrast CSI and RSSI, \figref{fig:example_fsf_shape} shows packet delivery rates, RSSIs, and subcarrier-level CSI for four links that exhibit similar performance when using the 802.11n single-stream 52\Mbps rate. Multipath causes some subcarriers to work markedly better than others although all use the same modulation and coding. These channel details, and not simply the overall signal strength as given by RSSI, affect packet delivery. The fading profiles vary significantly across the four links. One distribution is quite flat across the subcarriers, while the other three exhibit frequency-selective fading of varying degrees. Two of the links have two deeply-faded subcarriers that are more than 20\dB down from the peak.

These links harness the received power with different efficiencies.
The more faded links are more likely to have errors that must be repaired with coding, and require extra transmit power to compensate. Thus, while the performance is roughly the same, the most frequency-selective link needs a much higher overall packet SNR~(30.2\dB) than the frequency-flat link (16.5\dB). This difference of almost 14\dB highlights why RSSI-based SNR does not reliably predict performance. Fading and its effects are well-known. However, it is rare to see data that shows fading for real links and NICs because it has been difficult to measure.

\section{Effective SNR Model for 802.11n}
The example of \figref{fig:example_fsf_shape} indicated that the extra information in CSI can help explain performance for real wireless links. Here, we develop a model that can accurately predict the packet delivery probability of commodity 802.11 NICs for a given physical layer configuration operating over a given channel. We want our model to be simple and practical, so that it can be readily deployed, and to cover a wide range of physical layer configurations, so that it can be applied in many settings and for many tasks. In particular, the scope of our model is 802.11n including multiple antenna modes, including OFDM and MIMO\@. This scope is sufficient for many current and future networks. In our preliminary work, we have modeled delivery for single packet transmission only.

\heading{Model Design.}
The structure of our model is simple: given 1) a current CSI measurement of the RF channel between transmitter and receiver, and 2) a target physical layer configuration of the transmit and receive NICs, it predicts whether that link will reliably deliver packets in that configuration.
With this simple decision primitive, we can easily build higher layer optimization protocols. These include selecting the best rate, number of spatial streams, or transmit antenna set; whether to use 20\MHz or the entire 40\MHz channel; or choosing the lowest transmit power at which the link supports a particular rate.

For the model output, we define that the link will work, i.e., will reliably deliver packets, if we predict $\geq$90\% packet reception rate. We do not try to make predictions in the transition region during which a link changes from lossy to reliable. Predictions there are likely to be variable, and simply knowing when the link starts to work is useful information in practice.

%%%%%%%%%%%%%%%%%%%%%%%%%%%%%%%%%%
\ifx\mainfile\undefined
%
% ==========   Bibliography   ==========
%
%\nocite{*}   % include everything in the uwthesis.bib file
\bibliographystyle{plain}
\bibliography{dhalperi_thesis}

\end{document}
\fi
